%%*****************************************************************************
%% $Id$
%%*****************************************************************************
%% Author: Gerd Neugebauer
%%-----------------------------------------------------------------------------
\documentclass[11pt]{scrartcl}

\usepackage[latin1]{inputenc}
\usepackage[T1]{fontenc}
\usepackage{shortvrb}

\providecommand*{\ExTeX}{%
  \textrm{% das Logo grunds�tzlich serifenbehaftet
    \ensuremath{\textstyle\varepsilon_{\kern-0.15em\mathcal{X}}}%
    \kern-.15em\TeX
  }%
}
\MakeShortVerb|

\begin{document}%%%%%%%%%%%%%%%%%%%%%%%%%%%%%%%%%%%%%%%%%%%%%%%%%%%%%%%%%%%%%%%

\title{\ExTeX\ -- Design Guidelines}
\author{Gerd Neugebauer}
\date{}
\maketitle

\begin{abstract}
  
\end{abstract}

\section{Objectives}

The aim of \ExTeX\ is the implementation of a typesetting system based
on the ideas of \TeX.

\begin{itemize}

\item \ExTeX\ should reproduce as good as possible the results of \TeX:
  the visual appearance of a processed document in compatibility mode should
  be indistinguishable.

  This does not mean that the produced files (dvi, pdf, aux, log) are the
  identical to those produced by \TeX.

\item \ExTeX\ should be configurable and extendable. Thus it can serve as a
  testbed for typesetting experiments. The default configuration reflects the
  evolving state of the art. 

\item The implementation language is Java. Thus the features of this language
  should be used, e.g. the Unicode character set.

\end{itemize}



\section{Code formatting}

The Coding Standards \cite{coding-conventions} od Sun have been become
the de facto standard in the context of Java programming. Thus these
rule are used as basis for this project as well.

Any additions or overwriting of rules given there will be documented
here.


\section{Design Rationals}

\subsection{Java 1.4}

\ExTeX\ is assumed to run on Java 1.4 and upwards. Thus the features
of this release -- which has been the current one at the beginnig of
the development -- can be used. No provisions have to be made to
support any version of Java prior to Version 1.4.


\subsection{No Static Class Variables}

Static class variables act like global variables. In the context of
object-oriented programming it is considered as bad design to use
them. Thus each static variable has to be justified extensively.

\subsection{No System.exits() Outside of main()}

The method |System.exit()| terminates the execution of the programn
and returns an exit status back to the operating system. Invoking this
method somewhere in code makes it impossible to easily test the code
from within Java (e.g. with JUnit)

The method |System.exit()| is allowed in the method |main()| only.


\bibliographystyle{alpha}
\bibliography{Design_Guidelines}

\end{document}%%%%%%%%%%%%%%%%%%%%%%%%%%%%%%%%%%%%%%%%%%%%%%%%%%%%%%%%%%%%%%%%%
%
% Local Variables: 
% mode: latex
% TeX-master: nil
% End: 
