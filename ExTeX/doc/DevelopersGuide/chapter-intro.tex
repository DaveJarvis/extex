%%*****************************************************************************
%% $Id$
%%*****************************************************************************
%% @author Gerd Neugebauer
%%-----------------------------------------------------------------------------
\chapter{Introduction}

\ExTeX{} aims at providing a high-quality typesetting system. The
development of \ExTeX\ has been inspired by the experiences with \TeX.
The focus lies on an open design and a high degree of configurability.
Thus \ExTeX\ should be a good base for further development.

On the other hand we have to take care not to leave the current user
base of \TeX\ behind. pdf\TeX\ has taught us that a migration path
from \TeX\index{TeX@\TeX} has a positive value in it. In the mean time
the majority of \TeX\ users applies in fact
pdf\TeX\index{pdfTeX@pdf\TeX}.

To provide a backward compatibility of \ExTeX\ with
\TeX\index{TeX@\TeX} one special configuration is provided. Thus
backward compatibility is just a matter of configuration.


\section{Audience}

This document is meant for developers and those interested in the
sources and development processes of \ExTeX. It should contain all
information for getting started quickly.


\section{Mailing Lists}

If you are ready to try \ExTeX{} you might as well want to join a
mailing list to get in contact with the community. The following
mailing lists might be of interest:

\begin{description}
\item[extex@dante.de] \ \\
  A general mailing list about \ExTeX. It has low traffic and is
  mainly in German. Subscribe and unsubscribe via the Web form
  \url{http://www.dante.de/listman/extex}.

\item[extex-eng@dante.de] \ \\
  A general mailing list about \ExTeX. It has currently very low
  traffic and is in English. Subscribe and unsubscribe via the Web
  form \url{http://www.dante.de/listman/extex-eng}.

\item[extex-devel@dante.de] \ \\
  A mailing list for the exchange of the developers of \ExTeX. It has
  low traffic and is partly in German. Subscribe and unsubscribe via
  the Web form \url{http://www.dante.de/listman/extex-devel}.

\item[extex-cvs@list.berlios.de] \ \\
  A mailing list for automatic notification about changes in the CVS
  repository of \ExTeX. It is not meant to post mails on this list.
  This list is not archived. Subscribe and unsubscribe via the Web
  form \url{https://lists.berlios.de/mailman/listman/extex-cvs}. You
  need to be logged in at Berlios when registering.

\item[extex-bugs@list.berlios.de] \ \\
  A mailing list for bug messages of \ExTeX. Subscribe and unsubscribe
  via the Web form
  \url{http://lists.berlios.de/mailman/listman/extex-bugs}.
\end{description}


\section{Organizational Agreements}

The developers of \ExTeX\ have agreed on some rule for cooperation.
Those rules are documented here.

\subsection{Language}

The official project language for \ExTeX\ is English in the US
dialect. The sources are documented in this language and the major
documents are written in this language.

Since some of the developers are German this language might slip in
during intensive discussions.


\subsection{Maintainers of Files}

Each file has a single maintainer -- even if there are several
authors. The maintainer has to be informed and has to agree on any
changes in the file. The maintainership is usually indicated in the
Java sources with the help of te tag \texttt{@author}. The first
author is always the maintainer.

Changes to a file can be carried out by the maintainer or delegated to
somebody else. The maintainer can change if both the old and new
maintainer agree in this.

