%%*****************************************************************************
%% $Id$
%%*****************************************************************************
%% Author: Michael Niedermair
%%-----------------------------------------------------------------------------


\chapter{Fonttypen}

F�r die Verwendung von Fonts ben�tigt \ExTeX\ eine eigene Fontinformationsdatei (Dateieinung \emph{.emf}) im XML-Format, welche genauere Informationen �ber den Font enth�lt.

Dies enth�lt Informationen �ber den Fonttype (\TeX-Font, PS-Type1-Font, True Type Font, Open Type Font, \dots), die Metriken (Breite eines Zeichens \dots), das Encoding und viele weitere Informationen .

\section{\TeX-Font (TFM)}
\index{tfm}\index{Font!tfm}

Wird in \TeX\ ein Font angefordert, so wird in der \emph{map}-Datei\footnote{meist psfonts.map} nach dem Fontnamen gesucht.

\begin{lstlisting}
...
antpr AntykwaPoltawskiego-Regular "encantp ReEncodeFont" <antp.enc <antpr.pfb
...
\end{lstlisting}

Die Saplten haben dabei folgende Bedeutung:
\begin{itemize}
 \item \textbf{antpr}\\
       Der \TeX-Fontname
 \item \textbf{AntykwaPoltawskiego-Regular}\\
       Der PostScript Fontname.
 \item \textbf{encantp ReEncodeFont}\\
       PS-Eintrag.
 \item \textbf{<antp.enc} (optional)\\
       Der zu verwendende Encoding-Vector. Wird hier nichts angegeben, so wird
       das Standard-Encoding, welches im Font festgelegt ist, verwendet.
 \item \textbf{<antpr.pfb} (optional)\\
       Die dazuge�rige \emph{pfb}-Datei. Bei pdf\LaTeX\ ist hier auch
       eine \emph{ttf}-Datei m�glich. Wird hier nichts angegeben, so handelt es sich um
       einen der Standard-PostScript-Fonts\footnote{Die M�glichkeiten von \emph{metafont} werden hier nicht ber�cksichtigt.}.
\end{itemize}

Zu jedem \TeX-Font gibt es eine \emph{tfm}-Datei, welche die Metriken, die Ligaturen, Kerniginformationen und weitere Daten enth�lt.

Den Font selbst (PS-Type1 Font) wird dabei von \TeX\ nicht ben�tigt\footnote{Ausnahme ist hier pdf\TeX, welches der PS-Type1 Font und auch TTF direkt in das PDF-Dokumetn einbindet}.

TODO \emph{vf}-Dateien.

\section{PS-Typ1-Font (AFM/PFB)}
\index{afm}\index{Font!afm}
\index{pfb}\index{Font!pfb}

Der Ps-Type1-Font besteht aus zwei Dateien. Die \emph{afm}-Datei, welche die Metriken, die Ligaturen und die Kerninginformationen enth�lt und die \emph{pfb}-Datei\footnote{oder auch eine \emph{pfa}-Datei}, welche die Informationen �ber die Glyphen etc.\, enth�lt.

In der \emph{afm}-Datei wird das Standard-Encoding f�r den Font festgelegt. Soll ein anderes Encoding verwendet werden, so muss ein Encoding-Vector verwendet werden.

\section{True Type Font (TTF)}
\index{ttf}\index{Font!ttf}

Der \emph{ttf}-Font enth�lt alle Informationen in einer Datei. Um ein Zeichen zu verwenden, muss zuerst das entsprechende Encoding des Fonts ausgew�hlt werden.
Die meisten \emph{ttf}-Fonts besitzen mindestens zwei Encodings. Erstens ein Enoding f�r Apple Systeme (8-bit f�r 256 Glyphen) und ein Endoing f�r Windows Systeme (16-bit f�r max. 65535 Glpyhen). Zus�tzlich k�nnen weitere Encodings definiert sein.

In der Encoding-Tabelle (CMAP) werden die Zeichen mit einem Glpyhen in Verbindung gesetzt.


\section{Open Type Font (OTF)}

TODO

\section{\ExTeX-Font (EFM}
\index{efm}\index{Font!efm}




\endinput
