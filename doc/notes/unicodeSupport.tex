\documentclass{scrartcl}

\usepackage{booktabs}

\providecommand*{\ExTeX}{%
  \textrm{% das Logo grunds�tzlich serifenbehaftet
    \ensuremath{\textstyle\varepsilon_{\kern-0.15em\mathcal{X}}}%
    \kern-.15em\TeX
  }%
}

%\usepackage[T1]{fontenc}

\begin{document}%%%%%%%%%%%%%%%%%%%%%%%%%%%%%%%%%%%%%%%%%%%%%%%%%%%%%%%%%%%%%%%

\title{Considerations on Unicode~Font~Support for~\ExTeX}

\author{Gerd Neugebauer}
\date{}
\maketitle

\begin{abstract}
  Unicode support has been introduced for the \TeX\ world with Omega.
  Now has come the time in the implementation of \ExTeX\ to explore
  the consequences of the decision to support Unicode.
\end{abstract}

\section{\ExTeX}

The \ExTeX\ project has bee started with the aim to provide a
typesetting system based on the ideas of \TeX. On one side the
software design of \TeX\ can nowadays no longer be considered as state
of the art. On the other side the virtues of \TeX\ concerning the
qualities of typesetting are mostly unreached.

%
% Decisions which were reasonable at the time



\section{Unicode and \ExTeX}

There are different aspects concerning Unicode compatibility:

\begin{itemize}
\item use a 32 bit representation for internal addressing
\item composite characters (base character combined with an arbitrary
number of combining characters)
\item Unicode semantics on glyphs
\item normalization
\item bidirectional properties of glyphs
\item sorting (not yet relevant)
\end{itemize}

We do have two main goals:

\begin{itemize}
\item Unicode compatibility (make aspects as quoted above available to
the user and \ExTeX{} internals)
\item backward compatibility
\end{itemize}


\section{Underlying Data Structures}

The basic data structures in \ExTeX\ are the following ones

\begin{description}
\item[Token] \ \\
  The tokens are produced by the scanner. Tokens carry a category --
  expressed as category code -- and a character. An escape token
  carries an additional name.\footnote{In contrast to \TeX\ an escape
  token also carries a character -- the escape character of the
  control sequence.}
\item[Node] \ \\
  Te node is the basic data structure passed to the typesetter and
  processed there. The node type relevant for Unicode support is the
  CharNode. A CharNode carries a character and the font to typeset it
  in.
\end{description}

In \ExTeX\ the underlying character is not only 8 bit wide as in \TeX.
It has at least a size of 32 bit. To distinguish the two cases we will
call this underlying character a Unicode character.

\section{Unicode Characters}

The name Unicode character has another implication beside the size of
32 bit. This implication is that the characters have a defined
semantics. This means that the characters are more than mere code
points. For instance is the Unicode character with the code point 65
the uppercase Latin letter A. No freedom for interpretation is left in
this case.

In addition to the code points with a fixed semantic some areas are
left undefined in the Unicode standard. Some of them are reserved for
use in applications. Those areas are called public areas.

\section{Impacts on the Scanner}

The scanner is assumed to produced tokens based on Unicode characters.
Thus it needs to know or assume the encoding of the external data
source.


\section{Impacts on the Fonts}

\subsection{Encoding of Fonts}

The fonts have to provide information for Unicode characters. This
means that some re-encoding is necessary to work with \TeX\ encoded
fonts. 

Whenever a font is loaded the font subsystem has to know how the
external font is encoded. For instance the font cmr10 in OT1 encoding
contains mainly the usual characters in the ASCII range. These
characters are identical to those of the Unicode encoding.

In the lower range with the code points 0 to 32 and the upper range
123 to 127 the encoding differs. In figure~\ref{fig:mapping} you can
see how these characters can be mapped to corresponding Unicode
positions. Note that the Unicode points are given hexadecimal while
the code is in decimal.
\begin{table}[htbp]
  \caption{Mapping for cmr10 to Unicode}\label{fig:mapping}\bigskip

  \centering\footnotesize
  \begin{tabular}{rclc}\toprule
  Code& Char    & Description & Unicode point\\\midrule
   0  & \char0  & Greek capital letter gamma   & "0393 \\
   1  & \char1  & Greek capital letter delta   & "0394 \\
   2  & \char2  & Greek capital letter theta   & "0398 \\
   3  & \char3  & Greek capital letter lambda  & "039B \\
   4  & \char4  & Greek capital letter xi      & "039E \\
   5  & \char5  & Greek capital letter pi      & "03A0 \\
   6  & \char6  & Greek capital letter sigma   & "03A3 \\
   7  & \char7  & Greek capital letter upsilon & "03A5 \\
   8  & \char8  & Greek capital letter phi     & "03A6 \\
   9  & \char9  & Greek capital letter psi     & "03A8 \\
  10  & \char10 & Greek capital letter omega   & "03A9 \\
  11  & \char11 & Latin small ligature ff      & "FB00 \\
  12  & \char12 & Latin small ligature fi      & "FB01 \\
  13  & \char13 & Latin small ligature fl      & "FB02 \\
  14  & \char14 & Latin small ligature ffi     & "FB03 \\
  15  & \char15 & Latin small ligature ffl     & "FB04 \\
  16  & \char16 & Latin small letter dotless i & "0131 \\
  17  & \char17 & Latin small letter dotless j &  ??? \\
  18  & \char18 & Modifier letter grave accent & "02CB \\
  19  & \char19 & Modifier letter acute accent & "02CA \\
  20  & \char20 & Caron                        & "02C7 \\
  21  & \char21 & Breve                        & "02D8 \\
  22  & \char22 & Modifier letter macron       & "02C9 \\
  23  & \char23 & Ring above                   & "02DA \\
  24  & \char24 & Cedille                      &  ??? \\
  25  & \char25 & Latin small letter sharp s   & "00DF \\
  26  & \char26 & Latin small letter ae        & "00E6 \\
  27  & \char27 & Latin small letter oe        & "0152 \\
  28  & \char28 & Latin small letter o slash   & "00F8 \\
  29  & \char29 & Latin capital ligature AE    & "00C6 \\
  30  & \char30 & Latin capital ligature OE    & "0153 \\
  31  & \char31 & Latin capital letter O slash & "00D8 \\
  32  & \char32 & ...                          &  ??? \\

  33  & \char33 & & \\
      &         & Latin characters and punctation & \\
  122 & \char122& & \\

  123 & \char123& En dash                      & "2013 \\
  124 & \char124& Em dash                      & "2014 \\
  125 & \char125& Hungarian Umlaut             &  ??? \\
  126 & \char126& Tilde accent                 &  ??? \\
  127 & \char127& Umlaut accent                &  ??? \\\bottomrule
  \end{tabular}
\end{table}

Whenever the font cmr10 is loaded it has to be re-encoded on the fly.
Thus internally the font can always be treated as Unicode encoded.

If the font contains characters which are not defined in the Unicode
standard then these characters have to be placed in the private area.
\ExTeX\ uses the private area starting with "EE00.



\subsection{Font Dimens}

Usually the font dimens are numbered in \TeX. In \ExTeX\ the key can
be an arbitrary String.


\subsection{Scalable Characters}

Finetuning the typography can be achieved with slight variations of
the width of characters. This idea has been implemented in pdf\TeX.
There this feature has been achieved with the help of a number of
different fonts.

To improve the efficiency the scalability of characters could be
incorporated into the underlying font mechanism. A simple solution
would be to allow glue values instead of dimen valued for width,
height and depth.


\subsection{Scalable Math Characters}

One instance where scalable characters are used in \TeX\ are the
various sizes of parentheses and math operators. It seems to be
archaric to compose those characters from small pieces and allow for
discontinuities in the interpolation. Here a potential for improvement
shows up.


\subsection{Mathematical Fonts}

For \TeX\ Donald Knuth has discovered that a font with 256 characters
is not enough for math typesetting. Thus math families have been
introduced. In the world of Unicode fonts all the mathematical
characters available in \TeX\ -- and some more -- are defined in the
Unicode encoding already. Thus it should be considered to use a single
Unicode font instaed of a math family.

One issue here is that certain parameters are stored in the fontdimen
values of distinct fonts. In figure~{tab:mathdimensions} we have
compiles the fontdimen indices for parameters in the different \TeX\
fonts types. Fortunately the overlapping range have the same values

The combined math Unicode font can provide the math parameters as
named font dimens.

To make use of the new math fonts the file \texttt{plain.tex} has to
be adapted.

\begin{table}[htp]
  \caption{Font Dimensions for Math Typesetting}\label{tab:mathdimensions}\bigskip

  \centering
  \small
  \begin{tabular}{lccc}\toprule
    Name		&\TeX&\TeX\ Math&\TeX\ Extension\\\midrule
    slant		&  1 &  1 &  1 \\
    space		&  2 &  2 &  2 \\
    stretch		&  3 &  3 &  3 \\
    shrink		&  4 &  4 &  4 \\
    xheight		&  5 &  5 &  5 \\
    quad		&  6 &  6 &  6 \\
    extraspace		&  7 &  7 &  7 \\
    num1		&  . &  8 &  . \\
    num2		&  . &  9 &  . \\
    num3		&  . & 10 &  . \\
    denom1		&  . & 11 &  . \\
    denom2		&  . & 12 &  . \\
    sup1		&  . & 13 &  . \\
    sup2		&  . & 14 &  . \\
    sup3		&  . & 15 &  . \\
    sub1		&  . & 16 &  . \\
    sub2		&  . & 17 &  . \\
    supdrop		&  . & 18 &  . \\
    subdrop		&  . & 19 &  . \\
    delim1		&  . & 20 &  . \\
    delim2		&  . & 21 &  . \\
    axisheight		&  . & 22 &  . \\
    defaultrulethickness&  . & .  &  8 \\
    bigopspacing1	&  . & .  &  9 \\
    bigopspacing2	&  . & .  & 10 \\
    bigopspacing3	&  . & .  & 11 \\
    bigopspacing4	&  . & .  & 12 \\
    bigopspacing5	&  . & .  & 13 \\\bottomrule
  \end{tabular}
\end{table}


\subsection{Backward Compatibility}

The font knows which characters have been re-encoded. For backward
compatibility this information is needed by some primitives. The
central method of a \texttt{Font} is the one to get access to a glyph:

\begin{quotation}
  \texttt{Glyph} \texttt{getGlyph}\texttt{(} \texttt{UnicodeChar c} \texttt{);}
\end{quotation}

In addition the following methods are needed:
\begin{quotation}
  \texttt{Glyph} \texttt{getGlyph}\texttt{(} \texttt{int code} \texttt{);}
\end{quotation}
This method takes as argument a \TeX\ code in the range from 0 to 255.
It returns the corresponding glyph when the used encoding is applied.
As usual \texttt{null} is returned when the glyph is not defined.

For instance the argument \texttt{getGlyph(0)} returns the same result
as \texttt{getGlyph(new UnicodeChar(0x0393))} when invoked on cmr10.

\begin{quotation}
  \texttt{UnicodeChar} \texttt{getChar}\texttt{(} \texttt{int code} \texttt{);}
\end{quotation}
This method takes as argument a \TeX\ code in the range from 0 to 255.
It returns the corresponding Unicode character when the used encoding
is applied.



\section{Impacts on the Interpreter}

\subsection{The Back-Tick}

The back-tick can be used to translate an input character into a
number. Since this construction deals with the input encoding only
nothing has to done.

Consider an input encoding which allows a direct inut of the character
$\Gamma$. Then the following input should produce the Gamma on the
page:
\begin{quotation}
  \texttt{\textbackslash char`\char0}
\end{quotation}

This might open the way to a minor incompatibility for the characters
32 and 123 to 127.


\subsection{The Primitive \textbackslash char}

The primitive \verb|\char| sends a character node to the typesetter.
The character code stored in the character node has to be remapped. Thus
\verb|getChar()| has to be applied here.


\subsection{The Primitive \textbackslash chardef}

The primitive \verb|\chardef| allows to define a control sequence
which can be used instead of a single character. Internally the code is
stored. For a great part of the fonts this makes no difference since
the \TeX\ encoding (based on ASCII) and Unicode coincide here.

Minor incompatibilities might show up in the less interesting parts.


\subsection{The Primitive \textbackslash catcode}

The primitive \verb|\catcode| sets and queries the mapping of category
codes for the mapping from external characters to tokens. Here
especially the distigtion of the category ``letter'' and the category
``other'' is interesting. In \TeX\ this mapping is initialised
sparsely. In \verb|plain.tex| the missing category codes are provided.
This is a feasible way for 128 or even 255 character codes. It is not
the right way for thousands of Unicode characters.

Unicode characters have an inherent meaning as letter or other
character. Initially the meaning of \verb|\catcode| should reflect
this semantics.


\subsection{The Primitive \textbackslash escapechar}

The count register \verb|\escapechar| contains the escape character.
It is used to print control sequences. This count register is not
affected by any compatibility issue.


\subsection{The Primitive \textbackslash hyphenchar}

The primitive \verb|\hyphenchar| gives reading and writing access to
the hyphenation character of a font. In \TeX\ this defaults to 45
which is the code of the Minus sign in ASCII -- and Unicode. Thus
nothing as to be done.


\subsection{The Primitive \textbackslash lccode}

The primitive \verb|\lowercase| translates a sequence of characters to
their lowercase counterpart. This translation is controlled by the
array \verb|\lccode|. Like for catcodes it is feasible to initialise the
values from the Unicode semantics instead of relying on
\verb|plain.tex| to initialise a huge array.


\subsection{The Primitive \textbackslash mathchar}

The primitive \verb|\mathchar| plays a similar role in math mode like
\verb|\char| in text mode.
Here the compensation for math families and a Unicode math font is
located. 


\subsection{The Primitive \textbackslash mathchardef}

The primitive \verb|\mathchardef| can be used to define mathemaical
characters like \verb|\char| does for text characters.

*** Implications? ***

Maybe the distiction of \verb|\char| and \verb|\mathchar| vanishes
with the use of Unicode fonts.


\subsection{The Primitive \textbackslash sfcode}

The values of \verb|\sfcode| influences the spacing. The value for
uppercase letters is initialized to 999 whereas the other characters
have a value of 1000. For a better typography it might be worthwhile to
initialise the values for all uppercase letters. 


\subsection{The Primitive \textbackslash skewchar}

The primitive \verb|\skewchar| gives reading and writing access to the
skew character of a font. In \TeX\ this defaults to -1 which is the
code for undefined. Nothing as to be done.


\subsection{The Primitive \textbackslash uccode}

The primitive \verb|\uppercase| translates a sequence of characters to
their uppercase counterpart. This translation is controlled by the
array \verb|\uccode|. Like for catcodes it is feasible to initialise the
values from the Unicode semantics instead of relying on
\verb|plain.tex| to initialise a huge array.


\subsection{The Primitive \textbackslash uchyph}

The count register \verb|\uchyph| controls whether or not words
starting with an uppercase letter are hyphenated. All actions are
already performed for \verb|\uccode|. Thus nothing else has to be
done. 


\section{Impact on the Back-Ends}

The features of fonts required for the back-ends heavily depends on
the output formats produced. At least \ExTeX\ needs a font converting
engine to produce Type1 fonts for the sake of PDF and PostScript.

\subsection{The dvi Writer}

Here it depends on in the further processing chain which fon formats
are needed. Especially the feature of stretchable characters needs
further investigation. Most probably it will be necessary to produce
fonts in som format in parallel to the dvi file.

\subsection{The PDF Writer}

For the PDF Writer PostScript and Type1 fonts are digestable. Other
formats might be acceptable. Especially Open Type seems to be the
favoured font format at Adobe.

\subsection{The PostScript Writer}

Here Type1 and Type3 fonts are required. The fonts are embedded into
the Postscript code.


\section{Fonts and FontClusters}

\subsection {Virtual Fonts}

Virtual fonts are in use in \TeX\ already. They have to be extended to
deal with Unicode fonts and even compose a Unicode font from plain old
\TeX\ fonts of PostScript Type1 fonts.


\subsection {Font Clusters}

The font handling mechanism of \TeX\ is rather minimalistic. You can
load a single font only. Since this is not really user friendly
systems on the macro level have been introduced to compensate this
deficiency. One example is the NFSS of \LaTeX. \ExTeX\ should provide
a much enhanced functionality right from the start.

A font should be considered always to be part of a font cluster. In
this cluster the font is characterized by a set of dimensions, i.e.
pairs of name and values. The names are strings. The value can be one
of several types:

\begin{description}
\item[Arbitrary Strings] \ \\
  The arbitrary string values might be restricted by a finite set of
  predefined values. If a unknown value is used it should be
  substituted by a default.
  
  For instance the dimension \texttt{family} can have the value
  \texttt{cm}.

\item[Float numbers] \ \\
  These might be inherent restrictions on the actual usable values. 
  
  For instance the dimension \texttt{size} can have the value
  \texttt{14.4}.

\item[Integer numbers] \ \\
  Again the restrictions might come in otherwhere and not in the data
  structure.
\item[Boolean values] \ \\
  Just to complete the set of possibilities.
\end{description}

In principle this set can have arbitrary key and values. Nevertheless
some of the properties should be present since some primitives might
assume their existence.

Such a font specification can be manipulated to represent an arbitrary
font. This step is independent from the physical existence of a font.

Consider you start wit a font described by the following dimensions:
\begin{quotation}\noindent
  \texttt{family} $\mapsto$ cm\\
  \texttt{size}   $\mapsto$ 14.4\\
  \texttt{weight} $\mapsto$ normal\\
  \texttt{slant}  $\mapsto$ upright\\
\end{quotation}
This font might correspond to \texttt{cmr10 at 14.4pt}. Now you could
change the dimension \texttt{weight} to \texttt{ultrathin}. This would
be a valid font specification -- even if this font does not exist
anywhere. 

Whenever a font specification is completed. It can be used to request
an actual font for it from a font factory. The font factory will try
to deliver a suitable font. For this it can apply substitution rules
defined for the family. This means that the dimension family is the
primary dimension which must be used. It can be used by the font
factory to find the configuration with the substitution rules. If no
configuration can be found the family might be taken literally to find
a single font.


\subsection{100\% Compatibility}

To achieve a 100\% compatibility to \TeX\ the \ExTeX\ team came up
with a solution to map all input characters to a private Unicode area
and pass them through the whole machinery. The solution proposed in
this document is less invasive. Some of the implication have still to
be worked out.


\end{document}
