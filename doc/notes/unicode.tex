\documentclass{scrartcl}
\providecommand*{\ExTeX}{%
  \textrm{% das Logo grunds�tzlich serifenbehaftet
    \ensuremath{\textstyle\varepsilon_{\kern-0.15em\mathcal{X}}}%
    \kern-.15em\TeX
  }%
}
\begin{document}
\section{Unicode and \ExTeX}
There are different aspects concering Unicode compatibililty:
\begin{itemize}
\item Unicode semantics on glyphs
\item use a 32 bit representation for internal addressing
\item composite characters (base character combined with an arbitrary
number of combining characters)
\item normalization
\item bidirectional properties of glyphs
\item sorting (not yet relevant)
\end{itemize}

We do have two main goals:
\begin{itemize}
\item Unicode compatibility (make aspects as quoted above available to
the user and \ExTeX{} internals)
\item backward compatibility
\end{itemize}

\section{Impact on the scanner}

The scanner is responsible for creating tokens from the input streams.
Some of the tokens are letter and other tokens, which represent
integers and will be mapped to glyphs at some later point. While in
regular \TeX{} those integers will be in range 0 to 255, in \ExTeX{} an
offset will be used to move them into a private area of Unicode. An
offset of 0xee00 will be used.

There will be other techniques (extending regular \TeX) to construct
letter and other tokens different from this private Unicode area.
Those extensions need to be discussed.

\section{Impact on the interpreter}

In the interpreter at certain points the integers of the letter and
other tokens can be used for computations. Here the offset has to be
undone. When letter and other token are outside of the extended ascii
range then the Unicode semantics will be used to identify feasible
character classes, for instance minuscule numerals.

The catcode handling need to be adjusted to provide the necessary
information for the scanner. It's worth to consider to use the Unicode
semantics for all characters not contained in the private area for
catcode selection.

\section{Impact on the typesetter}

\subsection{Construction of CharNodes}

In a first step the tokenizer will construct CharNodes out of the
letter and other tokens as before.

There will be no CharNodes for combining characters. Instead the
combining characters are stored in a VirtualCharNode. This translation
step will use Unicode semantics on composite characters.

Note that the characters in the private Unicode area are non-composite
characters as \TeX{} is not aware of such constructions as well.

For regular \TeX{} fonts without further configuration the font subsystem
will keep the CharNodes in the private area and transparently perform
the necessary mapping to the font information given for the fonts to
the other components of \ExTeX. However, it is possible to construct a
font (an "\ExTeX{} Unicode font"), which creates CharNodes outside of the
private Unicode area although the original tokens are in the private
area.

Note that different "\TeX{} fonts" are associated with a single
\ExTeX{} Unicode font. This means we do not change the \ExTeX{} Unicode font
when changing the \TeX{} font but the translation step from tokens to
CharNodes takes into account the current \TeX{} font setting as stated in
the interpreter context. A \TeX{} font is infact composed of an \ExTeX{}
Unicode font and a mapping for tokens to CharNodes.

\subsection{Hyphernation}

Both the hyphernation information and the CharNodes are treated as
Unicode for hyphernation purposes. The private area is mapped for
standard ascii. In \TeX{} and \ExTeX{} the hypernation can be deactivated by
invalidating the hypenchar in the \TeX{} font. The hyphenator should take
into account that different Unicode characters might be equivalent
(like composite vs. decomposed characters).

The Unicode semantics on space characters and word boundaries are
worth a consideration for the extraction of words for hyphernation.

\section{Impact on the font subsystem}

\subsection{compatibility type font handling}

We keep everything working in the private area. The font subsystem
handles this range and maps the information available from font
metrices etc. into that range. This is the default (i.e. when no
manually created efm files or translation tables are provided).

\subsection{\ExTeX{} Unicode style fonts}

An Unicode style font allows to combine several \TeX{} fonts and map the
glyphs to the proper Unicode code points. This mapping will be applied
to letter and other tokens when building the CharNodes.

\end{document}
