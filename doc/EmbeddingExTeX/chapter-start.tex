%%*****************************************************************************
%% $Id$
%%*****************************************************************************
%% Author: Gerd Neugebauer
%%-----------------------------------------------------------------------------


\chapter{Getting Started}

To get started we have to decide which component of \ExTeX{} to
integrate. For the beginning we want to integrate the whole engine
into out application. Later we will see how to refine the requirements
and juse just a few core components.


\section{Hello World}

To get up and running we want to start with the most simple program to
do a job. The central class is \texttt{de.dante.extex.ExTeX}. This is
the entry point which ties together the various parts of the system.

Just to dive into some code we can use the following example:

\begin{lstlisting}{}
package org.extex.tutorial;

import de.dante.extex.ExTeX;
import de.dante.extex.interpreter.exception.InterpreterException;
import de.dante.util.framework.configuration.exception.ConfigurationException;

public class MyExTeX {

    public static void main(final String[] args)
            throws InterpreterException,
                ConfigurationException,
                IOException {

        ExTeX extex = new ExTeX(System.getProperties());
        extex.run();
    }
}
\end{lstlisting}

Here we declare to live in a certain package and import some of the
\ExTeX\ specific classes. The class \class{MyExTeX} just has a static
main method to allow us to run the program.

The first thing we ave to do is to create a new instance of an
\class{ExTeX} object. The constructor takes one single argument. This
argument are properties which can be used to control the behaviour of
the engine. In our example we have used the system properties since
they can be set from the command line. Later we will see several
properies to be used.

The instance itself is just there to do some job for us. This is
achieved with the method \method{run}. For the beginning this is
enough. 

You should compile this first program. For this purpose make sure that
the classpath contains \jar{extex.jar} and all other jars
distributed with \ExTeX. This depends heavily on the system you are
using and will not be described in detail here.

Now we can run the program and see what it does:

\begin{lstlisting}{}
$ java org.extex.tutorial.Main
This is ExTeX, Version 0.0 (ExTeX mode)
No pages of output.
Transcript written on ./texput.log.
\end{lstlisting}

The lines shown appear on the standard error stream. The behaviour is
similar to an invocation of \ExTeX{} without input file and nothing to
read from.

\section{Controlling \ExTeX{} with Properties}

Since we have passed the system properties to the \texttt{ExTeX}
object we can simply put the controlling properties into the system
properties. This can be done on the command line.

\begin{lstlisting}{}
$ java -Dextex.file=sample org.extex.tutorial.Main
\end{lstlisting}

This invocation tries to run \ExTeX{} on the file \file{sample} or
\file{sample.tex} -- whichever is found. The details of searching
files will be covered later.

There are more properies to use. Some of them will be described later.
Now we will first exercise how to integrate the properties into the
program itself. Thus we get less dependant of external files.

In prepration of things to come we rework our example to use a class
of our own to encapsulate the \ExTeX\ functionality. The \file{main}
method instanciates our class and invoces its \method{run} method.
Tis is shown in the following source code.

\begin{lstlisting}{}
package org.extex.tutorial;

import java.io.IOException;
import java.util.Properties;

import de.dante.extex.ExTeX;
import de.dante.extex.interpreter.exception.InterpreterException;
import de.dante.util.framework.configuration.exception.ConfigurationException;

public class MyExTeX {

    public static void main(final String[] args)
            throws InterpreterException,
                ConfigurationException,
                IOException {

        new MyExTeX().run();
    }

    private MyExTeX() {

        super();
    }

    private Properties makeProperties() {

        Properties properties = System.getProperties();
        properties.put("extex.file", "sample");
        return properties;
    }

    public void run()
            throws InterpreterException,
                ConfigurationException,
                IOException {

        ExTeX extex = new ExTeX(makeProperties());
        extex.run();
    }
}
\end{lstlisting}

In addition we have moved the setting of the properties into a method
of its own. This is the method \method{makeProperties}. Here we take
the system properties and reset some of its values. Thus it is nor
necessary -- and possible -- any more to adjust these values from the
command line.

If we want to go even further we can just omit the system properties
at all. Instead we just create a new \class{Properties} object. This
leads to the follwoing variant of the method \method{makeProperties}.

\begin{lstlisting}{firstline=25}
    private Properties makeProperties() {

        Properties properties = new Properties();
        properties.put("extex.file", "sample");
        return properties;
    }
\end{lstlisting}

\subsection{extex.nobanner}

If we want to get rid of the banner output on stderr we can use the
property \property{extex.nobanner}. This proepry can be set to true to
suppress the banner. This is shown in the following modified method
\method{makeProperties}.

\begin{lstlisting}{firstline=25}
    private Properties makeProperties() {

        Properties properties = new Properties();
        properties.put("extex.file", "sample");
        |properties.put("extex.nobanner", "true");|
        return properties;
    }
\end{lstlisting}

\subsection{extex.interaction}

\begin{lstlisting}{firstline=25}
    private Properties makeProperties() {

        Properties properties = new Properties();
        properties.put("extex.file", "sample");
        properties.put("extex.nobanner", "true");
        |properties.put("extex.interaction", "batchmode");|
        return properties;
    }
\end{lstlisting}

\subsection{extex.halt.on.error}

\begin{lstlisting}{firstline=25}
    private Properties makeProperties() {

        Properties properties = new Properties();
        properties.put("extex.file", "sample");
        properties.put("extex.nobanner", "true");
        properties.put("extex.interaction", "batchmode");
        |properties.put("extex.halt.on.error", "false");|
        return properties;
    }
\end{lstlisting}

\subsection{extex.output}

\begin{lstlisting}{firstline=25}
    private Properties makeProperties() {

        Properties properties = new Properties();
        properties.put("extex.file", "sample");
        properties.put("extex.nobanner", "true");
        properties.put("extex.interaction", "batchmode");
        properties.put("extex.halt.on.error", "false");
        |properties.put("extex.output", "pdf");|
        return properties;
    }
\end{lstlisting}

\subsection{extex.paper}

\begin{lstlisting}{firstline=25}
    private Properties makeProperties() {

        Properties properties = new Properties();
        properties.put("extex.file", "sample");
        properties.put("extex.nobanner", "true");
        properties.put("extex.interaction", "batchmode");
        properties.put("extex.halt.on.error", "false");
        properties.put("extex.output", "pdf");
        |properties.put("extex.paper", "a4");|
        return properties;
    }
\end{lstlisting}

\subsection{extex.code}

\begin{lstlisting}{}
    |private Properties makeProperties(final String code) {|

        Properties properties = new Properties();
        |properties.put("extex.code", code);|
        properties.put("extex.nobanner", "true");
        properties.put("extex.interaction", "batchmode");
        properties.put("extex.halt.on.error", "false");
        properties.put("extex.output", "pdf");
        properties.put("extex.paper", "a4");
        return properties;
    }
\end{lstlisting}

\subsection{extex.config}

\begin{lstlisting}{firstline=25}
    private Properties makeProperties(final String code) {

        Properties properties = new Properties();
        properties.put("extex.code", code);
        properties.put("extex.nobanner", "true");
        properties.put("extex.interaction", "batchmode");
        properties.put("extex.halt.on.error", "false");
        properties.put("extex.output", "pdf");
        properties.put("extex.paper", "a4");
        |properties.put("extex.config", "extex");|
        return properties;
    }
\end{lstlisting}


\section{Configuring \ExTeX{}}


%
% Local Variables: 
% mode: latex
% TeX-master: "embedExTeX.tex"
% End: 
