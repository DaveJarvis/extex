%%*****************************************************************************
%% $Id: section-ide-shell.tex 5558 2007-05-02 08:59:42Z gene $
%%*****************************************************************************
%%
%% Copyright (C) 2005-2008 The ExTeX Group and individual authors listed below
%%
%% This library is free software; you can redistribute it and/or modify it
%% under the terms of the GNU Lesser General Public License as published by the
%% Free Software Foundation; either version 2.1 of the License, or (at your
%% option) any later version.
%%
%% This library is distributed in the hope that it will be useful, but WITHOUT
%% ANY WARRANTY; without even the implied warranty of MERCHANTABILITY or
%% FITNESS FOR A PARTICULAR PURPOSE. See the GNU Lesser General Public License
%% for more details.
%%
%% You should have received a copy of the GNU Lesser General Public License
%% along with this library; if not, write to the Free Software Foundation,
%% Inc., 59 Temple Place, Suite 330, Boston, MA 02111-1307 USA
%%
%%*****************************************************************************
%% @author Gerd Neugebauer
%%-----------------------------------------------------------------------------

\section{Command Line Use}

\subsection{Downloading the Sources}

The sources of \ExTeX\ are stored in a Subversion repository. To
access this repository you need access to the internet and Subversion
installed and configured in some way\footnote{For instance if you are
  behind a firewall you might have to configure the proxy settings in
  \texttt{servers.xml}}.

The coordinates of the repository are:

\begin{quotation}
  http://svn.berlios.de/svnroot/repos/extex/trunk
\end{quotation}

These coordinates provide anonymous access to the sources with reading
permissions only. You are not allowed to commit changes when you are
using this URL.

Developers with commit permissions need to have an account at Berlios
and be registered as committers. They can use one of the following
URLs instead:

\begin{quotation}
  https://svn.berlios.de/svnroot/repos/extex/trunk
  svn+ssh://svn.berlios.de/svnroot/repos/extex/trunk
\end{quotation}

You need to download the sources of \ExTeX. On the command line this
can be done with the following commands:

\begin{lstlisting}{}
# svn co http://svn.berlios.de/svnroot/repos/extex/trunk
\end{lstlisting}{}

After some time you should have a working copy of the files on your
computer.

\subsection{Maven}\label{sec:Maven}

Maven (\url{http://maven.apache.org}) is a build tool which comes with a
set of conventions and plugins which are very useful. The primary
features of Maven are the dependency management and the automatic
downloading of missing dependencies.


%\subsection{Checkstyle}
%
%Checkstyle is a source code checker.
%\ExTeX\ should show a homogeneous appearance of the sources. Thus
%certain rules should be followed. Some of the rules are checked by the
%following command:
%
%\begin{lstlisting}{}
%# ant checkstyle
%\end{lstlisting}{}
%
%The result can be found in the file \File{target/checkstyle.txt}.

\subsection{Ant}\label{sec:Ant}

Apache Ant (\url{http://ant.apache.org}) is a build system for Java.
It can be considered state of the art for Java programs to come with
Ant scripts. \ExTeX\ supports Ant by providing a \texttt{build.xml}
file for various tasks.

The files needed for running Ant are included in the \ExTeX\
repository. Thus no additional installation is required. Just some
setting need to be performed before Ant can be used.

An environment variable \verb|JAVA_HOME| should be defined which points
to the JDK. The following jars should be placed on the environment
variable \verb|CLASSPATH|:
\begin{itemize}
\item \verb|$JAVA_HOME/lib/tools.jar|
\item \verb|$JAVA_HOME/lib/classes.zip|
\item and all jars found in \File{develop/lib}
\end{itemize}

The Ant can be invoked like in

\begin{lstlisting}{}
#  $JAVA_HOME/bin/java -Dant.home=$ANT_HOME org.apache.tools.ant.Main compile
Buildfile: build.xml

compile:

BUILD SUCCESSFUL
Total time: 1 second
#
\end{lstlisting}

These steps are performed by the shell script \File{build} in the
\ExTeX\ directory. Thus you can achieve the same effect -- without any
preparations except setting \verb|JAVA_HOME| -- with the following command:

\begin{lstlisting}{}
# cd build
# ant compile
Buildfile: build.xml

compile:

BUILD SUCCESSFUL
Total time: 1 second
#
\end{lstlisting}

The Ant configuration can be found in the file \File{build.xml} in the
\ExTeX\ main directory. This configuration contains at least the
following targets:

\begin{description}
\item [all] This target builds nearly everything.
\item [compile] Tis target compiles all Java files of the sources into
  the directora \File{target/classes}. Note, that the test classess
  are not compiled. See also section~\ref{sec:shell-compile}.
\item [jar] This target arranges that the file \File{target/extex.jar}
  is created. It contains the compiled sources of  \ExTeX.
\item [javadoc] This target creates the Javadoc HTML files in the
  directory \File{target/javadoc}. See also section~\ref{sec:shell-javadoc}.
\item [checkstyle] This target applies checkstyle and creates a report
  in \File{target/checkstyle.txt}.
\item [tests] This target aplies all JUnit test cases. See also
  section~\ref{sec:shell-junit}.
\item [installer] This target creates the installer with the graphical
  user interface. The result is placed in the file
  \File{target/ExTeX-setup.jar}. See also section~\ref{sec:shell-installer}.
\item [clean] This target deletes some generated files.
\end{description}


\subsection{Compiling \ExTeX}\label{sec:shell-compile}

Compiling \ExTeX form the command line can be accomplished with the
help of the build script. The build script is a wrapper around Ant. It
can be invoked with the following command:

\begin{lstlisting}{}
# cd build
# ant compile
Buildfile: build.xml

compile:

BUILD SUCCESSFUL
Total time: 1 second
#
\end{lstlisting}

The generated files are placed in the sub-directory
\texttt{target/classes}. Thus if this directory and the jars in
\File{lib} are on the class path then \ExTeX\ can be run immediately.


\subsection{Running \ExTeX}

\ExTeX\ can be run with the help of the \ExTeX\ script in the main
directory or by a direct invocation of Java. The start script is
provided for Unix under the name \File{extex} and for Windows under
the name \File{extex.bat}.
\begin{lstlisting}{}
# extex work/empty.tex
This is ExTeX, Version 0.0 (ExTeX mode)
(work/empty)
No pages of output.
Transcript written on ./empty.log.
#
\end{lstlisting}{}

For the usual purposes these scripts can be used as a plug-in
replacement for \TeX. See the User's Guide for the command line
options.

To run \ExTeX\ from the command line prepare the class path -- i.e.
the environment variable \texttt{CLASSPATH} -- to contain all
libraries found in the directory \texttt{lib}. In addition the
directory \texttt{target/classes} have to be on the class path.
Then you can invoke \ExTeX\ like in the following example:

\begin{lstlisting}{}
#  java de.dante.extex.Main.TeX work/empty
This is ExTeX, Version 0.0 (ExTeX mode)
(work/empty)
No pages of output.
Transcript written on ./empty.log.
#
\end{lstlisting}{}

The command line arguments are the same as for \File{extex} mentioned
above.


\subsection{JUnit}\label{sec:shell-junit}

JUnit is the state of the art concerning test automation for Java
programs. Thus \ExTeX\ provides some test cases in the form of JUnit
classes.

All test can be run from the command line with the build script:

\begin{lstlisting}{}
# cd build
# ant tests
Buildfile: build.xml

compile:

jar:
      [jar] Building jar: /home/gene/src/ExTeX/target/lib/extex.jar

compile.tests:
     [copy] Copying 148 files to /home/gene/src/ExTeX/target/classes

jar.tests:
      [jar] Building jar: /home/gene/src/ExTeX/target/lib/testsuite.jar

tests:
    [mkdir] Created dir: /home/gene/src/ExTeX/tmp/tests
    [junit] Running de.dante.etex.CurrentgrouplevelTest
    [junit] Tests run: 1, Failures: 0, Errors: 0, Time elapsed: 1,623 sec
...

BUILD SUCCESSFUL
Total time: 3 minutes 54 seconds
#
\end{lstlisting}

This invocation runas all JUnit test cases found in the directory
\File{src/test}. The results can be found in the directory
\File{target/tests} with one file per test class.

To run single cases or a selected set of test cases you can use the
parameter \texttt{cases}. This parameter is added to the command line
arguments with the prefix \verb|-D|. The value follows after an equals
sign. The value is a pattern to select the test case files to be used.
The pattern \verb|**| denotes an arbitrary deep directory hierarchy.
\verb|*| denotes an arbitrary sequence of characters. Note that the
pattern should end in \verb|Test.java|.

\begin{lstlisting}{}
# cd build
# ant tests -Dcases=**/RelaxTest.java
Buildfile: build.xml

compile:

jar:

compile.tests:

jar.tests:

tests:
   [delete] Deleting directory /home/gene/src/ExTeX/target/tests
    [mkdir] Created dir: /home/gene/src/ExTeX/target/tests
    [junit] Running de.dante.extex.interpreter.primitives.RelaxTest
    [junit] Tests run: 5, Failures: 0, Errors: 0, Time elapsed: 1,062 sec

BUILD SUCCESSFUL
Total time: 5 seconds
#
\end{lstlisting}

Details on testing and test cases can be found in
section~\ref{chapter:testing}.


\subsection{Creating Javadoc}\label{sec:shell-javadoc}

Creating the Javadoc HTML pages can best be complished with the help
of the build script. Here the target \texttt{javadoc} does everythign
necessary:

\begin{lstlisting}{}
# cd build
# ant javadoc
Buildfile: build.xml

javadoc:
  [javadoc] Generating Javadoc
  [javadoc] Javadoc execution
  [javadoc] Loading source files for package de.dante.extex...
  [javadoc] Loading source files for package de.dante.extex.color...
  [javadoc] Loading source files for package de.dante.extex.color.model...
...

BUILD SUCCESSFUL
Total time: 2 minutes 20 seconds
#
\end{lstlisting}{}

As the result of this invocation the Javadoc HTML pages are stored in
the sub-directory \File{target/javadoc}.


\subsection{Creating the Installer}\label{sec:shell-installer}

The installer can be build with the help of the build script. The
invocation looks as follows:

\begin{lstlisting}{}
# cd build
# ant installer
Buildfile: build.xml

compile:

jar:
      [jar] Building jar: /home/gene/src/ExTeX/target/lib/extex.jar

installer:
   [izpack] Adding resource : IzPack.uninstaller ...
   [izpack] Setting the installer informations ...
   [izpack] Setting the GUI preferences ...
... 
   [izpack] Writing Packs ...
   [izpack] Writing Pack #0 : Core
   [izpack] Writing Pack #1 : Libraries
   [izpack] Writing Pack #2 : User Settings
   [izpack] Writing Pack #3 : Documentation
   [izpack] Writing Pack #4 : Fonts
   [izpack] Writing Pack #5 : Sources

BUILD SUCCESSFUL
Total time: 1 minute 30 seconds
#
\end{lstlisting}{}

After the work is complete the installer can be found in the file
\File{ExTeX-setup.jar} in the directory \File{target}. The use of the
installer is described in the User's Manual.

Alternatively the installer can also be created with the Ant task
\texttt{installer}. Using this method can be applied from the command
line and from within Eclipse.

Note that the installer is automatically created once a day and
provided in the snapshot directory of the \ExTeX\ Web site.

