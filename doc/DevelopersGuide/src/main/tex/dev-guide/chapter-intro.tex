%%*****************************************************************************
\SVN$Id$
%%*****************************************************************************
%%
%% Copyright (C) 2005-2009 The ExTeX Group and individual authors listed below
%%
%% This library is free software; you can redistribute it and/or modify it
%% under the terms of the GNU Lesser General Public License as published by the
%% Free Software Foundation; either version 2.1 of the License, or (at your
%% option) any later version.
%%
%% This library is distributed in the hope that it will be useful, but WITHOUT
%% ANY WARRANTY; without even the implied warranty of MERCHANTABILITY or
%% FITNESS FOR A PARTICULAR PURPOSE. See the GNU Lesser General Public License
%% for more details.
%%
%% You should have received a copy of the GNU Lesser General Public License
%% along with this library; if not, write to the Free Software Foundation,
%% Inc., 59 Temple Place, Suite 330, Boston, MA 02111-1307 USA
%%
%%*****************************************************************************
%% @author Gerd Neugebauer
%%-----------------------------------------------------------------------------
\chapter{Introduction}

\ExTeX{} aims at providing a high-quality typesetting system. The
development of \ExTeX\ has been inspired by the experiences with
\TeX\index{TeX@\TeX} \cite{knuth:texbook}. The focus lies on an open
design and a high degree of configurability. Thus \ExTeX\ should be a
good base for further development.

On the other hand we have to take care not to leave the current user
base of \TeX\ behind. pdf\TeX\ has taught us that a migration path
from \TeX\index{TeX@\TeX} has a positive value in it. In the mean time
the majority of \TeX\ users applies in fact
pdf\TeX\index{pdfTeX@pdf\TeX}.

To provide a backward compatibility of \ExTeX\ with
\TeX\index{TeX@\TeX} one special configuration is provided. Thus
backward compatibility is just a matter of configuration.


\section{Audience}

This document is meant for developers and those interested in the
sources and development processes of \ExTeX. It should contain all
information for getting started quickly.


\section{Mailing Lists}\index{Mailing list|(}

If you are ready to try \ExTeX{} you might as well want to join a
mailing list to get in contact with the community. The following
mailing lists might be of interest:

\begin{description}
\item[extex@dante.de] \ \index{Mailing list!general}\\
  A general mailing list about \ExTeX. It has low traffic and is
  mainly in German. Subscribe and unsubscribe via the Web form
  \url{http://www.dante.de/listman/extex}.

\item[extex-eng@dante.de] \ \index{Mailing list!English}\\
  A general mailing list about \ExTeX. It has currently very low
  traffic and is in English. Subscribe and unsubscribe via the Web
  form \url{http://www.dante.de/listman/extex-eng}.

\item[extex-devel@dante.de] \ \index{Mailing list!developers}\\
  A mailing list for the exchange of the developers of \ExTeX. It has
  low traffic and is partly in German. Subscribe and unsubscribe via
  the Web form \url{http://www.dante.de/listman/extex-devel}.

\item[extex-svn.berlios.de] \ \index{Mailing list!SVN}\\
  A mailing list for automatic notification about changes in the Subversion
  repository of \ExTeX. It is not meant to post mails on this list.
  This list is not archived. Subscribe and unsubscribe via the Web
  form \url{https://lists.berlios.de/mailman/listman/extex-cvs}. You
  need to be logged in at Berlios\index{Berlios} when registering.

\item[extex-bugs@list.berlios.de] \ \index{Mailing list!bugs}\\
  A mailing list for bug messages of \ExTeX. Subscribe and unsubscribe
  via the Web form
  \url{http://lists.berlios.de/mailman/listman/extex-bugs}.
\end{description}
\index{mailing list|)}

\section{Organizational Agreements}

The developers of \ExTeX\ have agreed on some rule for cooperation.
Those rules are documented here.

\subsection{Language}\index{language}

The official project language for \ExTeX\ is English in the US
dialect. The sources are documented in this language and the major
documents are written in this language.

Since some of the developers are German this language might slip in
during intensive discussions.


\subsection{Maintainers of Files}\index{maintainer}

Each file has a single maintainer -- even if there are several
authors. The maintainer has to be informed and has to agree on any
changes in the file. The maintainership is usually indicated in the
Java sources with the help of the tag \texttt{@author}\index{author}.
The first author is always the maintainer.

Changes to a file can be carried out by the maintainer or delegated to
somebody else. The maintainer can change if both the old and new
maintainer agree in this.


\section{Design Rationals}

\subsection{Java 6}

\ExTeX\ is assumed to run on \+Java+ 6 and upwards. Thus the features
of this release can be used. No provisions have to be made to
support any version of Java prior to Version 6.


\subsection{Library Character}

\ExTeX\ should provide a \+library+ for typesetting. It should be
embeddable in arbitrary programs. Thus minimal assumptions have to be
made about the execution environment.

For instance this means that it is possible that \ExTeX\ is run is an
application server. Application servers usually deal with their own
thread management. Thus \ExTeX\ should avoid to play around with
threads.


\subsection{Code Formatting}

The Coding Standards \cite{coding-conventions} of Sun have been become
the de facto standard in the context of Java programming. Thus these
rules are used as base for this project as well.

The coding standards -- with some deviations from the Sun Coding
Conventions -- have been manifested in the form of \+Checkstyle+ rules
and an Ecli�pse Java Formatter configuration (see
section~\ref{sec:eclipse.checkstyle}).


\subsection{No Static Class Variables}

Static class variables act like global variables. In the context of
object-oriented programming it is considered as bad design to use
them. Thus each static variable has to be justified extensively.


\subsection{No System.exits() Outside of main()}

The method |System.exit()| terminates the execution of the programn
and returns an exit status back to the operating system. Invoking this
method somewhere in code makes it impossible to easily test the code
from within Java (e.g. with \+JUnit+)

The method \verb|System.exit()| is allowed in the method \verb|main()|
only.


