%%*****************************************************************************
\SVN$Id: chapter-directory.tex 7897 2009-08-09 20:18:27Z gene $
%%*****************************************************************************
%%
%% Copyright (C) 2005-2009 The ExTeX Group and individual authors listed below
%%
%% This library is free software; you can redistribute it and/or modify it
%% under the terms of the GNU Lesser General Public License as published by the
%% Free Software Foundation; either version 2.1 of the License, or (at your
%% option) any later version.
%%
%% This library is distributed in the hope that it will be useful, but WITHOUT
%% ANY WARRANTY; without even the implied warranty of MERCHANTABILITY or
%% FITNESS FOR A PARTICULAR PURPOSE. See the GNU Lesser General Public License
%% for more details.
%%
%% You should have received a copy of the GNU Lesser General Public License
%% along with this library; if not, write to the Free Software Foundation,
%% Inc., 59 Temple Place, Suite 330, Boston, MA 02111-1307 USA
%%
%%*****************************************************************************
%% @author Gerd Neugebauer
%%-----------------------------------------------------------------------------
\chapter{The Source Tree Organization}

In this section the description of the directory hierarchy is
contained. This structure is oriented on the structuring proposed by
Maven (\url{http://maven.apache.org}).


\section{General File Organization}


\begin{description}
\item[\File{README.html}] The file \File{README.html} contains the
  starting and overview inforamtion about the current directory.
\item[\File{LICENSE.txt}] The file \File{LICENSE.txt} contains the
  lincense conditions for the contents of the current directory.
\item[\File{pom.xml}] The file \File{pom.xml} is the configuration
  file for Maven. It includes the specification for the artifacts to b
  created in this directory.
\item[\Dir{src}] The directory \Dir{src} contains the source for the
  current module. This follows the conventions of Maven. The
  subdirectories are usually \Dir{main} and \Dir{test}. 
\item[\Dir{target}] The directory \Dir{target} is used by Maven to
  stor intermediaray files and the result of the build process.
\end{description}


\section{The Toplevel Directory}

The toplevel directory of an \ExTeX\ project contains the sub-projects
of \ExTeX. The sources are organized in components with predefined
dependencies. Those components are Eclipse projects as well as Maven
modules.

The components are cotained in directories staring with
\texttt{ExTeX-}. In addition some directories contain additional
material, the build system, or the web pages.

The following is an excpert of the modules present in the top-level
directory.

{
\begin{multicols}3 \footnotesize\sf\obeylines
CLI
ExBib
ExIndex
ExTeX
ExTeX-resource
Sandbox
build
develop
doc
library
maven2
site
texmf
tools
util
www
%  ExTeX-Backend-dump
%  ExTeX-Backend-dvi
%  ExTeX-Backend-dvix
%  ExTeX-Backend-pdfbox
%  ExTeX-Backend-ps
%  ExTeX-Backend-rtf
%  ExTeX-Backend-svg
%  ExTeX-Backend-text
%  ExTeX-Backend-xml
%  ExTeX-Font
%  ExTeX-Font-API
%  ExTeX-Font-afm
%  ExTeX-Font-otf
%  ExTeX-Font-tfm
%  ExTeX-Interpreter-API
%  ExTeX-Interpreter-max
%  ExTeX-Main-fmt
%  ExTeX-Main-tex
%  ExTeX-Scanner
%  ExTeX-Scanner-API
%  ExTeX-Scanner32
%  ExTeX-Typesetter-API
%  ExTeX-Typesetter-tex
%  ExTeX-Unit-color
%  ExTeX-Unit-etex
%  ExTeX-Unit-extex
%  ExTeX-Unit-image
%  ExTeX-Unit-namespace
%  ExTeX-Unit-native
%  ExTeX-Unit-omega
%  ExTeX-Unit-pdftex
%  ExTeX-Unit-tex
%  ExTeX-base
%  ExTeX-base-ext
%  ExTeX-core
%  ExTeX-fontware
\end{multicols}
}


Some of the modules will be described in the following sections.


\section{\texttt{ExTeX-*}: Components}

Those sub-directories starting with \texttt{ExTeX-} are considered to
contain source code for modules. Internally it is organized according
to the Maven conventions. Thus the following sub-directories may be
present:

\begin{description}
\item[\Dir*{src/main}\Dir{java}] \ \\
  This directory contains the Java source code and additional files
  meant for documenting it with Javadoc. Note that other files will
  not be copied to the classpath or included into the jar.
\item[\Dir*{src/main}\Dir{resources}] \ \\
  This directory contains files which are copied to the output
  directory and included in the jar. For instance properties and XML
  configurations files for the productive version are contained here.
\item[\Dir*{src/test}\Dir{java}] \ \\
  This directory contains Java source code for the tests. These are
  \+JUnit+ test cases and test suites as well as supporting files.
  These files are not contained in the jar and not be present in the
  productive environment. Note that non-Java files will not be copied
  to the outut directory and thus they will not be present during the
  tests.
\item[\Dir*{src/test}\Dir{resources}] \ \\
  This directory contains files which are copied to the output
  directory for test files.
\item[\Dir*{src}\Dir{site}] \ \\
  This directory contains files included into the site produced by
  \+Maven+. This site describes the result of building and testing the
  current module.
\item[\Dir{target}] \ \\
  This directory contains files generated during the build process.
  Thus this directory is excluded from the Subversion repository.
\item[\Dir*{target}\Dir{classes}] \ \\
  This directory continas the classes and files for the productive
  environment.
\item[\Dir*{target}\Dir{test-classes}] \ \\
  This directory contains the additional classes and files for the
  tests.
\end{description}

Note that Eclipse does not provide a fine-grained differentiation as
proposed above. Thus many files will be placed in the same output
directory. As a consequence some constellations might run under
Eclipse and produce a result which is different from the one in a
productive environment.

Until Maven is fully integrated the following directories may contain
third-party libraries needed for the differet phases:



\begin{description}
\item[\Dir{lib}] \ \\
  These libraries are needed for production.
\item[\Dir*{src/test}\Dir{lib}] \ \\
  These libraries are needed for tests only.
\end{description}


\section{\texttt{build}: The Build System}

The directory \texttt{build} contains the build system for \ExTeX.
This means all nuts and bolds needed for the build exclusively are
located here.


\section{\texttt{doc}: Documentation}

The directory \texttt{doc} contains documentation written for \ExTeX.

\begin{description}
\item[DevelopersGuide] 
\item[Library] 
\item[Publications] 
\item[UsersGuide] 
\item[models] 
\item[notes] 
\end{description}


\section{\texttt{library}: Collected Works}

The sub-directory \texttt{library} contains documents not written
within the \ExTeX\ project. Interesting papers on \TeX\ and friends
might be found here.



\section{\texttt{util}: Utilities}

This directory contains vairous utitlies and scripts.

\begin{description}
\item[Installer] 
\end{description}

\section{\texttt{develop}: Development Support}

This directory contains some files needed for development. This
document references some of those.

\begin{description}
\item[eclipse] 
\item[lib] 
\end{description}

\section{\texttt{Sandbox-*}: User's Working Area}

Any developer may have some files in the project area which are not
yet ready for production. Nevertheless it might be desirable to have
them incuded in the Subversion repository.  For this purpose the
directory \texttt{Sandbox-*} is reserved where * is one of the user
ids of the committer.

\section{\texttt{www}: The Web Site}

This directory contains the source files for the web site for \ExTeX.

\begin{description}
\item[src] 
\end{description}

