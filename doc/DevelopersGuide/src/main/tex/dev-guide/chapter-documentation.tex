%%*****************************************************************************
\SVN$Id$
%%*****************************************************************************
%%
%% Copyright (C) 2005-2011 The ExTeX Group and individual authors listed below
%%
%% This library is free software; you can redistribute it and/or modify it
%% under the terms of the GNU Lesser General Public License as published by the
%% Free Software Foundation; either version 2.1 of the License, or (at your
%% option) any later version.
%%
%% This library is distributed in the hope that it will be useful, but WITHOUT
%% ANY WARRANTY; without even the implied warranty of MERCHANTABILITY or
%% FITNESS FOR A PARTICULAR PURPOSE. See the GNU Lesser General Public License
%% for more details.
%%
%% You should have received a copy of the GNU Lesser General Public License
%% along with this library; if not, write to the Free Software Foundation,
%% Inc., 59 Temple Place, Suite 330, Boston, MA 02111-1307 USA
%%
%%*****************************************************************************
%% @author Gerd Neugebauer
%%-----------------------------------------------------------------------------
\chapter{Source Code Documentation}


The source code has to be documented. \TeX\ shows us a good example of
a proper documenatation. Donald Knuth\index{Knuth, Donald} has invented
the \+Web system+ to keep together the documentation and the source code.
This is called \+literate programming+. The source code and
documentation are extracted from a common file. In the Java world the
\+Javadoc+ system has been invented for a similar purpose. This is
used -- with some extensions -- in \ExTeX\ development.


\section{Javadoc}\index{Javadoc|(}

The Javadoc conventions for comments make it possible to extract the
relevant part of the documentation and generate several output formats
from it. The primary output format is \+HTML+ (ref. figure~\ref{fig:javadoc}).
\begin{figure}[tbh]
  \centering
  \includegraphics[scale=.33]{image/javadoc}
  \caption{Javadoc in the Browser}\label{fig:javadoc}
\end{figure}

Javadoc includes optionally the description of packages into the
generated output -- if the appropriate input files are present. Here
the \ExTeX\ team follows the convention to use the files names
\File{package-info.java} since referenes contained herein are expanded
to links\footnote{In contrast to the old fashioned files
  \File{package.html}}.

In each main package a file \File{package-info.java} should be
present. This does not include test packages or resource packages.
\index{Javadoc|)}


\section{Documentation Files}

The documentation of a component does not only consist of Javadoc.
Each component resides in its own directory. For instance the
component \texttt{CLI} lives in the directory \Dir{CLI}.

The directory containing a component usually has a file \File{pom.xml}
for compiling and packaging its contents. Besides a few files should
also be present in each component directory:

\begin{description}
\item[\File{README.html}] contains a short description of the
  component and instruction on compiling it. Images included in this
  file are usually located in \Dir{src/images}.
\item[LICENSE.html] contains the license on which the component is
  distributed. Usually it should be the text of the GPL 2 or the FDL
  for documentation modules.
\end{description}

\subsection{Side Bar for \texttt{README.html}}

The files \File{README.html} have a common appearance. Part of this
appearance is a gray side bar with a fading and the component name
sideways from botton to top.

The image for this side bar can be generated with the help of the file
\File{doc/src/main/tex/splash.tex}. This \LaTeX\ file contains the
definition to produce a PDF with the appropriate content. The next
steps are as follows:

\begin{itemize}
\item Run pdflatex on \File{splash.tex} to obtain the PDF file.
\item Extract the logo for the left side by opening the PDF file with
  Gimp, clip the image to the needed portion, and scale the image to a
  width of 115 pixels. Finally store the image as PNG in the
  directory \Dir{src/images}. 
\item Clip the image to contain just the last line of pixels and store
  the result as another file in the directory \Dir{src/images}.
\end{itemize}

The first file obtained is used as logo for the left side. The second
file contains the repeated image for the gray bar.

\section{Documentation of Primitives}

The documentation of the primitives is contained in the \+Javadoc+
comments of the implementing Java classes. A script is used to extract
the information from the sources for the \textit{\ExTeX\ User's
  Manual}. To make this happen, the documentation meant for the manual
has to be marked and formatted specially.

\INCOMPLETE


\section{Generating The Documentation}\index{documentation|(}

The documentation is contained in the directory \File{doc} in the
\ExTeX\ top level directory.

\subsection{Preparation}

The build system for the documentation is based on Maven 3.

Some prerequisites are mandatory for the build to work. First we need
the \verb|doc-tools|\index{doc-tools} in order to work. The
|doc-tools| can be found in the directory \verb|tools/doc-tools|, Thus
the following commands ensure that the doc tools are installed in the
local repository:

\begin{lstlisting}{}
  # cd tools/doc_tools
  # mvn install
\end{lstlisting}

Since \ExBib\index{\ExBib@ExBib} tries to use its own programs as far
as possible we start with \ExBib. This can be installed with the
following commands:

\begin{lstlisting}{}
  # cd ExBib
  # mvn install
\end{lstlisting}

Now we are ready and all prerequisited are satisfied.


\subsection{The \ExTeX\ User's Manual}\index{user's manual}

After the preparating steps the documentation can be build with the
Maven task \texttt{compile} as shown below:

\begin{lstlisting}{}
  # cd doc/UsersGuide
  # mvn compile
\end{lstlisting}

\subsection{The Developer's Guide}\index{developer's manual}

After the preparating steps the documentation can be build with the
Maven task \texttt{compile} as shown below:

\begin{lstlisting}{}
  # cd doc/DevelopersGuide
  # mvn compile
\end{lstlisting}

\index{documentation|)}
