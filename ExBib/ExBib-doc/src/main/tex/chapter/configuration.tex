%%*****************************************************************************
%% $Id: configuration.tex,v 0.00 2008/05/03 14:23:24 gene Exp $
%%*****************************************************************************
%% Author: Gerd Neugebauer
%%-----------------------------------------------------------------------------


\section{Configuring \ExBib}

\subsection{Locating the Configuration}

If you are looking for the configurations in the distribution or want
to provide a configuration of your own you need to know how a
configuraion is found. \ExBib\ has a general mechanism for finding
resources. Unfortunately this mechanism can not be used for the
configuration since it is defined in the configuration itself. Thus
the location of a configuration uses a simpler search.

There are two places where \ExBib\ looks for a configuration. Those
are mainly located in the \ExBib\ installation directory (cf.
section~\ref{sec:inst.dir}).
\begin{itemize}
\item The current directory may contain a configuration with the
  proper name and path.
\item The directory \File{config} in the \ExBib\ installation
  directory contains an unpacked directory tree. Any configuration in
  this directory tree with a proper name and path is considered.
\item The directory \File{lib} in the \ExBib\ installation directory
  contains jar archives. Any configuration in a jar archive with a
  proper name and path is considered.
\end{itemize}

The top level configuration file is located with the path
\texttt{exbib}. The prefix \texttt{config/} is tried first; then the
name withpout prefix. The configuration is used with the additional
extension \texttt{.xml} and as given. Thus the configuration
\texttt{cfg} passed in as property or via the command line is sought
with the following full names:
\begin{verbatim}
  config/exbib/cfg.xml
  config/exbib/cfg
  exbib/cfg.xml
  exbib/cfg
\end{verbatim}


\subsection{Extracting a Configuration from an Archive}

To get started you may want to examine the configurations in a jar
archive. This can be accomplished with the program \Prog{jar} whoich
is part of the installed Java. We assume that the program can be found
on your search path for executables.

First you can list the files as follows:
\begin{lstlisting}[language=sh]
  # jar -tvf exbib.jar
\end{lstlisting}

Then you can extract the files you want from the archive
\begin{lstlisting}[language=sh]
  # jar -xvf exbib.jar config/exbib/exbib.xml
\end{lstlisting}

Note that a directory tree is created containing the extracted files.


\subsection{The Configuration Format}

The configuration files are XML files.

\begin{lstlisting}[language=XML]
  # jar -xvf exbib.jar config/exbib/exbib.xml
\end{lstlisting}


\INCOMPLETE


\endinput
%
% Local Variables: 
% mode: latex
% TeX-master: "../exbib-manual"
% End: 
