%%*****************************************************************************
%% $Id: finder.tex,v 0.00 2008/05/03 14:23:54 gene Exp $
%%*****************************************************************************
%% Author: Gerd Neugebauer
%%-----------------------------------------------------------------------------

\section{Finding Files and Resources}

All inputs needed are called resources. In the simplest case those
resources are plain files. Whenever a resource is needed a search for
it takes place. We will see how the search works and how it can be
influenced.


\subsection{The Configuration File}

\ExBib\ inherits its resource search from \ExTeX\index{ExTeX@\ExTeX}.
This resource search is rather flexible and highly configurable. With
this respect it follows the path defined by kpathsea\index{kpathsea}.

The search for a resource can be configured in a configuration file.
This configuration files utilizes resource finders which have a
configuration section each.

\INCOMPLETE

\subsection{Locating the Configuration File in \ExBib}


\INCOMPLETE

\subsection{File Search in the Directory Tree}

The simplest search strategy is to look into a list of directories and
maybe try a list of extensions until a suitable file is found. This
strategy is implemented in the class
\texttt{org.extex.resource.FileFinder}.

Attributes
\begin{description}
\item[class] The attribute \texttt{class} has the value
  \texttt{org.extex.resource.FileFinder} for the search in
  archives. It is mandatory.
\item[default] The attribute \texttt{default} names the type shich
  should be used if no special section for the type at hand is
  present. This attribute is mandatory.
\item[trace] The attribute \texttt{trace} can be used to turn on the
  tracing from within the configuration. It takes a boolean value as
  argument (\texttt{true} or \texttt{false}). The tracing can be
  overwritten at runtime. This attribute is optional. The default
  value is \texttt{false}.
\end{description}

\begin{lstlisting}[language=XML]
  <Finder class="org.extex.resource.FileFinder"
          default="default">
    <default>
      <path>.</path>
      <path property="extex.texinputs"/>
      <path property="texinputs"/>
      <extension>.{type}</extension>
      <extension/>
    </default>
  </Finder>
\end{lstlisting}


\INCOMPLETE

\subsection{Searching in texmf Trees}

Attributes
\begin{description}
\item[class] The attribute \texttt{class} has the value
  \texttt{org.extex.resource.ClasspathArchiveFinder} for the search in
  archives. It is mandatory.
\item[default] The attribute \texttt{default} names the type shich
  should be used if no special section for the type at hand is
  present. This attribute is mandatory.
\item[capacity] The attribute \texttt{capacity} gives an initial
  capacity for the size of the hash used to store the index entries
  from the \File{ls-R} files. Its fallback is rather small and the
  performance can be influenced positively by using this attribute
  with a proper value.
\item[trace] The attribute \texttt{trace} can be used to turn on the
  tracing from within the configuration. It takes a boolean value as
  argument (\texttt{true} or \texttt{false}). The tracing can be
  overwritten at runtime. This attribute is optional. The default
  value is \texttt{false}.
\end{description}

\begin{lstlisting}[language=XML]
  <Finder class="org.extex.resource.LsrFinder"
          trace="false"
          capacity="88888"
          default="default">
    <path property="extex.texinputs"/>
    <path property="texinputs"/>

    <default>
      <extension>.{type}</extension>
      <extension/>
    </default>
  </Finder>
\end{lstlisting}

\INCOMPLETE

\subsection{Recursively Searching in Directory Trees}

\begin{lstlisting}[language=XML]
\end{lstlisting}

\INCOMPLETE

\subsection{Searching in Archives}

The preferred way to store resources in \ExTeX\ are archives. Archives
are conpressed files which contain a complete tree of directories and
files. This archive also contains administrative information.

The archive used in \ExTeX\ have the extension \texttt{.jar}. They are
special jar files which in turn are special zip files. Thus you can
use the programs \Prog{jar}, \Prog{unzip}, or \Prog{Winzip} to get a
listing or extract the directory tree from the archive.

Attributes
\begin{description}
\item[class] The attribute \texttt{class} has the value
  \texttt{org.extex.resource.ClasspathArchiveFinder} for the search in
  archives. It is mandatory.
\item[default] The attribute \texttt{default} names the type shich
  should be used if no special section for the type at hand is
  present. This attribute is mandatory.
\item[trace] The attribute \texttt{trace} can be used to turn on the
  tracing from within the configuration. It takes a boolean value as
  argument (\texttt{true} or \texttt{false}). The tracing can be
  overwritten at runtime. This attribute is optional. The default
  value is \texttt{false}.
\end{description}

\begin{lstlisting}[language=XML]
  <Finder class="org.extex.resource.ClasspathArchiveFinder"
          trace="false"
          default="default">
    <default>
      <extension>.{type}</extension>
    </default>
  </Finder>
\end{lstlisting}

\INCOMPLETE

\subsection{The Default Search}

\INCOMPLETE

\begin{lstlisting}[language=XML]
\end{lstlisting}

\subsection{Tracing the Search}

The tracing can be activated with the trace attributes for some
sections or globally for the whole search at runtime. As a consequence
the log file should contain messages describing the search -- where it
looked and what it found.

\INCOMPLETE

\begin{lstlisting}[language=XML]
\end{lstlisting}

\endinput
%
% Local Variables: 
% mode: latex
% TeX-master: "../exbib-manual"
% End: 
