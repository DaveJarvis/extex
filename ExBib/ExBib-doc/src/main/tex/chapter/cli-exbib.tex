%%*****************************************************************************
%% $Id: cli-exbib.tex,v 0.00 2008/05/02 17:03:28 gene Exp $
%%*****************************************************************************
%% Author: Gerd Neugebauer
%%-----------------------------------------------------------------------------


%------------------------------------------------------------------------------
\section{Running \ExBib}
%@author Gerd Neugebauer

Currently \ExBib\ can be run from the command line. In this respect it
is more or less identical to \BibTeX\ and can be used as a plug-in
replacement. In addition the features of \BibTeX~8 are present as
well. The marks in the margin indicate where the different features
are coming from.

The exxecutable for \ExBib\ is called \Prog{exbib}. It is a wrapper
for the real program. It comes as shell script \Prog{exbib} and the
Windows batch script \Prog{exbib.bat}. After a successful installation
those scripts should be resent on your execution path.


\subsection{Command Line Parameters}
%@author Gerd Neugebauer

The invocation of the executable \Prog{exbib} can be controlled by
large number of command line arguments. Those command line arguments
are described in the following sections.

\subsubsection{The Aux File}

\begin{description}
\item[\Arg{file}]\ \\
  This parameter contains the aux file to read from. A file name may not
  start with minus sign. It has no default.
  \begin{lstlisting}{}
    # exbib doc.aux
    This is exbib, Version 0.1
  \end{lstlisting}
  
\item[\CLI{} \Arg{file}]
\item[\CLI{-} \Arg{file}]\ \\
  This parameter terminates the intercepts processing of arguments. The
  next argument -- if present -- is interpreted as input file. With
  this construction it is possible to process an input file which
  starts with the special character \verb|-|.
  \begin{lstlisting}{}
    # exbib -- doc.aux
    This is exbib, Version 0.1
  \end{lstlisting}

\end{description}

The file name given is used to determine the name of an aux file. This
means that it is either the name of an aux file or the base name which
is augmented by the extension \texttt{.aux} to find the aux file.

The main control information is taken from this aux file. This means
it contains the foloowing items:

\begin{itemize}
\item The database files to consult.
\item The citations to extract.
\item The bst to use for formatting.
\item References to other aux files to consult as well.
\end{itemize}

\subsubsection{Version Information and Help}

\begin{description}
\item[\CLI{-copying}]\ \\
  This command line option produces a copyright notice on the standard
  output stream and terminates the program afterwards.
  \begin{lstlisting}{}
# exbib --copying
                 GNU LESSER GENERAL PUBLIC LICENSE
                      Version 2.1, February 1999
Copyright (C) 1991, 1999 Free Software Foundation, Inc.
    59 Temple Place, Suite 330, Boston, MA  02111-1307  USA
Everyone is permitted to copy and distribute verbatim copies
of this license document, but changing it is not allowed.
[This is the first released version of the Lesser GPL.  It also counts
as the successor of the GNU Library Public License, version 2, hence
the version number 2.1.]
                           Preamble
 The licenses for most software are designed to take away your
freedom to share and change it.  By contrast, the GNU General Public
Licenses are intended to guarantee your freedom to share and change
free software--to make sure the software is free for all its users.
    ...
  \end{lstlisting}

\item[\CLI{h}]
\item[\CLI{?}]
\item[\CLI{-help}]\ \\
  This command line option produces a short usage description on the
  standard output stream and terminates the program afterwards.
  \begin{lstlisting}{}
# exbib --help
This is exbib, Version 0.1
Usage: exbib <options> file
The following options are supported:
        -[-] <file>
                Use this argument as file name -- even when it looks like an option.
        --trad[itional] | -7
                operate in the original 7-bit mode.
        --8[bit] | -8
                force 8-bit mode, no CS file used.
    ...
\end{lstlisting}
  
\item[\CLI{-release}]\ \\
  This command prints the release number to stdout and exits the
  program. This can be used to enable external programs to easily
  determine the version number of \ExBib.
\begin{lstlisting}{}
# exbib --release
0.1
#
\end{lstlisting}

\item[\CLI{-version}]\ \\
  This command line parameter forces that the version information is
  written to standard output and the program is
  terminated.\index{version} The version of \ExBib\ is shown. The
  following example shows a possible invocation and the resulting
  output:
\begin{lstlisting}{}
# exbib --version
This is exbib, Version 0.1
Copyright (C) 2002-2008 Gerd Neugebauer (mailto:gene@gerd-neugebauer.de).
There is NO warranty.  Redistribution of this software is
covered by the terms of the GNU Library General Public License.
For more information about these matters, use the command line
switch -copying.
#
\end{lstlisting}
\end{description}


\subsubsection{Internationalization}

\begin{description}
\item[\CLI{L} \Arg{lang}]
\item[\CLI{-language} \Arg{lang}]\ \\
  This command line option switches the language to the given
  language. The argument is a two-letter ISO code for a language. For
  instance the value \texttt{en} represents English and \texttt{de}
  represents German.

  The language is used to select the appropriate messages for logging
  and error messages. If the given language is not supported English
  is silently used as fallback.
\begin{lstlisting}{}
# exbib --language de --help
Dies ist exbib, Version 0.1
Copyright (C) 2002-2008 Gerd Neugebauer (mailto:gene@gerd-neugebauer.de).
There is NO warranty.  Redistribution of this software is
covered by the terms of the GNU Library General Public License.
For more information about these matters, use the command line
switch -copying.
#
\end{lstlisting}
\end{description}


\subsubsection{Configurations}
\index{configuration|(}

\ExBib\ is highly configurable. The whole system is assembled from
components at run time. The assembly is controlled from a set of
configuration files. There is one central configuration file which
acts as entry point.

\begin{description}
\item[\CLI{c} \Arg{config}]
\item[\CLI{-configuration} \Arg{config}]\ \\
  Set the configuration file to be used for assenbling \ExBib. The
  default is the configuration \texttt{exbib}.
\begin{lstlisting}{}
# exbib --configuration bibtex099 doc.aux
This is exbib, Version 0.1
...
\end{lstlisting}

\item[\CLI{-bibtex}]
\item[\CLI{-strict}]\ \\
  Use the configuration \texttt{bibtex099} for assembling \ExBib.
\begin{lstlisting}{}
# exbib --bibtex doc.aux
This is exbib, Version 0.1
...
\end{lstlisting}
\end{description}

The following configurations are present in the distribution.

\begin{description}
\item[exbib]\ \\
  This is the default configuration which includes all features
  described in this reference manual.
\item[bibtex099]\ \\
  This is the configuration for backward compatibility. It emulates
  the features of \BibTeX~0.99c as closely as possible. The extended
  features of \ExBib\ are not present in this configuration.
\end{description}

\index{configuration|)}

\subsubsection{Encodings}
\index{encoding|(}

The internal representation of characters uses Unicode. In general it
is necessary to translate from and to the internal representation when
reading and writing files. For this purpose the encodings to be used
can be configured.

The default is to use the default encoding for the platform \ExBib\ is
currently running. Thus it is not necessary to specify an encoding at
all.

It is guaranteed that at least the following encodings are present on
your system:

\begin{description}
\item[US-ASCII\index{encoding!US-ASCII}] 
  Seven-bit ASCII.
\item[ISO-8859-1\index{encoding!ISO-8859-1}] 
  ISO Latin Alphabet 1
\item[UTF-8\index{encoding!UTF8}] 
  Eight-bit UCS Transformation Format.
\item[UTF-16BE\index{encoding!UTF16BE}] 
  Sixteen-bit UCS Transformation Format in big-endian byte order.
\item[UTF-16LE\index{encoding!UTF16LE}] 
  Sixteen-bit UCS Transformation Format in little-endian byte order.
\item[UTF\index{encoding!UTF}] Sixteen-bit UCS Transformation Format;
  the byte order identified by an optional byte-order mark.
\end{description}

The following list has been obtained at the time of writing this
document (on a Windows system):

\noindent
\begin{multicols}5\obeylines\scriptsize\parindent=0pt
  Big5\index{encoding!Big5}
  Big5-HKSCS\index{encoding!Big5-HKSCS}
  EUC-JP\index{encoding!EUC-JP}
  EUC-KR\index{encoding!EUC-KR}
  GB18030\index{encoding!GB18030}
  GB2312\index{encoding!GB2312}
  GBK\index{encoding!GBK}
  IBM-Thai\index{encoding!IBM-Thai}
  IBM00858\index{encoding!IBM00858}
  IBM01140\index{encoding!IBM01140}
  IBM01141\index{encoding!IBM01141}
  IBM01142\index{encoding!IBM01142}
  IBM01143\index{encoding!IBM01143}
  IBM01144\index{encoding!IBM01144}
  IBM01145\index{encoding!IBM01145}
  IBM01146\index{encoding!IBM01146}
  IBM01147\index{encoding!IBM01147}
  IBM01148\index{encoding!IBM01148}
  IBM01149\index{encoding!IBM01149}
  IBM037\index{encoding!IBM037}
  IBM1026\index{encoding!IBM1026}
  IBM1047\index{encoding!IBM1047}
  IBM273\index{encoding!IBM273}
  IBM277\index{encoding!IBM277}
  IBM278\index{encoding!IBM278}
  IBM280\index{encoding!IBM280}
  IBM284\index{encoding!IBM284}
  IBM285\index{encoding!IBM285}
  IBM297\index{encoding!IBM297}
  IBM420\index{encoding!IBM420}
  IBM424\index{encoding!IBM424}
  IBM437\index{encoding!IBM437}
  IBM500\index{encoding!IBM500}
  IBM775\index{encoding!IBM775}
  IBM850\index{encoding!IBM850}
  IBM852\index{encoding!IBM852}
  IBM855\index{encoding!IBM855}
  IBM857\index{encoding!IBM857}
  IBM860\index{encoding!IBM860}
  IBM861\index{encoding!IBM861}
  IBM862\index{encoding!IBM862}
  IBM863\index{encoding!IBM863}
  IBM864\index{encoding!IBM864}
  IBM865\index{encoding!IBM865}
  IBM866\index{encoding!IBM866}
  IBM868\index{encoding!IBM868}
  IBM869\index{encoding!IBM869}
  IBM870\index{encoding!IBM870}
  IBM871\index{encoding!IBM871}
  IBM918\index{encoding!IBM918}
  ISO-2022-CN\index{encoding!ISO-2022-CN}
  ISO-2022-JP\index{encoding!ISO-2022-JP}
  ISO-2022-JP-2\index{encoding!ISO-2022-JP-2}
  ISO-2022-KR\index{encoding!ISO-2022-KR}
  ISO-8859-1\index{encoding!ISO-8859-1}
  ISO-8859-13\index{encoding!ISO-8859-13}
  ISO-8859-15\index{encoding!ISO-8859-15}
  ISO-8859-2\index{encoding!ISO-8859-2}
  ISO-8859-3\index{encoding!ISO-8859-3}
  ISO-8859-4\index{encoding!ISO-8859-4}
  ISO-8859-5\index{encoding!ISO-8859-5}
  ISO-8859-6\index{encoding!ISO-8859-6}
  ISO-8859-7\index{encoding!ISO-8859-7}
  ISO-8859-8\index{encoding!ISO-8859-8}
  ISO-8859-9\index{encoding!ISO-8859-9}
  JIS\_X0201\index{encoding!JIS\_X0201}
  JIS\_X0212-1990\index{encoding!JIS\_X0212-1990}
  KOI8-R\index{encoding!KOI8-R}
  KOI8-U\index{encoding!KOI8-U}
  Shift\_JIS\index{encoding!Shift\_JIS}
  TIS-620\index{encoding!TIS-620}
  US-ASCII\index{encoding!US-ASCII}
  UTF-16\index{encoding!UTF-16}
  UTF-16BE\index{encoding!UTF-16BE}
  UTF-16LE\index{encoding!UTF-16LE}
  UTF-32\index{encoding!UTF-32}
  UTF-32BE\index{encoding!UTF-32BE}
  UTF-32LE\index{encoding!UTF-32LE}
  UTF-8\index{encoding!UTF-8}
  windows-1250\index{encoding!windows-1250}
  windows-1251\index{encoding!windows-1251}
  windows-1252\index{encoding!windows-1252}
  windows-1253\index{encoding!windows-1253}
  windows-1254\index{encoding!windows-1254}
  windows-1255\index{encoding!windows-1255}
  windows-1256\index{encoding!windows-1256}
  windows-1257\index{encoding!windows-1257}
  windows-1258\index{encoding!windows-1258}
  windows-31j\index{encoding!windows-31j}
  x-Big5-Solaris\index{encoding!x-Big5-Solaris}
  x-euc-jp-linux\index{encoding!x-euc-jp-linux}
  x-EUC-TW\index{encoding!x-EUC-TW}
  x-eucJP-Open\index{encoding!x-eucJP-Open}
  x-IBM1006\index{encoding!x-IBM1006}
  x-IBM1025\index{encoding!x-IBM1025}
  x-IBM1046\index{encoding!x-IBM1046}
  x-IBM1097\index{encoding!x-IBM1097}
  x-IBM1098\index{encoding!x-IBM1098}
  x-IBM1112\index{encoding!x-IBM1112}
  x-IBM1122\index{encoding!x-IBM1122}
  x-IBM1123\index{encoding!x-IBM1123}
  x-IBM1124\index{encoding!x-IBM1124}
  x-IBM1381\index{encoding!x-IBM1381}
  x-IBM1383\index{encoding!x-IBM1383}
  x-IBM33722\index{encoding!x-IBM33722}
  x-IBM737\index{encoding!x-IBM737}
  x-IBM834\index{encoding!x-IBM834}
  x-IBM856\index{encoding!x-IBM856}
  x-IBM874\index{encoding!x-IBM874}
  x-IBM875\index{encoding!x-IBM875}
  x-IBM921\index{encoding!x-IBM921}
  x-IBM922\index{encoding!x-IBM922}
  x-IBM930\index{encoding!x-IBM930}
  x-IBM933\index{encoding!x-IBM933}
  x-IBM935\index{encoding!x-IBM935}
  x-IBM937\index{encoding!x-IBM937}
  x-IBM939\index{encoding!x-IBM939}
  x-IBM942\index{encoding!x-IBM942}
  x-IBM942C\index{encoding!x-IBM942C}
  x-IBM943\index{encoding!x-IBM943}
  x-IBM943C\index{encoding!x-IBM943C}
  x-IBM948\index{encoding!x-IBM948}
  x-IBM949\index{encoding!x-IBM949}
  x-IBM949C\index{encoding!x-IBM949C}
  x-IBM950\index{encoding!x-IBM950}
  x-IBM964\index{encoding!x-IBM964}
  x-IBM970\index{encoding!x-IBM970}
  x-ISCII91\index{encoding!x-ISCII91}
  x-ISO-2022-CN-CNS\index{encoding!x-ISO-2022-CN-CNS}
  x-ISO-2022-CN-GB\index{encoding!x-ISO-2022-CN-GB}
  x-iso-8859-11\index{encoding!x-iso-8859-11}
  x-JIS0208\index{encoding!x-JIS0208}
  x-JISAutoDetect\index{encoding!x-JISAutoDetect}
  x-Johab\index{encoding!x-Johab}
  x-MacArabic\index{encoding!x-MacArabic}
  x-MacCentralEurope\index{encoding!x-MacCentralEurope}
  x-MacCroatian\index{encoding!x-MacCroatian}
  x-MacCyrillic\index{encoding!x-MacCyrillic}
  x-MacDingbat\index{encoding!x-MacDingbat}
  x-MacGreek\index{encoding!x-MacGreek}
  x-MacHebrew\index{encoding!x-MacHebrew}
  x-MacIceland\index{encoding!x-MacIceland}
  x-MacRoman\index{encoding!x-MacRoman}
  x-MacRomania\index{encoding!x-MacRomania}
  x-MacSymbol\index{encoding!x-MacSymbol}
  x-MacThai\index{encoding!x-MacThai}
  x-MacTurkish\index{encoding!x-MacTurkish}
  x-MacUkraine\index{encoding!x-MacUkraine}
  x-MS950-HKSCS\index{encoding!x-MS950-HKSCS}
  x-mswin-936\index{encoding!x-mswin-936}
  x-PCK\index{encoding!x-PCK}
  x-UTF-16LE-BOM\index{encoding!x-UTF-16LE-BOM}
  X-UTF-32BE-BOM\index{encoding!X-UTF-32BE-BOM}
  X-UTF-32LE-BOM\index{encoding!X-UTF-32LE-BOM}
  x-windows-50220\index{encoding!x-windows-50220}
  x-windows-50221\index{encoding!x-windows-50221}
  x-windows-874\index{encoding!x-windows-874}
  x-windows-949\index{encoding!x-windows-949}
  x-windows-950\index{encoding!x-windows-950}
  x-windows-iso2022jp\index{encoding!x-windows-iso2022jp}
\end{multicols}

The following command line options are related to encodings:

\begin{description}
\item[\CLI{-availableCharsets}]\ \\
  This instruction lists the available character sets on standard
  output and exits the program.
\begin{lstlisting}{}
# exbib --availableCharsets
Big5
Big5-HKSCS
EUC-JP
EUC-KR
GB18030
GB2312
GBK
IBM-Thai
IBM00858
IBM01140
...
\end{lstlisting}

\item[\CLI{E} \Arg{enc}]
\item[\CLI{-bib-encoding} \Arg{enc}]
\item[\CLI{-bib.encoding} \Arg{enc}]\ \\
  Set the configuration for reading bib database files. The encoding
  needs to be a valid character set.
\begin{lstlisting}{}
# exbib --bib-encoding=UTF8 doc.aux
...
\end{lstlisting}

\item[\CLI{e} \Arg{enc}]
\item[\CLI{-encoding} \Arg{enc}]\ \\

  \INCOMPLETE
\begin{lstlisting}{}
# exbib --encoding=UTF8 doc.aux
...
\end{lstlisting}

\end{description}
%        --e[ncoding] | -e <enc>
%        	Nutze das gegebene Encoding f�r die Ausgabedatei.
\index{encoding|)}

\subsubsection{CS Files}%
\index{csf|(}

\begin{description}
\item[\CLI{-csfile} \Arg{csfile}]\ \\

  \INCOMPLETE
\begin{lstlisting}{}
# exbib --csfile=iso8859-7.csf doc.aux
...
\end{lstlisting}
%        --cs[file] <csfile>
%        	Nutze das csf zur Definition von Zeichen und Sortierung.

\item[\CLI{7}]
\item[\CLI{-traditional}]\ \\
  \INCOMPLETE
\begin{lstlisting}{}
# exbib --traditional doc.aux
...
\end{lstlisting}
%        --trad[itional] | -7
%        	arbeite im originalen 7-Bit-Modus.

\item[\CLI{8}]
\item[\CLI{-8bit}]\ \\
  \INCOMPLETE
\begin{lstlisting}{}
# exbib --8bit doc.aux
...
\end{lstlisting}
%        --8[bit] | -8
%        	Erzwinge  8-Bit-Modus, es wird keine CS-Datei eingesetzt.
\end{description}
\index{csf|)}

\subsubsection{Redirecting Output}

\begin{description}
\item[\CLI{l} \Arg{file}]
\item[\CLI{-logfile} \Arg{file}]\ \\
  This option redirects the log output to the given file. The default
  name of the log file is derived from the base name of the aux file
  by appending \texttt{.blg}. This option overwrites this default.
  behaviour.
\begin{lstlisting}{}
# exbib --logfile=my.log doc.aux
...
\end{lstlisting}

  If the given file name is the value \texttt{-} then the output is
  sent to stdout.
\begin{lstlisting}{}
# exbib --logfile=- doc.aux
...
\end{lstlisting}

  If the given file name is empty then the log output is discarted.
\begin{lstlisting}{}
# exbib --logfile= doc.aux
...
\end{lstlisting}

\item[\CLI{o} \Arg{file}]
\item[\CLI{-output} \Arg{file}]
\item[\CLI{-outfile} \Arg{file}]\ \\
  This option redirects the output to the given file. The default
  name of the output file is derived from the base name of the aux file
  by appending \texttt{.bbl}. This option overwrites this default.
  behaviour.
\begin{lstlisting}{}
# exbib --outfile=my.out doc.aux
...
\end{lstlisting}

  If the given file name is the value \texttt{-} then the output is
  sent to stdout.
\begin{lstlisting}{}
# exbib --outfile=- doc.aux
...
\end{lstlisting}

  If the given file name is empty then the output is discarted.
\begin{lstlisting}{}
# exbib --outfile= doc.aux
...
\end{lstlisting}

\end{description}


\subsubsection{Changing the Style}

\begin{description}
\item[\CLI{b} \Arg{style}]
\item[\CLI{-bst} \Arg{style}]\ \\
  This option sets the name of the bib style to be used. The bib style
  is normally read from the aux file. This instruction overrules
  whatever the aux file contains.
\begin{lstlisting}{}
# exbib --bst=alpha doc.aux
...
\end{lstlisting}\index{bst!alpha}

\end{description}

\subsubsection{Minimum Crossreferences}

\begin{description}
\item[\CLI{M} \Arg{n}]
\item[\CLI{-min-crossrefs} \Arg{n}]
\item[\CLI{-min.crossrefs} \Arg{n}]
  This option sets the minimum number of crossreferences before an
  entry is not collaped.
\begin{lstlisting}{}
# exbib --min-crossrefs=4 doc.aux
...
\end{lstlisting}

\end{description}

\subsubsection{Naming the Program}

\begin{description}
\item[\CLI{p} \Arg{name}]
\item[\CLI{-progname} \Arg{name}]
\item[\CLI{-program-name} \Arg{name}]
\item[\CLI{-program.name} \Arg{name}]\ \\
  This option sets the name of the porogram. Thus it is possible to
  influence how the program calls itself in logging and error messages
  from outside .
\begin{lstlisting}{}
# exbib --progname=BibTeX doc.aux
This is BibTeX, Version 0.1
...
\end{lstlisting}

\end{description}


\subsubsection{Tracing and Debugging}

\begin{description}
\item[\CLI{d} \Arg{mode}]
\item[\CLI{-debug} \Arg{mode}]\ \\
\INCOMPLETE

\end{description}
%        --d[ebug] | -d
%        	Arbeite im Debug-Modus.

\begin{description}
\item[\CLI{q}]
\item[\CLI{-terse}]
\item[\CLI{-quiet}]\ \\
  This option switches the operation to quiet mode. Nearly all
  informative messages are suppressed oin stdanard output.
  Nevertheless they can be found in the log file -- if one is written.
\begin{lstlisting}{}
# exbib --quiet doc.aux
#
\end{lstlisting}

\item[\CLI{t}]
\item[\CLI{-trace}]\ \\
  Write a detailed log of internal operations to the log file.  The
  tracing can be very useful when you try to understand the operations
  of the bst interpreter.
  
  Note that this option can drastically decrease the performance of
  operation.
\begin{lstlisting}{}
# exbib --trace doc.aux
This is BibTeX, Version 0.1
...
\end{lstlisting}

\item[\CLI{v}]
\item[\CLI{-verbose}]\ \\
  This option switches the operation to verbose mode. Some more
  informative messaged might be presented during the operation.
\begin{lstlisting}{}
# exbib --verbose doc.aux
#
\end{lstlisting}
\end{description}


\subsubsection{Ignored Options}

\IM{8x} Several command line options have a special meaning in
\BibTeX~8 without a corresponding pendant in \ExBib. Most of them are
related to memory allocation. In \ExBib\ the memory allocation is
fully dynamic and no predefined sizes are necessary.

For compatibility those options are silently ignored:
\begin{description}
\item[\CLI{s}]
\item[\CLI{-statistics}]
\item[\CLI{B}]
\item[\CLI{-big}]
\item[\CLI{H}]
\item[\CLI{-huge}]
\item[\CLI{W}]
\item[\CLI{-wolfgang}]
\item[\CLI{-mcites}]
\item[\CLI{-mentints}]
\item[\CLI{-mentstrs}]
\item[\CLI{-mfields}]
\item[\CLI{-mpool}]
\item[\CLI{-mstrings}]
\item[\CLI{-mwizfuns}]
\end{description}


\subsection{Abbreviation of Long Parameters}
%@author Gerd Neugebauer

Command line parameters can be abbreviated up to a unique prefix --
and sometimes even more. Thus the following invocations are
equivalent:\index{prefix}

\begin{verbatim}
  exbib --vers
  exbib --versi
  exbib --versio
  exbib --version  
\end{verbatim}


\endinput
%
% Local Variables: 
% mode: latex
% TeX-master: nil
% End: 
%        --r[elease]
%        	Zeige die Versionsnummer und beende das Programm.
