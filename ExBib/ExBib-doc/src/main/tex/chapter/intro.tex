%%*****************************************************************************
%% $Id: bst-language.tex,v 0.00 2008/04/30 23:26:21 gene Exp $
%%*****************************************************************************
%% Author: Gerd Neugebauer
%%-----------------------------------------------------------------------------

\chapter{Introduction}
%@author Gerd Neugebauer

\ExTeX{} aims at providing a high-quality typesetting system. The
development of \ExTeX\ has been inspired by the experiences with \TeX.
The focus lies on an open design and a high degree of configurability.

A tight integration of several components is one of the possibilities
opened by \ExTeX. To work into this direction \ExBib\ has been
implemented. It is a plug-in replacement for \BibTeX~0.99c\
\cite{btxdoc,btxhak} or \BibTeX~8.

\section{This Document}

This document is meant to be a reference document. It should contain
all information necessary to know. It is not meant to be a tutorial.
Thus do not expect tutorial type material in this document.


\section{Web Site}
%@author Gerd Neugebauer

There is a web site devoted to \ExTeX. \index{WWW}\index{Web Site}This
web site can be reached via the URL

\begin{quotation}
  \url{http://www.extex.org}
\end{quotation}


\section{Mailing Lists}
%@author Gerd Neugebauer

If you are ready to try \ExTeX{} you might as well want to join a
mailing list to get in contact with the community.\index{Mailing list}

\begin{quotation}
  \url{http://www.dante.de/listman/extex}
\end{quotation}


\section{Reporting Bugs}
%@author Gerd Neugebauer


If you find any bugs in \ExTeX\ you can submit them 
%either 
via a HTML form.
% or via email. 
You can find the HTML form at
\begin{quotation}
  \url{http://www.extex.org/bugs}
\end{quotation}
%Emails containing the description can be sent to
%\begin{quotation}
%  \href{mailto:extex-bugs@dante.de}{extex-bugs@dante.de}
%\end{quotation}

Please include in your description 
\begin{itemize}
\item the source of a \emph{minimal} example showing the problem
\item the log file resulting from running this example
\item a description why you think that something went wrong and what
  the expected result would be
\item a description of the environment you are using (host
  architecture, operating system, Java version)
\end{itemize}

\endinput
%
% Local Variables: 
% mode: latex
% TeX-master: "exbib-users"
% End: 
