%%*****************************************************************************
%% $Id: bst-language.tex,v 0.00 2008/04/30 23:26:21 gene Exp $
%%*****************************************************************************
%% Author: Gerd Neugebauer
%%-----------------------------------------------------------------------------

\chapter{The Data Base}


\section{Syntax}

The data base in \BibTeX\index{BibTeX@\BibTeX} style consists of a
simple text file. \IM{08x1}

The following characters have a special meaning for the \BibTeX\
syntax:
\begin{verbatim}
    @ { } ( ) , # " =
\end{verbatim}
Any other character is treated equally as ordinary character.

An instruction is started with an at sign (@) followed by its name.
The name is composed of upper or lowercase letters and digits.

Whatever follows the name of the instruction depends on the
instruction. In most cases the parameters for the instruction are
following. They are enclosed in braces. For compatibility with
Scribe\index{Scribe} parentheses are sometimes allowed instead of the
braces.


\section{The \texttt{@input} Instruction}%
\index{@input|(}

Sometimes it might be desirable to split a database into several
segements. This is supported by the ability to pass inseveral
databases via the aux file. The \texttt{@input} instruction provides
another mechanism for the same which acts on the level of the database
files. \IM{x1}

The instruction takes as argument a resource name. It includes the
content as if it where present at the place of the instruction.

\begin{verbatim}
  @input(some/other/resource)
\end{verbatim}

\index{@input|)}

\section{Entries}

Any instruction which has no special meaning is considered to be an
entry in the database.
\IM{08x1}

\INCOMPLETE

\section{Names}\label{sec:names}
\IM{08x1}

Names are especially complicated and deserve a description of their
own.

\INCOMPLETE


\section{Comments}

Anything outside of entries and other declarations are considered as a
comment -- and mainly ignored. Thus you can put anything in between
the entries.

\IM{081} There is one special tag to mark comments. It is the tag
\texttt{@comment}. Since anything outside of declarations is already a
comment. Is has been considered sufficient to ignore the tag in the
input stream.
\begin{lstlisting}{}
  @comment
\end{lstlisting}

Unfortunately in the age of internet it is desirable to include email
addresses into comment -- and those may contain an @. Thus the
definition of the \texttt{@comment} declaration is slightly different
from \BibTeX\index{BibTeX@\BibTeX}:
\IM{x}
\begin{itemize}
\item If the next non-space character is an opening brace (\verb|{|)
  then a block is read and treated as comment. This means that the
  block can contain arbitrary characters -- especially the @ sign.
  On the other side the block needs to have balanced braces.
\begin{lstlisting}{}
  @comment{ This is a comment with embedded @ }
\end{lstlisting}
    
\item If the next non-space character is not an open brace character
  then just the tag is ignored.
\begin{lstlisting}{}
  @comment This is a comment
\end{lstlisting}
  
\end{itemize}


\section{The \texttt{@alias} Instruction}
\IM{x1}

\begin{verbatim}
  @alias( abc = def )
\end{verbatim}

\INCOMPLETE

\section{The \texttt{@modify} Instruction}

The \texttt{@modify} directive can be used to alter the content of
certain fields in an entry. The question is why would such a directive
be necessary when you could simply alter the entry itself. The answer
is that the entry might not be under your control. It might be
contained in another file which is included via the \texttt{@include}
directive.
\IM{x1}

\begin{verbatim}
  @modify( abc, author = {A.U. Thor} )
\end{verbatim}

\INCOMPLETE

\section{The \texttt{@string} Instruction}
\IM{08x1}

\begin{verbatim}
  @string( abc = {The value} )
\end{verbatim}

\INCOMPLETE

\section{The \texttt{@preamble} Instruction}
\IM{08x1}

\begin{verbatim}
  @preamble( "\providecommand\BibTeX{\textsc{Bib}\TeX}" )
\end{verbatim}

\INCOMPLETE


\endinput
%
% Local Variables: 
% mode: latex
% TeX-master: "../exbib-manual"
% End: 
