%%*****************************************************************************
%% $Id: bst-language.tex,v 0.00 2008/04/30 23:26:21 gene Exp $
%%*****************************************************************************
%% Author: Gerd Neugebauer
%%-----------------------------------------------------------------------------

\chapter{The Data Base}


\section{Syntax}

The data base in \BibTeX\ style consists of a simple text file.
\IM{08x1}

The following characters have a special meaning for the \BibTeX\
syntax:
\begin{verbatim}
    @ { } ( ) , # " =
\end{verbatim}
Any other character is treated equally as ordinary character.

\INCOMPLETE


\section{\texttt{@input}}
\IM{x1}

\begin{verbatim}
  @input(some/other/resource)
\end{verbatim}

\INCOMPLETE

\section{Entries}
\IM{08x1}

\INCOMPLETE

\section{Names}\label{sec:names}
\IM{08x1}

Names are especially complicated and deserve a description of their
own.

\INCOMPLETE


\subsection{Comments}

Anything outside of entries and other declarations are considered as a
comment -- and mainly ignored. Thus you can put anything in between
the entries.

\IM{081} There is one special tag to mark comments. It is the tag
\texttt{@comment}. Since anything outside of declarations is already a
comment. Is has been considered sufficient to ignore the tag in the
input stream.
\begin{lstlisting}{}
  @comment
\end{lstlisting}

Unfortunately in the age of internet it is desirable to include email
addresses into comment -- and those may contain an @. Thus the
definition of the \texttt{@comment} declaration is slightly different
from \BibTeX:
\IM{x}
\begin{itemize}
\item If the next non-space character is an opening brace (\verb|{|)
  then a block is read and treated as comment. This means that the
  block can contain arbitrary characters -- especially the @ sign.
  On the other side the block needs to have balanced braces.
\begin{lstlisting}{}
  @comment{ This is a comment with embedded @ }
\end{lstlisting}
    
\item If the next non-space character is not an open brace character
  then just the tag is ignored.
\begin{lstlisting}{}
  @comment This is a comment
\end{lstlisting}
  
\end{itemize}


\section{\texttt{@alias}}
\IM{x1}

\begin{verbatim}
  @alias( abc = def )
\end{verbatim}

\INCOMPLETE

\section{\texttt{@modify}}
\IM{x1}
The \texttt{@modify} directive can be used to alter the content of
certain fields in an entry. The question is why would such a directive
be necessary when you could simply alter the entry itself. The answer
is that the entry might not be under your control. It might be
contained in another file which is included via the \texttt{@include}
directive or

\begin{verbatim}
  @modify( abc, author = {A.U. Thor} )
\end{verbatim}

\INCOMPLETE

\section{\texttt{@string}}
\IM{08x1}

\begin{verbatim}
  @string( abc = {The value} )
\end{verbatim}

\INCOMPLETE

\section{\texttt{@preamble}}
\IM{08x1}

\begin{verbatim}
  @preamble( "\providecommand\BibTeX{Bib\TeX}" )
\end{verbatim}

\INCOMPLETE


\endinput
%
% Local Variables: 
% mode: latex
% TeX-master: "../exbib-manual"
% End: 
