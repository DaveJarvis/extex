%%^^A%%%%%%%%%%%%%%%%%%%%%%%%%%%%%%%%%%%%%%%%%%%%%%%%%%%%%%%%%%%%%%%%%%%%%%%
%%^^A $Id$
%%^^A%%%%%%%%%%%%%%%%%%%%%%%%%%%%%%%%%%%%%%%%%%%%%%%%%%%%%%%%%%%%%%%%%%%%%%%
%\iffalse
%% Purpose:
%%      productbox.dtx   provides  a  style  file   for  typesetting  a
%%      three-diemnsional product box. This product box can be rendered
%%      as it is standing on a surface and  some light is shed onto it.
%%      Alternatively  it can be typeset as a wireframe  to be  cut out
%%      and glued together. This will lead to a physical product box.
%%
%% Documentation:
%%      The  documentation  can  be  generated  from the  original file
%%      productbox.dtx  with  the  doc style/package.  LaTeX  the  file    
%%      productbox.tex to get the full documentation in pdf format. 
%%
%% Prerequisite:
%%      This package  needs  PGF and  TikZ  to be present  at least  in
%%      version 2.0.
%%
%% Author: Gerd Neugebauer
%%         Im Lerchelsb\"ohl 5
%%         64521 Gro\ss-Gerau
%% Mail:   gene@gerd-neugebauer.de
%%
%% Copyright (C) 2010 Gerd Neugebauer
%%
%%      productbox.dtx is distributed in the hope that it will be useful,
%%      but  WITHOUT ANY WARRANTY.  No  author  or  distributor  accepts
%%      responsibility to anyone for the consequences of using it or for
%%      whether it serves any particular purpose or works at all, unless
%%      he says so in writing.
%% 
%%      Everyone is granted permission to  copy, modify and redistribute
%%      productbox.dtx, provided this copyright notice is preserved  and
%%      any modifications are indicated.
%% 
%\fi
%^^A%%%%%%%%%%%%%%%%%%%%%%%%%%%%%%%%%%%%%%%%%%%%%%%%%%%%%%%%%%%%%%%%%%%%%%%
%
% \title{A \LaTeX\  Style for
%    Typesetting a Three-Dimensional Product Box\thanks{This file%    
%    documents \filename\ version \fileversion\ as of \filedate.}} 
% \author{Gerd Neugebauer\\\small
%	  Im Lerchelsb\"ohl 5\\\small
%	  64521 Gro\ss-Gerau (Germany)\\\small
%	  Email: \texttt{gene@gerd-neugebauer.de}}
%
% \date{{\footnotesize Documentation date: \docdate}}
%
% \maketitle
%
%^^A%%%%%%%%%%%%%%%%%%%%%%%%%%%%%%%%%%%%%%%%%%%%%%%%%%%%%%%%%%%%%%%%%%%%%%%
% \changes{1.0}{2010/12/20}{First public release.}
%^^A%%%%%%%%%%%%%%%%%%%%%%%%%%%%%%%%%%%%%%%%%%%%%%%%%%%%%%%%%%%%%%%%%%%%%%%
%
% \DoNotIndex{\ ,\",\',\.,\[,\\,\],\^,\`,\~,\@,\@dottedtocline}
% \DoNotIndex{\@gobble,\@gobbletwo,\@highpenalty,\@ifnextchar,\@ifstar}
% \DoNotIndex{\@ifundefined,\@namedef,\@nameuse,\@pnumwidth,\@startsection}
% \DoNotIndex{\@starttoc,\@tempdima,\@thefnmark,\@undefined}
% \DoNotIndex{\@ixpt,\@vpt,\@vipt,\@viipt,\@viiipt}
% \DoNotIndex{\@xpt,\@xipt,\@xiipt,\@xivpt,\@xvii,\@xxpt,\@xxvpt}
% \DoNotIndex{\AA,\AE}
% \DoNotIndex{\CurrentOption,\DeclareOption,\H,\L,\LARGE,\LaTeX,\Large}
% \DoNotIndex{\LoadClass,\O,\OE,\OptionNotUsed,\PassOptionsToClass}
% \DoNotIndex{\ProcessOptions,\ProvidesClass,\RequirePackage,\TeX}
% \DoNotIndex{\aa,\addcontentsline,\addpenalty,\advance,\addvspace,\ae}
% \DoNotIndex{\ast,\arabic}
% \DoNotIndex{\b,\baselineskip,\begin,\begingroup,\bf,\bgroup,\egroup}
% \DoNotIndex{\bigskip,\box,\bullet}
% \DoNotIndex{\c,\cal,\catcode,\chardef,\circ,\clearpage,\closein,\closeout}
% \DoNotIndex{\clubpenalty,\csname}
% \DoNotIndex{\d,\def,\dimen,\diamond,\divide,\documentclass,\dp}
% \DoNotIndex{\else,\em,\empty,\end,\endcsname,\endgroup,\endinput,\endlist}
% \DoNotIndex{\expandafter,\fbox,\fi,\footskip,\framebox,\frenchspacing}
% \DoNotIndex{\futurelet,\gdef,\global,\gobble}
% \DoNotIndex{\hangafter,\hangindent,\hbox,\headheight,\headsep,\hfil}
% \DoNotIndex{\hfill,\hrule,\hskip,\hspace,\hss,\ht,\huge,\ifcat,\ifeof}
% \DoNotIndex{\ifdim,\ifx,\ignorespaces,\immediate,\index,\input,\it}
% \DoNotIndex{\itemindent,\itemsep,\jobname,\kern,\l,\labelsep,\labelwidth}
% \DoNotIndex{\large,\leavevmode,\leftmark,\leftskip,\let,\list}
% \DoNotIndex{\llap,\long,\lower}
% \DoNotIndex{\m@th,\makebox,\magstep,\makeindex,\markboth,\mbox,\medskip}
% \DoNotIndex{\newblock,\newcommand,\newcount,\newenvironment,\newfont}
% \DoNotIndex{\newif,\newpage,\newread,\newwrite}
% \DoNotIndex{\nobreak,\noindent,\normalsize,\null,\o,\oe}
% \DoNotIndex{\openin,\openout,\or,\pagestyle,\par,\paragraph,\parbox}
% \DoNotIndex{\parfillskip,\parindent,\parsep,\parskip,\partopsep}
% \DoNotIndex{\penalty,\quad}
% \DoNotIndex{\raggedbottom,\raise,\relax,\renewcommand,\renewenvironment}
% \DoNotIndex{\rightskip,\rm,\rule}
% \DoNotIndex{\sbox,\sc,\setcounter,\settowidth,\sf,\sfcode,\sl}
% \DoNotIndex{\sloppy,\small,\space,\ss,\string,\symbol}
% \DoNotIndex{\t,\tenex,\textstyle,\the,\thepage,\thispagestyle}
% \DoNotIndex{\topmargin,\topsep,\tt,\typeout}
% \DoNotIndex{\u,\unitlength,\usecounter,\v,\varepsilon,\vbox}
% \DoNotIndex{\vfill,\vsize,\vskip,\vspace,\vss}
% \DoNotIndex{\wd,\widowpenalty,\write,\xdef,\z@}
% \DoNotIndex{\usetikzlibrary,\usepackage,\tikzfading,\setkeys}
% \DoNotIndex{\setbox,\scalebox,\pgftransformyslant,\pgftransformyshift}
% \DoNotIndex{\pgftransformyscale,\pgftransformxslant,\pgftransformxshift}
% \DoNotIndex{\pgftransformxscale,\pgfsetcornersarced,\pgfpoint}
% \DoNotIndex{\newbox,\ignorespacesafterend,\foreach,\filldraw}
% \DoNotIndex{\fill,\errmessage,\draw,\define@key,\copy,\clip}
% \DoNotIndex{\RecordChanges,\ProvidesPackage,\NeedsTeXFormat}
% \DoNotIndex{\InputIfFileExists,\EnableCrossrefs,\DocInput}
% \DoNotIndex{\DeleteShortVerb,\CodelineIndex}
%
%^^A%%%%%%%%%%%%%%%%%%%%%%%%%%%%%%%%%%%%%%%%%%%%%%%%%%%%%%%%%%%%%%%%%%%%%%%
% \CheckSum{791}
%%%%%%%%%%%%%%%%%%%%%%%%%%%%%%%%%%%%%%%%%%%%%%%%%%%%%%%%%%%%%%%%%%%%%%%%
%%  \CharacterTable
%%  {Upper-case    \A\B\C\D\E\F\G\H\I\J\K\L\M\N\O\P\Q\R\S\T\U\V\W\X\Y\Z
%%   Lower-case    \a\b\c\d\e\f\g\h\i\j\k\l\m\n\o\p\q\r\s\t\u\v\w\x\y\z
%%   Digits        \0\1\2\3\4\5\6\7\8\9
%%   Exclamation   \!     Double quote  \"     Hash (number) \#
%%   Dollar        \$     Percent       \%     Ampersand     \&
%%   Acute accent  \'     Left paren    \(     Right paren   \)
%%   Asterisk      \*     Plus          \+     Comma         \,
%%   Minus         \-     Point         \.     Solidus       \/
%%   Colon         \:     Semicolon     \;     Less than     \<
%%   Equals        \=     Greater than  \>     Question mark \?
%%   Commercial at \@     Left bracket  \[     Backslash     \\
%%   Right bracket \]     Circumflex    \^     Underscore    \_
%%   Grave accent  \`     Left brace    \{     Vertical bar  \|
%%   Right brace   \}     Tilde         \~}
%%
%^^A%%%%%%%%%%%%%%%%%%%%%%%%%%%%%%%%%%%%%%%%%%%%%%%%%%%%%%%%%%%%%%%%%%%%%%%
%
%    \begin{abstract}
%      \textsf{productbox.dtx} provides a style file for typesetting a
%      three-diemnsional product box. This product box can be rendered
%      as it is standing on a surface and some light is shed onto it.
%      Alternatively it can be typeset as a wireframe to be cut out and
%      glued together. This will lead to a physical product box.
%    \end{abstract}
%
%    \begin{center}%
%      \def\PB(#1,#2)#3{
%        \draw[xshift=-.2mm,yshift=.2mm] (#1,#2) node[white,opacity=.9]{#3};
%        \draw[xshift=.2mm,yshift=-.2mm] (#1,#2) node[black,opacity=.9]{#3};
%        \draw (#1,#2) node[blue]{#3};
%      }%
%    \begin{ProductBox}[mirror=true,flare=true,scale=.6]
%      \begin{Front}
%        \shade [top color=yellow!20!red,bottom color=yellow!80!red]
%          rectangle (88mm,100mm);
%        \PB(44mm,85mm){\Huge\bf Product Box}
%        \draw (77mm,90mm) node[circle,fill=white,rotate=330]{\small\bf\fileversion};
%        \draw (44mm,75mm) node{A \LaTeX\ package for just another kind of boxes};
%    
%        \foreach \x in {1,2,3,4,5,6,7,8,9,10,11,12,13,14,15,16} {
%          \count0=\x \multiply\count0 4
%          \draw [red,line width=.2mm]
%          (0mm,0 mm) .. controls (88mm,\the\count0 mm) and (\x mm,66mm) ..
%          (88mm,\the\count0 mm);
%          \draw [xshift=.4mm,white,line width=.2mm,opacity=.5]
%          (0mm,0 mm) .. controls (88mm,\the\count0 mm) and (\x mm,66mm) ..
%          (88mm,\the\count0 mm);
%        }
%    
%        \clip (0mm,0mm) .. controls (88mm,64mm) and (16mm,66mm) ..
%        (88mm,64mm) -- (88mm,100mm) -- (0mm,100mm) -- cycle;
%        \draw (4mm,66mm) node [white!30!black,anchor=north west] {
%          \begin{minipage}{160mm}\tt\tiny\raggedright
%
%    |\usepackage{productbox}|\\
%    | |\\
%    | |\\ 
%    |\begin{ProductBox}[scale=.8]|\\
%    |  \begin{Front}|\\
%    |    \shade [top color=yellow!20!red,bottom color=yellow!80!red]|\\
%    |      rectangle (88mm,100mm);|\\
%    |    \PB(44mm,85mm){\Huge\bf Product Box}|\\
%    |    \draw (77mm,90mm) node[circle,fill=white]{\small\bf\fileversion};|\\
%    |    \draw (44mm,75mm) node{A \LaTeX\ package for just another kind of boxes};|\\
%    ||\\
%    |    \foreach \x in {1,2,3,4,5,6,7,8,9,10,11,12,13,14,15,16} {|\\
%    |      \count0=\x \multiply\count0 4|\\
%    |      \draw [red,line width=.2mm]|\\
%    |      (0mm,0 mm) .. controls (88mm,\the\count0 mm) and (\x mm,66mm) ..|\\
%    |      (88mm,\the\count0 mm);|\\
%    |      \draw [xshift=.4mm,white,line width=.2mm,opacity=.5]|\\
%    |      (0mm,0 mm) .. controls (88mm,\the\count0 mm) and (\x mm,66mm) ..|\\
%    |      (88mm,\the\count0 mm);|\\
%    |    }|\\
%    |    \clip (0mm,0mm) .. controls (88mm,64mm) and (16mm,66mm) ..|\\
%    |    (88mm,64mm) -- (88mm,100mm) -- (0mm,100mm) -- cycle;|\\
%    |  \end{Front}|\\
%    |  \begin{Top}|\\
%    |    \shade [outer color=yellow!20!red,inner color=yellow!80!red]|\\
%    |      rectangle (88mm,30mm);|\\
%    |    \PB(44mm,15mm){\huge\bf Product Box}|\\
%    |  \end{Top}|\\
%    |  \begin{Left}|\\
%    |    \fill [top color=yellow!20!red,bottom color=yellow!80!red]|\\
%    |      rectangle (30mm,100mm);|
%          \end{minipage}};
%      \end{Front}
%      \begin{Top}
%        \shade [outer color=yellow!20!red,inner color=yellow!80!red]
%          rectangle (88mm,30mm);
%        \PB(44mm,15mm){\huge\bf Product Box}
%      \end{Top}
%      \begin{Left}
%        \fill [top color=yellow!20!red,bottom color=yellow!80!red]
%          rectangle (30mm,100mm);
%        \PB(15mm,90mm){
%          \begin{minipage}{25mm}\centering
%            \Large\bf Product Box
%          \end{minipage}}
%        \draw (15mm,25mm) node {
%          \begin{minipage}{25mm}\sf\tiny\raggedright\parskip=.75ex
%            Copyright (C) 2010 Gerd Neugebauer
%    
%            productbox.dtx is distributed in the hope that it will be useful,
%            but  WITHOUT ANY WARRANTY.  No  author  or  distributor  accepts
%            responsibility to anyone for the consequences of using it or for
%            whether it serves any particular purpose or works at all, unless
%            he says so in writing.
%    
%            Everyone is granted permission to  copy, modify and redistribute
%            productbox.dtx, provided this copyright notice is preserved  and
%            any modifications are indicated.
%          \end{minipage}};
%      \end{Left}
%    \end{ProductBox}
%    \end{center}
%    
%    \newpage
%    \tableofcontents
%    \newpage
%
%    \section{The User Interface}
%
%    This style is based on Ti\emph{k}Z and \textsc{pgf} at least in
%    version 2.0. It is best used with a \LaTeX\ variant which is able
%    to produce PDF. If no proper tool chain is used then some of the
%    effects should not be used.
%    
%
%    \DescribeEnv{ProductBox}
%    The envrionment |ProductBox| provides the central means for
%    producing a product box. The contents is used to define the
%    appearance of the faces. Finally the selected type of output is
%    produced. 
%
%    {\footnotesize\tt
%    |\begin{ProductBox}|\\
%    |  \begin{Front}|\\
%    |    \shade [bottom color=white!30!blue,top color=white]|\\
%    |      (0mm,0mm) rectangle (88mm,100mm);|\\
%    |    \draw (44mm,50mm) node{{\Huge Product Box}};|\\
%    |  \end{Front}|\\
%    |  \begin{Back}|\\
%    |    \shade [top color=yellow!30!red,bottom color=white]|\\
%    |      (0mm,0mm) rectangle (88mm,100mm);|\\
%    |    \draw (44mm,50mm) node{{\large Back}};|\\
%    |  \end{Back}|\\
%    |  \begin{Top}|\\
%    |    \shade [outer color=white!30!red,inner color=white]|\\
%    |      (0mm,0mm) rectangle (88mm,30mm);|\\
%    |    \draw (44mm,15mm) node{{\large Top}};|\\
%    |  \end{Top}|\\
%    |  \begin{Bottom}|\\
%    |    \fill [outer color=red,inner color=black]|\\
%    |      (0mm,0mm) rectangle (88mm,30mm);|\\
%    |    \draw (44mm,15mm) node{{\large Bottom}};|\\
%    |  \end{Bottom}|\\
%    |  \begin{Left}|\\
%    |    \fill [left color=green,right color=yellow]|\\
%    |      (0mm,0mm) rectangle (30mm,100mm);|\\
%    |    \draw (15mm,50mm) node{{\large Left}};|\\
%    |  \end{Left}|\\
%    |  \begin{Right}|\\
%    |    \fill [left color=white,right color=black]|\\
%    |      (0mm,0mm) rectangle (30mm,100mm);|\\
%    |    \draw (15mm,50mm) node{{\large Right}};|\\
%    |  \end{Right}|\\
%    |\end{ProductBox}|
%    }
%
%    \def\sample[#1]{%
%    \begin{ProductBox}[scale=.25,#1]
%      \begin{Front}
%        \shade [bottom color=white!30!blue,top color=white] (0mm,0mm) rectangle (88mm,100mm);
%        \draw (44mm,50mm) node{{\Huge Product Box}};
%      \end{Front}
%      \begin{Back}
%        \shade [top color=yellow!30!red,bottom color=white] (0mm,0mm) rectangle (88mm,100mm);
%        \draw (44mm,50mm) node{{\large Back}};
%      \end{Back}
%      \begin{Top}
%        \shade [outer color=white!30!red,inner color=white] (0mm,0mm) rectangle (88mm,30mm);
%        \draw (44mm,15mm) node{{\large Top}};
%      \end{Top}
%      \begin{Bottom}
%        \fill [outer color=red,inner color=black] (0mm,0mm) rectangle (88mm,30mm);
%        \draw (44mm,15mm) node{{\large Bottom}};
%      \end{Bottom}
%      \begin{Left}
%        \fill [left color=green,right color=yellow] (0mm,0mm) rectangle (30mm,100mm);
%        \draw (15mm,50mm) node{{\large Left}};
%      \end{Left}
%      \begin{Right}
%        \fill [left color=white,right color=black] (0mm,0mm) rectangle (30mm,100mm);
%        \draw (15mm,50mm) node{{\large Right}};
%      \end{Right}
%    \end{ProductBox}%
%    }
%
%    \newcommand\SampleEnv[1]{\noindent\par
%    \fcolorbox{red!30!black}{yellow!30!white}{\begin{minipage}[b]{.98\textwidth}\footnotesize\tt
%    \textcolor{blue}{\char92begin}\char123#1\char125\\
%    \hspace*{2em}...\\
%    \textcolor{blue}{\char92end}\char123#1\char125
%    \end{minipage}}\par}
%
%    \subsection{Content of the Environment}
%
%    The content of the environment is simply evaluated. It can
%    contain any code you like. Useful for the production of a product
%    box are some inner environments. They are called \texttt{Front},
%    \texttt{Back}, \texttt{Left}, \texttt{Right}, \texttt{Top}, and
%    \texttt{Bottom}. They are defined inside the main 
%    environment only. Those environments can be used to specify the
%    contenzs of the respective faces of the box.
%
%    Note that in the three-dimensional rendering will show at most
%    three of the faces. It doies not hurt to define all of them, even
%    if they are not shown at all.
%
%    \DescribeEnv{Front}
%    The environment \texttt{Front} is used to define the content of
%    the font face. The environment prosesses its contents inside a
%    \texttt{tikzpikture} of the default size 88mm$\times$100mm.
%    Usually anything outside of this range is clipped.
%
%    \SampleEnv{Front}
%
%    \DescribeEnv{Back}
%    The environment \texttt{Back} is used to define the content of
%    the back face. The environment prosesses its contents inside a
%    \texttt{tikzpikture} of the default size 88mm$\times$100mm.
%    Usually anything outside of this range is clipped.
%
%    \SampleEnv{Back}
%
%    \DescribeEnv{Left}
%    The environment \texttt{Left} is used to define the content of
%    the left face, i.e. the face left to the front page. The
%    environment prosesses its contents inside a \texttt{tikzpikture}
%    of the default size 30mm$\times$100mm. Usually anything outside
%    of this range is clipped.
%
%    \SampleEnv{Left}
%
%    \DescribeEnv{Right}
%    The environment \texttt{Right} is used to define the content of
%    the right face, i.e. the face right to the front page. The
%    environment prosesses its contents inside a \texttt{tikzpikture}
%    of the default size 30mm$\times$100mm. Usually anything outside of this
%    range is clipped.
%
%    \SampleEnv{Right}
%
%    \DescribeEnv{Top}
%    The environment \texttt{Top} is used to define the content of the
%    top face. The environment prosesses its contents inside a
%    \texttt{tikzpikture} of the default size 88mm$\times$30mm. Usually
%    anything outside of this range is clipped.
%
%    \SampleEnv{Top}
%
%    \DescribeEnv{Bottom}
%    The environment \texttt{Bottom} is used to define the content of the
%    top face. The environment prosesses its contents inside a
%    \texttt{tikzpikture} of the default size 88mm$\times$30mm. Usually
%    anything outside of this range is clipped.
%
%    \SampleEnv{Bottom}
%
%
%    \subsection{Settings and Options of the Environment}
%
%    The environment \texttt{ProductBox} can take some options to
%    influence the appearance of the product box. Those options are
%    comma separated.
%
%    \def\CODE#1{
%    \noindent
%    \fcolorbox{red!30!black}{yellow!30!white}{\begin{minipage}[b]{.98\textwidth}\footnotesize\tt
%    #1
%    \end{minipage}\medskip\par
%    }}%
%    \CODE{\textcolor{blue}{\char92begin}\char123ProductBox\char125[shape=3d]}
%
%    The settings are local to the environment. If an option is not
%    set then the fallback from the global settings are used.
%
%    \DescribeMacro{\ProductBoxSet}
%    The macro |\ProductBoxSet| modify the global setting of the
%    product box style. The arguments are the same like the optional
%    arguments of the environment \texttt{ProductBox} -- but enclosed
%    in braces instead of brackets.
%
%    \CODE{\char92ProductBoxSet\char123shape=3d\char125 }
%
%    The following options can be used to influence the result of the product
%    box.
%
%    \def\SAMPLE[#1]{
%      \noindent
%      \fcolorbox{red!30!black}{yellow!30!white}{\begin{minipage}[b]{.44\textwidth}\footnotesize\tt
%      \def\x{#1}%
%      \def\e{}%
%      \textcolor{blue}{\char92begin}\char123ProductBox\char125 \ifx\x\e\else\textcolor{red}{[#1]}\fi\\
%      \hspace*{2em}\textcolor{blue}{\char92begin}\char123Front\char125\\
%      \hspace*{4em}...\\
%      \hspace*{2em}\textcolor{blue}{\char92end}\char123Front\char125\\
%      \hspace*{2em}\textcolor{blue}{\char92begin}\char123Left\char125\\
%      \hspace*{4em}...\\
%      \hspace*{2em}\textcolor{blue}{\char92end}\char123Left\char125\\
%      \hspace*{2em}...\\
%      \textcolor{blue}{\char92end}\char123ProductBox\char125
%    \end{minipage}}\hfill
%    \begin{minipage}[b]{.54\textwidth}\hfill
%      \sample[#1]
%    \end{minipage}\medskip\par
%    }%
%
%
%    \DescribeMacro{style}
%    The box style determines, how the box is drawn. Several rendering
%    functions are provided to produce different effects. Any value is
%    accepted. Nevertheless unknown box styles will lead to an error message.
%
%    \SAMPLE[style=3D]
%
%    \DescribeMacro{3D}
%    The box style \texttt{3D} is the default. It produces a
%    three-dimensional box rendering with a drop shadow. The option
%    \texttt{3D} is an abbreviation for \texttt{style=3D}. It can also be
%    written as \texttt{3d} or \texttt{3D}.
%
%    \SAMPLE[]
%
%    \DescribeMacro{shadow}
%    The option \texttt{shadow} controls the drawing of the drop shadow in the
%    3D rendering. It is a boolean value taking the values |true| and
%    |false|. The default value is  |true|.
%
%    The option \texttt{shadow} is the abbreviation for |shadow=true|.
%
%    \SAMPLE[shadow=false]
%
%    \DescribeMacro{mirror}
%    The option \texttt{mirror} controls the rendering of the mirror effect in
%    the 3D rendering. It is a boolean value taking the values |true| and
%    |false|. The default value is  |false|.
%
%    The option \texttt{mirror} is the abbreviation for |mirror=true|.
%
%    \SAMPLE[mirror=true]
%
%    \DescribeMacro{flare}
%    The option \texttt{flare} controls the rendering of the flare effect in
%    the 3D rendering. It is a boolean value taking the values |true| and
%    |false|. The default value is  |false|.
%
%    The option \texttt{flare} is the abbreviation for |flare=true|.
%
%    \SAMPLE[flare=true]
%
%    \DescribeMacro{fold}
%    The option \texttt{fold} switches to the box style for rendering a
%    complete wire frame with all faces in place. The option
%    \texttt{fold} is an abbreviation for \texttt{style=fold}.
%
%    \SAMPLE[fold]
%
%    \DescribeMacro{front}
%    The option \texttt{front} switches to the box style for rendering the
%    front face only. The option \texttt{front} is an abbreviation for
%    \texttt{style=front}.
%
%    \SAMPLE[front]
%
%    \DescribeMacro{back}
%    The option \texttt{back} switches to the box style for rendering the
%    back face only. The option \texttt{back} is an abbreviation for
%    \texttt{style=back}.
%
%    \SAMPLE[back]
%
%    \DescribeMacro{left}
%    The option \texttt{left} switches to the box style for rendering the
%    back face only. The option \texttt{left} is an abbreviation for
%    \texttt{style=left}.
%
%    \SAMPLE[left]
%
%    \DescribeMacro{right}
%    The option \texttt{right} switches to the box style for rendering the
%    back face only. The option \texttt{right} is an abbreviation for
%    \texttt{style=right}.
%
%    \SAMPLE[right]
%
%    \DescribeMacro{top}
%    The option \texttt{top} switches to the box style for rendering the
%    back face only. The option \texttt{top} is an abbreviation for
%    \texttt{style=top}.
%
%    \SAMPLE[top]
%
%    \DescribeMacro{bottom}
%    The option \texttt{bottom} switches to the box style for rendering the
%    back face only. The option \texttt{bottom} is an abbreviation for
%    \texttt{style=bottom}.
%
%    \SAMPLE[bottom]
%
%    \DescribeMacro{scale}
%    The option \texttt{scale} controlls the scaling of the whole rendering.
%    It is a number where 1. represents the original size.\footnote{The
%    examples on the right side are normally rendered with a scale of 0.25.}
%
%    \SAMPLE[scale=.1]
%
%    \DescribeMacro{view}
%    The option |view| takes a name of a view definition and activates
%    the appropriate settings. A few views are predefined. The default
%    value is |1|.
%
%    \SAMPLE[view=1]
%
%    \SAMPLE[view=2]
%
%    \SAMPLE[view=3]
%
%    \DescribeMacro{width}
%    The option |width| takes a dimension denoting the width of the box. The
%    default value is 88mm.
%
%    \DescribeMacro{height}
%    The option |height| takes a dimension denoting the height of the box. The
%    default value is 100mm.
%
%    \DescribeMacro{depth}
%    The option |depth| takes a dimension denoting the depth of the box. The
%    default value is 30mm.
%
%    \DescribeMacro{clip}
%    The option \texttt{clip} controls the clipping of the faces to their
%    defined size. If it is turned off then the faces can be oversized
%    leading to strange effects. It is a boolean value taking the
%    values |true| and |false|. The default value is |true|.
%
%
%    \StopEventually
%
%    \newpage
%    \section{The Documentation Driver}
%
%    The documentation driver changes \textsf{productbox.dtx} into a
%    self-extracting documentation. Thus it is possible to run
%    \LaTeX{} on \textsf{productbox.dtx} to produce the documentation.
%
%    The documentation can be adapted in a file named productbox.dcf
%    (documentation configuration). This file can contain instructions
%    for |docstrip|. Especially useful might be the instruction 
%
%    |\OnlyDescription|
%
%    which supresses the generation of the implementation description.
%
%    \begin{macrocode}
%<*driver>
\documentclass{ltxdoc}
\InputIfFileExists{productbox.dcf}{}{}
\usepackage{productbox}
\usepackage{moreverb}
\RecordChanges
\EnableCrossrefs
\CodelineIndex
\setcounter{IndexColumns}2
\parindent=0pt
\parskip=1ex plus .5ex minus .25ex
\begin{document}
\DocInput{productbox.dtx}
\end{document}
%</driver>
%    \end{macrocode}
%
%    \section{The Implementation}
%
%    The implementation contains the code of the style.
%
%    \subsection{The Version Information}
%
%    The following lines define the version information for the class
%    file. 
%
%    \begin{macrocode}
%<*class>
\begingroup
  \def\ProductBox@VC$#1: #2 #3${#2}
  \def\ProductBox@VCdate$#1: #2-#3-#4 #5${#2/#3/#4}
  \xdef\fileversion{1.0}
  \xdef\filerevision{\ProductBox@VC$Revision$}
  \xdef\filedate{\ProductBox@VCdate$Date$}
  \xdef\filename{productbox.dtx}
\endgroup
\let\docversion=\fileversion
\let\docdate=\filedate
%</class>
%    \end{macrocode}
%
%
%    \subsection{Getting Started}
%
%    First we have to determine that the right kind of \LaTeX{} is
%    running and identify style file.
%    
%    \begin{macrocode}
%<*class>
\NeedsTeXFormat{LaTeX2e}
\ProvidesPackage{productbox}[\filedate Another type of boxes...]
%    \end{macrocode}
%
%   A bunch of packages is loaded to form the base of the work herein. 
%    \begin{macrocode}
\RequirePackage{keyval}
\RequirePackage{pgf}
\RequirePackage{tikz}
\usetikzlibrary{calc}
\usetikzlibrary{fadings}
%    \end{macrocode}
%
%    \subsection{Parameters}
%
%    The parameters are stored to be used later. Here the built-in defaults
%    are defined.
%
%    The boolean \texttt{ProductBox@shadow} determines whether or not the
%    shadow in the 3d rendering is shown.
%    \begin{macrocode}
\newif\ifProductBox@shadow\ProductBox@shadowtrue
%    \end{macrocode}
%
%    The boolean \texttt{ProductBox@mirror} determines whether or not the
%    mirror effect in the 3d rendering is shown.
%    \begin{macrocode}
\newif\ifProductBox@mirror
%    \end{macrocode}
%
%    The boolean \texttt{ProductBox@flare} determines whether or not the
%    flare effect in the 3d rendering is shown.
%    \begin{macrocode}
\newif\ifProductBox@flare
%    \end{macrocode}
%
%    The boolean \texttt{ProductBox@clip} determines whether or not the
%    additional clipping of the boxes should be enabled.
%    \begin{macrocode}
\newif\ifProductBox@clip\ProductBox@cliptrue
%    \end{macrocode}
%
%   The edge color in the 3D rendering.
%    \begin{macrocode}
\newcommand\ProductBox@edgeColor{white}
%    \end{macrocode}
%
%   The background color for the mirror effect.
%    \begin{macrocode}
\newcommand\ProductBox@bgColor{white}
%    \end{macrocode}
%
%   The default style is the 3D rendering.
%    \begin{macrocode}
\newcommand\ProductBox@style{threeD}
%    \end{macrocode}
%
%   A hook to add some code at the beginning.
%    \begin{macrocode}
\newcommand\ProductBox@startOptions{}
%    \end{macrocode}
%
%   A hook to add some code at the end.
%    \begin{macrocode}
\newcommand\ProductBox@endOptions{}
%    \end{macrocode}
%
%   The color of the additional lines in the fol rendering.
%    \begin{macrocode}
\newcommand\ProductBox@foldLine{white!80!black}
%    \end{macrocode}
%
%   The diameter of the flare.
%    \begin{macrocode}
\newcommand\ProductBox@flareDiameter{24mm}
%    \end{macrocode}
%
%   The width of the box.
%    \begin{macrocode}
\newcommand\ProductBox@x{88mm}
%    \end{macrocode}
%
%   The height of the box.
%    \begin{macrocode}
\newcommand\ProductBox@y{100mm}
%    \end{macrocode}
%
%   The depth of the box.
%    \begin{macrocode}
\newcommand\ProductBox@z{30mm}
%    \end{macrocode}
%
%    Define the parameters for the keyval package.
%    \begin{macrocode}
\define@key{ProductBox}{flat}[true]{\def\ProductBox@style{flat}}
\define@key{ProductBox}{fold}[true]{\def\ProductBox@style{fold}}
\define@key{ProductBox}{3d}[true]{\def\ProductBox@style{threeD}}
\define@key{ProductBox}{3D}[true]{\def\ProductBox@style{threeD}}
\define@key{ProductBox}{threeD}[true]{\def\ProductBox@style{threeD}}
\define@key{ProductBox}{top}[true]{\def\ProductBox@style{top}}
\define@key{ProductBox}{bottom}[true]{\def\ProductBox@style{bottom}}
\define@key{ProductBox}{front}[true]{\def\ProductBox@style{front}}
\define@key{ProductBox}{back}[true]{\def\ProductBox@style{back}}
\define@key{ProductBox}{left}[true]{\def\ProductBox@style{left}}
\define@key{ProductBox}{right}[true]{\def\ProductBox@style{right}}
\define@key{ProductBox}{style}{\def\ProductBox@style{#1}}
\define@key{ProductBox}{shadow}[true]{%
  \csname ProductBox@shadow#1\endcsname}
\define@key{ProductBox}{mirror}[true]{%
  \csname ProductBox@mirror#1\endcsname}
\define@key{ProductBox}{flare}[true]{%
  \csname ProductBox@flare#1\endcsname}
\define@key{ProductBox}{edgeColor}{\def\productBox@edgeColor{#1}}
\define@key{ProductBox}{backgroundColor}{\def\ProductBox@bgColor{#1}}
\define@key{ProductBox}{scale}{\def\ProductBox@scale{#1}}
\define@key{ProductBox}{width}{\def\ProductBox@x{#1}}
\define@key{ProductBox}{height}{\def\ProductBox@y{#1}}
\define@key{ProductBox}{depth}{\def\ProductBox@z{#1}}
\define@key{ProductBox}{flareDiameter}{%
  \def\ProductBox@flareDiameter{#1}}
\define@key{ProductBox}{view}{\@nameuse{ProductBox@View@#1}}
%    \end{macrocode}
%
%    \subsection{Internal Storage}
%
%    The box |\ProductBox@Front| contains the front material.
%    \begin{macrocode}
\newbox\ProductBox@Front
%    \end{macrocode}
%
%    The box |\ProductBox@Left| contains the left material.
%    \begin{macrocode}
\newbox\ProductBox@Left
%    \end{macrocode}
%
%    The box |\ProductBox@Top| contains the top material.
%    \begin{macrocode}
\newbox\ProductBox@Top
%    \end{macrocode}
%
%    The box |\ProductBox@Right| contains the right material.
%    \begin{macrocode}
\newbox\ProductBox@Right
%    \end{macrocode}
%
%    The box |\ProductBox@Bottom| contains the bottom material.
%    \begin{macrocode}
\newbox\ProductBox@Bottom
%    \end{macrocode}
%
%    The box |\ProductBox@Back| contains the back material.
%    \begin{macrocode}
\newbox\ProductBox@Back
%    \end{macrocode}
%
%    \subsection{Settings}
%
%    \begin{macro}{\ProductBoxSet}
%    The macro |\ProductBoxSet| defines the global parameters used by
%    the environment |ProductBox|. They can be overwritten either
%    within a group or in the optional parameter of the environment.
%    \begin{macrocode}
\newcommand\ProductBoxSet[1]{\setkeys{ProductBox}{#1}}
%    \end{macrocode}
%    \end{macro}
%
%    \subsection{The Main Environment}
%
%    \begin{environment}{ProductBox}
%    This is the central environment provided by this style. In the begin code
%    only the local environments are initialized and the optional parameters
%    are evaluated with the help of the package keyval.
%    \begin{macrocode}
\newenvironment{ProductBox}[1][]{%
  \setkeys{ProductBox}{#1}%
  %
  \def\Left{\ProductBox@Start\ProductBox@Left
    (\ProductBox@z,\ProductBox@y)}%
  \let\endLeft\ProductBox@end
  \def\Front{\ProductBox@Start\ProductBox@Front
    (\ProductBox@x,\ProductBox@y)}%
  \let\endFront\ProductBox@end
  \def\Top{\ProductBox@Start\ProductBox@Top
    (\ProductBox@x,\ProductBox@z)}%
  \let\endTop\ProductBox@end
  \def\Bottom{\ProductBox@Start\ProductBox@Bottom(
    \ProductBox@x,\ProductBox@z)}%
  \let\endBottom\ProductBox@end
  \def\Back{\ProductBox@Start\ProductBox@Back
    (\ProductBox@x,\ProductBox@y)}%
  \let\endBack\ProductBox@end
  \def\Right{\ProductBox@Start\ProductBox@Right
    (\ProductBox@z,\ProductBox@y)}%
  \let\endRight\ProductBox@end
  \ignorespacesafterend
}{%---
%    \end{macrocode}
%    The main activity is performed in the end code. Since the flexibility of
%    the environment is one of its design goals, the expansion of the macro
%    |\ProductBo@style| is used to invoke the macro storen in it.
%    \begin{macrocode}
  \@ifundefined{ProductBox@style@\ProductBox@style}{%
    \errmessage{Style '\ProductBox@style' for ProductBox is unknown}%
  }{%
    \@ifundefined{ProductBox@scale}{%
      \@nameuse{ProductBox@style@\ProductBox@style}}{%
      \scalebox{\ProductBox@scale}%
               {\@nameuse{ProductBox@style@\ProductBox@style}}
    }}%
  \ignorespacesafterend
}
%    \end{macrocode}
%    \end{environment}
%
%    \begin{macro}{\ProductBox@Start}
%    The macro |\ProductBox@Start| starts the environment storing a face.
%    \begin{macrocode}
\def\ProductBox@Start#1(#2){%
  \global\setbox#1\hbox\bgroup\begin{tikzpicture}%
    \ifProductBox@clip \clip (0,0) rectangle (#2); \fi
    \ignorespaces
  }%
%    \end{macrocode}
%    \end{macro}
%
%    \begin{macro}{\ProductBox@end}
%    The macro |\ProductBox@end| ends the environment storing a face.
%    \begin{macrocode}
\def\ProductBox@end{\end{tikzpicture}\egroup \ignorespacesafterend}
%    \end{macrocode}
%    \end{macro}
%
%    \subsection{Box Styles}
%
%    \begin{macro}{\ProductBox@style@flat}
%    This macro defines the sytle \emph{flat} for a product box.
%    \begin{macrocode}
\newcommand\ProductBox@style@flat{%
  \par\noindent
  \hspace*{\ProductBox@z}\kern1mm \copy\ProductBox@Top\\[1ex]
  \copy\ProductBox@Left\kern1mm \copy\ProductBox@Front\kern1mm
  \copy\ProductBox@Right\kern1mm \copy\ProductBox@Back\\[1ex]
  \hspace*{\ProductBox@z}\kern1mm \copy\ProductBox@Bottom\par
}
%    \end{macrocode}
%    \end{macro}
%
%    \subsubsection{Box Style \texttt{front}}
%
%    \begin{macro}{\ProductBox@style@front}
%    This macro defines the sytle \emph{front} for a product box. Only this
%    one face of the box is shown.
%    \begin{macrocode}
\newcommand\ProductBox@style@front{%
  \copy\ProductBox@Front
}
%    \end{macrocode}
%    \end{macro}
%
%    \subsubsection{Box Style \texttt{back}}
%
%    \begin{macro}{\ProductBox@style@back}
%    This macro defines the sytle \emph{back} for a product box. Only this
%    one face of the box is shown.
%    \begin{macrocode}
\newcommand\ProductBox@style@back{%
  \copy\ProductBox@Back
}
%    \end{macrocode}
%    \end{macro}
%
%    \subsubsection{Box Style \texttt{left}}
%
%    \begin{macro}{\ProductBox@style@left}
%    This macro defines the sytle \emph{left} for a product box. Only this
%    one face of the box is shown.
%    \begin{macrocode}
\newcommand\ProductBox@style@left{%
  \copy\ProductBox@Left
}
%    \end{macrocode}
%    \end{macro}
%
%    \subsubsection{Box Style \texttt{right}}
%
%    \begin{macro}{\ProductBox@style@right}
%    This macro defines the sytle \emph{right} for a product box. Only this
%    one face of the box is shown.
%    \begin{macrocode}
\newcommand\ProductBox@style@right{%
  \copy\ProductBox@Right
}
%    \end{macrocode}
%    \end{macro}
%
%    \subsubsection{Box Style \texttt{top}}
%
%    \begin{macro}{\ProductBox@style@top}
%    This macro defines the sytle \emph{top} for a product box. Only this
%    one face of the box is shown.
%    \begin{macrocode}
\newcommand\ProductBox@style@top{%
  \copy\ProductBox@Top
}
%    \end{macrocode}
%    \end{macro}
%
%    \subsubsection{Box Style \texttt{bottom}}
%
%    \begin{macro}{\ProductBox@style@bottom}
%    This macro defines the sytle \emph{bottom} for a product box. Only this
%    one face of the box is shown.
%    \begin{macrocode}
\newcommand\ProductBox@style@bottom{%
  \copy\ProductBox@Bottom
}
%    \end{macrocode}
%    \end{macro}
%
%    \subsubsection{Box Style \texttt{fold}}
%
%    \begin{macro}{\ProductBox@style@fold}
%    This macro defines the sytle \emph{fold} for a product box.
%    \begin{macrocode}
\newcommand\ProductBox@style@fold{%
  \begin{tikzpicture}
%    \end{macrocode}
%    front
%    \begin{macrocode}
    \begin{scope}[xshift=\ProductBox@z]
      \clip (0mm,0mm) rectangle (\ProductBox@x,\ProductBox@y);
      \draw
        node[anchor=south west,
             inner sep=0pt]{\copy\ProductBox@Front};
      \draw[\ProductBox@foldLine]
        (0mm,0mm) rectangle (\ProductBox@x,\ProductBox@y);
    \end{scope}
%    \end{macrocode}
%    back
%    \begin{macrocode}
    \begin{scope}[xshift=\ProductBox@x,
                  xshift=\ProductBox@z,
                  xshift=\ProductBox@z]
      \clip (0mm,0mm) rectangle (\ProductBox@x,\ProductBox@y);
      \draw
        node[anchor=south west,
             inner sep=0pt]{\copy\ProductBox@Back};
      \draw[\ProductBox@foldLine]
        (0mm,0mm) rectangle (\ProductBox@x,\ProductBox@y);
    \end{scope}
%    \end{macrocode}
%    top
%    \begin{macrocode}
    \begin{scope}[xshift=\ProductBox@z,yshift=\ProductBox@y]
      \clip (0mm,0mm) rectangle (\ProductBox@x,\ProductBox@z);
      \draw
        node[anchor=south west,
             inner sep=0pt]{\copy\ProductBox@Top};
      \draw[\ProductBox@foldLine]
        (0mm,0mm) rectangle (\ProductBox@x,\ProductBox@z);
    \end{scope}
%    \end{macrocode}
%    left
%    \begin{macrocode}
    \begin{scope}
      \clip (0mm,0mm) rectangle (\ProductBox@z,\ProductBox@y);
      \draw 
        node[anchor=south west,
             inner sep=0pt]{\copy\ProductBox@Left};
      \draw[\ProductBox@foldLine]
        (0mm,0mm) rectangle (\ProductBox@z,\ProductBox@y);
    \end{scope}
%    \end{macrocode}
%    right
%    \begin{macrocode}
    \begin{scope}[xshift=\ProductBox@z,xshift=\ProductBox@x]
      \clip (0mm,0mm) rectangle (\ProductBox@z,\ProductBox@y);
      \draw 
        node[anchor=south west,
             inner sep=0pt]{\copy\ProductBox@Right};
      \draw[\ProductBox@foldLine]
        (0mm,0mm) rectangle (\ProductBox@z,\ProductBox@y);
    \end{scope}
%    \end{macrocode}
%    bottom ear 3 front
%    \begin{macrocode}
    \begin{scope}[xshift=\ProductBox@z,yshift=-\ProductBox@z]
      \clip (0mm,\ProductBox@z) --
        ($.5*(\ProductBox@z,\ProductBox@z)$) --
        ($.5*(\ProductBox@z,0mm)$) --
        ($(\ProductBox@x,0mm)-.5*(\ProductBox@z,0mm)$) --
        ($(\ProductBox@x,0mm)+.5*(-\ProductBox@z,\ProductBox@z)$) --
        (\ProductBox@x,\ProductBox@z) -- cycle;
      \draw (\ProductBox@x,0mm) 
        node[anchor=south east,
             inner sep=0pt]{\copy\ProductBox@Bottom};
      \draw [\ProductBox@foldLine]
        (0mm,\ProductBox@z) --
        ($.5*(\ProductBox@z,\ProductBox@z)$) --
        ($.5*(\ProductBox@z,0mm)$) --
        ($(\ProductBox@x,0mm)-.5*(\ProductBox@z,0mm)$) --
        ($(\ProductBox@x,0mm)+.5*(-\ProductBox@z,\ProductBox@z)$) --
        (\ProductBox@x,\ProductBox@z) -- cycle;
      \draw ($.5*(\ProductBox@x,\ProductBox@z)-.25*(0mm,\ProductBox@z)$)
        node[circle,fill=white!90!black]{\small\sf1};
    \end{scope}
%    \end{macrocode}
%    bottom ear 1 back
%    \begin{macrocode}
    \begin{scope}[xshift=\ProductBox@x,
                  xshift=\ProductBox@z,
                  xshift=\ProductBox@z,
                  yshift=-\ProductBox@z]
      \clip (0mm,0mm) --
        ($.5*(\ProductBox@z,0mm)$) --
        ($.5*(\ProductBox@z,\ProductBox@z)$) --
        ($(\ProductBox@x,0mm)-.5*(\ProductBox@z,-\ProductBox@z)$) --
        ($(\ProductBox@x,0mm)-.5*(\ProductBox@z,0mm)$) --
        (\ProductBox@x,0mm) -- (\ProductBox@x,\ProductBox@z) --
        (0mm,\ProductBox@z) -- cycle; 
      \draw (0mm,\ProductBox@z)
         node[rotate=180,anchor=south east,
              inner sep=0pt]{\copy\ProductBox@Bottom};
       \draw[\ProductBox@foldLine]
        (0mm,0mm) --
        ($.5*(\ProductBox@z,0mm)$) --
        ($.5*(\ProductBox@z,\ProductBox@z)$) --
        ($(\ProductBox@x,0mm)-.5*(\ProductBox@z,-\ProductBox@z)$) --
        ($(\ProductBox@x,0mm)-.5*(\ProductBox@z,0mm)$) --
        (\ProductBox@x,0mm) -- (\ProductBox@x,\ProductBox@z) --
        (0mm,\ProductBox@z) -- cycle; 
      \draw ($.25*(\ProductBox@z,\ProductBox@z)$)
        node[circle,fill=white!90!black]{\small\sf1};
      \draw ($(\ProductBox@x,0mm)+.25*(-\ProductBox@z,\ProductBox@z)$)
        node[circle,fill=white!90!black]{\small\sf1};
    \end{scope}
%    \end{macrocode}
%    bottom ear 2 right
%    \begin{macrocode}
    \begin{scope}[xshift=\ProductBox@z,yshift=-\ProductBox@z]
      \clip
        (\ProductBox@x,\ProductBox@z) --
        ($(\ProductBox@x,\ProductBox@z)+(\ProductBox@z,0mm)$) --
        ($(\ProductBox@x,0mm) + .5*(\ProductBox@z,\ProductBox@z)$) -- 
        ($(\ProductBox@x,0mm) + .5*(\ProductBox@z,0mm)$) -- 
        (\ProductBox@x,0mm) -- cycle; 
      \begin{scope}[xshift=1mm,yshift=1mm]
        \draw[xshift=\ProductBox@z,yshift=\ProductBox@z]
          (\ProductBox@x,0mm)
          node[rotate=90,anchor=south east]{\copy\ProductBox@Bottom};
      \end{scope}
      \draw[\ProductBox@foldLine]
        (\ProductBox@x,\ProductBox@z) --
        ($(\ProductBox@x,\ProductBox@z)+(\ProductBox@z,0mm)$) --
        ($(\ProductBox@x,0mm) + .5*(\ProductBox@z,\ProductBox@z)$) -- 
        ($(\ProductBox@x,0mm) + .5*(\ProductBox@z,0mm)$) -- 
        (\ProductBox@x,0mm) -- cycle; 
      \draw[xshift=\ProductBox@x]
        ($.25*(\ProductBox@z,\ProductBox@z)$)
        node[circle,fill=white!90!black]{\small\sf2};
    \end{scope}
%    \end{macrocode}
%    bottom ear 2 left
%    \begin{macrocode}
    \begin{scope}[yshift=-\ProductBox@z]
      \clip
        (0mm,\ProductBox@z) -- (\ProductBox@z,\ProductBox@z) --
        (\ProductBox@z,0mm) -- ($.5*(\ProductBox@z,0mm)$) --
        ($.5*(\ProductBox@z,\ProductBox@z)$) -- cycle; 
      \draw (0mm,\ProductBox@z)
        node[rotate=270,anchor=south west,
             inner sep=0pt]{\copy\ProductBox@Bottom};
      \draw [\ProductBox@foldLine]
        (0mm,\ProductBox@z) -- (\ProductBox@z,\ProductBox@z) --
        (\ProductBox@z,0mm) -- ($.5*(\ProductBox@z,0mm)$) --
        ($.5*(\ProductBox@z,\ProductBox@z)$) -- cycle; 
      \draw ($(\ProductBox@z,0mm)+.25*(-\ProductBox@z,\ProductBox@z)$)
        node[circle,fill=white!90!black]{\small\sf2};
    \end{scope}
%    \end{macrocode}
%    glue ear
%    \begin{macrocode}
    \begin{scope}[xshift=\ProductBox@x,
                  xshift=\ProductBox@x,
                  xshift=\ProductBox@z,
                  xshift=\ProductBox@z,\ProductBox@foldLine]
      \clip
        (0mm,0mm) -- (6mm,3mm) --
        ($(6mm,\ProductBox@y)-(0mm,3mm)$) --
        (0mm,\ProductBox@y) -- cycle;
      \draw node[anchor=south west,
                 inner sep=0pt]{\copy\ProductBox@Left};
      \draw[\ProductBox@foldLine]
        (0mm,0mm) -- (6mm,3mm) --
        ($(6mm,\ProductBox@y)-(0mm,3mm)$) --
        (0mm,\ProductBox@y) -- cycle;
    \end{scope}
%    \end{macrocode}
%    top ear left
%    \begin{macrocode}
    \begin{scope}[yshift=\ProductBox@y]
      \clip
        (0mm,0mm) -- (6mm,12mm) --
        ($(\ProductBox@z,12mm)-(6mm,0mm)$)
        -- (\ProductBox@z,0mm) -- cycle;
      \draw
        node[rotate=90,anchor=north west,
              inner sep=0pt]{\copy\ProductBox@Top};
      \draw[\ProductBox@foldLine]
        (0mm,0mm) -- (6mm,12mm) --
        ($(\ProductBox@z,12mm)-(6mm,0mm)$)
        -- (\ProductBox@z,0mm) -- cycle;
    \end{scope}
%    \end{macrocode}
%    top ear right
%    \begin{macrocode}
    \begin{scope}[xshift=\ProductBox@x,
                  xshift=\ProductBox@z,
                  yshift=\ProductBox@y]
      \clip
        (0mm,0mm) -- (6mm,12mm) --
        ($(\ProductBox@z,12mm)-(6mm,0mm)$)
        -- (\ProductBox@z,0mm) -- cycle;
      \draw
          node[rotate=270,anchor=south east,
              inner sep=0pt]{\copy\ProductBox@Top};
      \draw[\ProductBox@foldLine]
        (0mm,0mm) -- (6mm,12mm) -- ($(\ProductBox@z,12mm)-(6mm,0mm)$)
        -- (\ProductBox@z,0mm) -- cycle;
    \end{scope}
%    \end{macrocode}
%    top ear back
%    \begin{macrocode}
    \begin{scope}[xshift=\ProductBox@z,
                  yshift=\ProductBox@y,
                  yshift=\ProductBox@z]
      \clip
        (0mm,0mm) -- (6mm,12mm) --
        ($(\ProductBox@x,12mm)-(6mm,0mm)$) --
        (\ProductBox@x,0mm) -- cycle;
      \draw (0mm,\ProductBox@y)
        node[rotate=180,anchor=south east,
             inner sep=0pt]{\copy\ProductBox@Back};
      \draw[\ProductBox@foldLine]
        (0mm,0mm) -- (6mm,12mm) --
        ($(\ProductBox@x,12mm)-(6mm,0mm)$) --
        (\ProductBox@x,0mm) -- cycle;
    \end{scope}
  \end{tikzpicture}}
%    \end{macrocode}
%    \end{macro}
%
%    \subsubsection{Box Style \texttt{threeD}}
%
%    \begin{macro}{\ProductBox@style@threeD}
%    This macro defines the sytle \emph{threeD} for a product box. The
%    faces are placed such the illusion of a three-dimensional box
%    appears. In addition a shadow is shown if not disabled. 
%    \begin{macrocode}
\newcommand\ProductBox@style@threeD{
  \begin{tikzpicture}\ProductBox@startOptions
    \ifProductBox@mirror \ProductBox@threeD@mirror \fi
    \ifProductBox@shadow \ProductBox@threeD@shadow \fi

    \begin{scope}
      \fill[white,
            yslant=\ProductBox@p@front@yslant,
            xscale=\ProductBox@p@front@xscale,
            anchor=south west]
        rectangle (\ProductBox@x,\ProductBox@y);
      \draw
        node[yslant=\ProductBox@p@front@yslant,
             xscale=\ProductBox@p@front@xscale,
             anchor=south west,
             inner sep=0pt] {\copy\ProductBox@Front};
      \fill[black,
            fill opacity=.025,
            yslant=\ProductBox@p@front@yslant,
            xscale=\ProductBox@p@front@xscale,
            anchor=south west]
        rectangle (\ProductBox@x,\ProductBox@y);
        \ifProductBox@flare \ProductBox@threeD@flare \fi
    \end{scope}
    \begin{scope}[shift={($\ProductBox@p@left@yslant
                             *\ProductBox@p@left@xscale
                             *(0mm,\ProductBox@z)
                          -\ProductBox@p@left@xscale
                             *(\ProductBox@z,0mm)$)}]
      \fill[white,
            yslant=-\ProductBox@p@left@yslant,
            xscale=\ProductBox@p@left@xscale,
            anchor=south west]
        rectangle (\ProductBox@z,\ProductBox@y);
      \draw
        node[yslant=-\ProductBox@p@left@yslant,
             xscale=\ProductBox@p@left@xscale,
             anchor=south west,
             inner sep=0pt] {\copy\ProductBox@Left};
      \fill[black,
            fill opacity=.25,
            yslant=-\ProductBox@p@left@yslant,
            xscale=\ProductBox@p@left@xscale,
            anchor=south west] rectangle (\ProductBox@z,\ProductBox@y);
    \end{scope}
    \begin{scope}[yshift=\ProductBox@y]
      \fill [white,
             xslant=-\ProductBox@p@top@xslant,
             yslant=\ProductBox@p@top@yslant,
             xscale=\ProductBox@p@top@xscale,
             yscale=\ProductBox@p@top@yscale,
             anchor=south west] rectangle (\ProductBox@x,\ProductBox@z);
      \draw
        node[xslant=-\ProductBox@p@top@xslant,
             yslant=\ProductBox@p@top@yslant,
             xscale=\ProductBox@p@top@xscale,
             yscale=\ProductBox@p@top@yscale,
             anchor=south west,
             inner sep=0pt] {\copy\ProductBox@Top};
      \fill [black,
             fill opacity=.0025,
             xslant=-\ProductBox@p@top@xslant,
             yslant=\ProductBox@p@top@yslant,
             xscale=\ProductBox@p@top@xscale,
             yscale=\ProductBox@p@top@yscale,
             anchor=south west] rectangle (\ProductBox@x,\ProductBox@z);
    \end{scope}
    \begin{scope}[shift={($\ProductBox@p@left@yslant
                             *\ProductBox@p@left@xscale
                             *(0mm,\ProductBox@z)
                           -\ProductBox@p@left@xscale
                             *(\ProductBox@z,0mm)$)}]
      \draw [\ProductBox@edgeColor,
             line width=.5mm,
             draw opacity=.25,
             yslant=-\ProductBox@p@left@yslant,
             xscale=\ProductBox@p@left@xscale]
        (\ProductBox@z,0mm) --
        (\ProductBox@z,\ProductBox@y) --
        (0mm,\ProductBox@y); 
    \end{scope}
    \draw [\ProductBox@edgeColor,
           line width=.5mm,
           draw opacity=.25,
           yslant=\ProductBox@p@front@yslant,
           xscale=\ProductBox@p@front@xscale,
           anchor=south west]
      (0mm,\ProductBox@y) --
      (\ProductBox@x,\ProductBox@y); 
      \ProductBox@endOptions
  \end{tikzpicture}}
%    \end{macrocode}
%    \end{macro}
%
%    Parameters
%
%    \begin{macrocode}
\newcommand\ProductBox@Set@params[9]{%
  \def\ProductBox@p@front@yslant{#1}
  \def\ProductBox@p@front@xscale{#2}
  \def\ProductBox@p@left@yslant{#3}
  \def\ProductBox@p@left@xscale{#4}
  \def\ProductBox@p@top@xslant{#5}
  \def\ProductBox@p@top@yslant{#6}
  \def\ProductBox@p@top@xscale{#7}
  \def\ProductBox@p@top@yscale{#8}
  \def\ProductBox@p@mirror@yscale{#9}
}
\@namedef{ProductBox@View@1}{
  \ProductBox@Set@params
    {.25}{.9090909}%
    {.5}{.6666666}%
    {2}{.1666666}{1.36363636}{.3333333}%
    {.2}}
\@namedef{ProductBox@View@2}{
  \ProductBox@Set@params
    {.15}{.9090909}%
    {.6666666}{.5}%
    {1.5}{.12}{1.11}{.333333}%
    {.2}}
\@namedef{ProductBox@View@3}{
  \ProductBox@Set@params
    {.05}{.95}%
    {.25}{.4}%
    {4}{.04}{1.133333}{.1}%
    {.2}}
\@nameuse{ProductBox@View@1}
%    \end{macrocode}
%
%    \begin{macro}{\ProductBox@threeD@shadow}
%    This macro defines the code to produce the shadow effect for the 3D rendering.
%    \begin{macrocode}
\newcommand\ProductBox@threeD@shadow{
    \begin{scope}
      \pgfsetcornersarced{\pgfpoint{2mm}{2mm}}
      \foreach \x in {.5,1,1.5,2,2.5,3,4,5,6,8,10} {
        \filldraw [black,
                   line width=\x mm,
                   opacity=.01,
                   shift={(-1mm,1mm)}]
          (0mm,0mm) --
          (-\ProductBox@p@top@xslant
           *\ProductBox@p@left@xscale
           *\ProductBox@z,
           \ProductBox@p@top@yscale
           *\ProductBox@z) -- 
          (0mm,
           \ProductBox@p@top@yscale*\ProductBox@z +
           \ProductBox@p@top@xslant
           *\ProductBox@p@left@xscale
           *\ProductBox@p@top@yslant
           *\ProductBox@z) -- 
          cycle;
      }
    \end{scope}
}
%    \end{macrocode}
%    \end{macro}
%
%    The following fading is used for the mirror effect in the 3D rendering.
%    \begin{macrocode}
\tikzfading[name=ProductBoxFade,
            top color=transparent!0,
            bottom color=transparent!60]
%    \end{macrocode}
%
%    The following fading is used for the flare effect in the 3D rendering.
%    \begin{macrocode}
\tikzfading[name=ProductBoxFlare,
            inner color=transparent!60,
            outer color=transparent!100]
%    \end{macrocode}
%
%    \begin{macro}{\ProductBox@threeD@mirror}
%    This macro defines the code to produce the mirror effect.
%    \begin{macrocode}
\newcommand\ProductBox@threeD@mirror{%
  \begin{pgflowlevelscope}{
      \pgftransformyslant{\ProductBox@p@front@yslant}
      \pgftransformyscale{-\ProductBox@p@mirror@yscale}
      \pgftransformxscale{\ProductBox@p@front@xscale}}
    \begin{scope}
      \clip rectangle (\ProductBox@x,50mm);
      \fill
        node[anchor=south west,
              inner sep=0pt]{\copy\ProductBox@Front};
    \end{scope}
    \fill [white,path fading=ProductBoxFade]
      rectangle (\ProductBox@x,50.5mm);
  \end{pgflowlevelscope}

  \begin{pgflowlevelscope}{
      \pgftransformxshift{-\ProductBox@p@left@xscale*\ProductBox@z}
      \pgftransformyshift{\ProductBox@p@top@yscale*\ProductBox@z}
      \pgftransformyslant{-\ProductBox@p@left@yslant}
      \pgftransformxscale{\ProductBox@p@left@xscale}
      \pgftransformyscale{-\ProductBox@p@mirror@yscale}}
    \begin{scope}
      \clip rectangle (\ProductBox@z,50mm);
      \fill
        node[anchor=south west,
             inner sep=0pt]{\copy\ProductBox@Left};
      \fill[opacity=.25] rectangle (\ProductBox@z,\ProductBox@y);
    \end{scope}
    \fill [white,path fading=ProductBoxFade]
      rectangle (\ProductBox@z,50.5mm);
  \end{pgflowlevelscope}
}
%    \end{macrocode}
%    \end{macro}
%
%    \begin{macro}{\ProductBox@threeD@flare}
%    This macro defines the code to produce the flare effect.
%    \begin{macrocode}
\newcommand\ProductBox@threeD@flare{
  \fill [white,path fading=ProductBoxFlare]
    (.8*\ProductBox@x,.9*\ProductBox@y)
    circle(\ProductBox@flareDiameter); 
}
%    \end{macrocode}
%    \end{macro}
%
%    \begin{macrocode}
\@namedef{ProductBox@style@3D}{\ProductBox@style@threeD}
%</class>
%    \end{macrocode}
%    That's all.
%
%    \newpage
%    \PrintChanges
%    \newpage
%    \PrintIndex
%    \Finale
\endinput
%
% Local Variables: 
% End: 
