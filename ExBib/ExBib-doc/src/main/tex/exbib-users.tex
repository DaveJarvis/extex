\nonstopmode
%%*****************************************************************************
%% $Id: extex-users.tex 5326 2007-02-27 13:09:26Z gene $
%%*****************************************************************************
%% Author: Gerd Neugebauer
%%-----------------------------------------------------------------------------
\documentclass{extex-doc}

\title{\ExBib\ User's Guide}
\author{Gerd Neugebauer}

\usepackage{makeidx}

\newcommand\Arg[1]{\(\langle\){\tt\itshape #1}\(\rangle\)}
\newcommand\CLI[1]{\texttt{-#1}\index{#1@\texttt{-#1}}}
\newcommand\Property[1]{\texttt{#1}\index{#1@\texttt{#1}}}
\newcommand\File[1]{\texttt{#1}\index{#1@\textsf{#1}}}
\newcommand\Prog[1]{\texttt{#1}\index{#1}}
\newcommand\Mode[1]{\texttt{#1}\index{#1}}
\newcommand\macro[1]{\texttt{\char`\\ #1}\index{#1@\texttt{\char`\\ #1}}}
\newcommand\Macro[1]{\texttt{\char`\\ #1}}
\newcommand\tag[1]{%
    \(\langle\)\textit{#1}\(\rangle\)%
    \index{#1@\protect\Tag{#1}}}
\newcommand\Tag[1]{\texorpdfstring{%
    \(\langle\)\textit{#1}\(\rangle\)}{<#1>}}

\newenvironment{syntax}{%
  \begin{tabbing}\kern2em\=\kern2em\=\kill
  }{%
  \end{tabbing}}
\newcommand\SyntaxDef{\>\(\rightarrow\)\>}
\newcommand\SyntaxOr{\>\(|\)\>}

\def\setVersion$#1: #2 ${\gdef\Version{0.1 (Revision #2)}}
\setVersion$Revision: 5326 $

\def\n{\char`\\n}
\def\t{\char`\\t}

\providecommand\BibTeX{\textsc{Bib}\TeX}

\makeindex

\begin{document}%%%%%%%%%%%%%%%%%%%%%%%%%%%%%%%%%%%%%%%%%%%%%%%%%%%%%%%%%%%%%%%

\begin{titlepage}
  \parindent=0pt
  \begin{center}
  \vspace*{1pt}
  \vfill
  \ExBibbox
  \vfill
  \textsf{\bfseries\Huge User's Guide}
  \vfill
  \textsf{\Large Version \Version}
  \vfill
  \textsf{\large Gerd Neugebauer}
  \vfill
  \vfill
%\maketitle

  \begin{abstract}\parindent=0pt
    This document describes \ExBib. It explains how to get \ExBib\ up
    and running and which features \ExBib\ offers to you.

    The intended audience for this document are end users of a
    bibliography processor who want to use \ExBib\ on the command line or
    as plug-in replacement of \BibTeX.
  \end{abstract}
  \unitlength=1mm
  \begin{picture}(0,0)
    \put(56,120){\makebox(0,0){\scalebox{6}{\rotatebox{45}{\color{red}\textsf{\Huge\bfseries Draft}}}}}
  \end{picture}
  \end{center}
\newpage
\footnotesize
\copyright\ 2008 The \ExTeX\ Group and individual authors listed below 
\medskip

Permission is granted to copy, distribute and/or modify this document
under the terms of the GNU Free Documentation License, Version 1.2 or
any later version published by the Free Software Foundation. A copy of
the license is included in the section entitled ``GNU Free
Documentation License''.
\bigskip

This product includes software developed by the Apache Software
Foundation (http://www.apache.org/).

\vfill

Gerd Neugebauer\\
Im Lerchelsb\"ohl 5\\
64521 Gro\ss-Gerau (Germany)
\smallskip

\href{mailto://gene@gerd-neugebauer.de}{gene@gerd-neugebauer.de}

\end{titlepage}

\tableofcontents

%------------------------------------------------------------------------------

%%*****************************************************************************
%% $Id: bst-language.tex,v 0.00 2008/04/30 23:26:21 gene Exp $
%%*****************************************************************************
%% Author: Gerd Neugebauer
%%-----------------------------------------------------------------------------

\chapter{Introduction}
%@author Gerd Neugebauer

\ExTeX\index{ExTeX@\ExTeX} aims at providing a high-quality
typesetting system. The development of \ExTeX\ has been inspired by
the experiences with \TeX\ \cite{knuth:texbook}. The focus lies on an
open design and a high degree of configurability.

A tight integration of several components is one of the possibilities
opened by \ExTeX. To work into this direction \ExBib\ has been
implemented. It is a plug-in replacement for
\BibTeX~0.99c\index{BibTeX 0.99c@\BibTeX~0.99c}\cite{btxdoc,btxhak}
or \BibTeX~8\index{BibTeX 8@\BibTeX~8}.

\begin{figure}[hb]
  \centering
  %%*****************************************************************************
%% $Id$
%%*****************************************************************************
%% Author: Gerd Neugebauer
%%-----------------------------------------------------------------------------
\begingroup\small
\def\processor(#1)#2{%
  \begin{scope}[shift={(#1)}]
    \draw[thick,color=white!80!gray,fill==white!70!gray]
    (1,-0.5) rectangle (9,4.5);
    \shade[top color=white!90!green,bottom color=white!60!green,draw=green!40!black,thick]
    (0.5,0) rectangle (8.5,5);
    \draw (4.5,2.5) node{#2};
    \shade[top color=white!90!green,bottom color=white!60!green,draw=green!40!black,thick]
    (0,1) rectangle (1,2);
    \shade[top color=white!90!green,bottom color=white!60!green,draw=green!40!black,thick]
    (0,3) rectangle (1,4);
  \end{scope}
}
\def\data(#1)#2{%
  \begin{scope}[shift={(#1)}]
    \draw[thick,color=white!80!gray,fill==white!70!gray]
    (0.3,-.3) -- (5.3,-.3) -- (5.3,1.7) -- (4.3,2.7) -- (0.3,2.7) -- cycle;
    \shade[top color=white!90!yellow,bottom color=white!60!yellow,draw=red!40!black,thick]
    (0,0) -- (5,0) -- (5,2) -- (4,3) -- (0,3) -- cycle;
    \draw (2.5,1.5) node{#2};
  \end{scope}
}
\def\datas(#1)#2{%
  \begin{scope}[shift={(#1)}]
    \draw[thick,color=white!80!gray,fill==white!70!gray]
    (0.3,-.3) -- (5.3,-.3) -- (5.3,1.7) -- (4.3,2.7) -- (0.3,2.7) -- cycle;
    \begin{scope}[shift={(.15,.15)}]
      \draw[thick,color=white!80!gray,fill==white!70!gray]
      (0.3,-.3) -- (5.3,-.3) -- (5.3,1.7) -- (4.3,2.7) -- (0.3,2.7) -- cycle;
    \end{scope}
    \begin{scope}[shift={(.1,.1)}]
      \draw[thick,color=white!80!gray,fill==white!70!gray]
      (0.3,-.3) -- (5.3,-.3) -- (5.3,1.7) -- (4.3,2.7) -- (0.3,2.7) -- cycle;
    \end{scope}
    \begin{scope}[shift={(.05,.05)}]
      \draw[thick,color=white!80!gray,fill==white!70!gray]
      (0.3,-.3) -- (5.3,-.3) -- (5.3,1.7) -- (4.3,2.7) -- (0.3,2.7) -- cycle;
    \end{scope}
    \begin{scope}[shift={(.3,.3)}]
      \shade[top color=white!90!yellow,bottom color=white!60!yellow,draw=red!40!black,thick]
      (0,0) -- (5,0) -- (5,2) -- (4,3) -- (0,3) -- cycle;
    \end{scope}
    \begin{scope}[shift={(.2,.2)}]
      \shade[top color=white!90!yellow,bottom color=white!60!yellow,draw=red!40!black,thick]
      (0,0) -- (5,0) -- (5,2) -- (4,3) -- (0,3) -- cycle;
    \end{scope}
    \begin{scope}[shift={(.1,.1)}]
      \shade[top color=white!90!yellow,bottom color=white!60!yellow,draw=red!40!black,thick]
      (0,0) -- (5,0) -- (5,2) -- (4,3) -- (0,3) -- cycle;
    \end{scope}
    \shade[top color=white!90!yellow,bottom color=white!60!yellow,draw=red!40!black,thick]
    (0,0) -- (5,0) -- (5,2) -- (4,3) -- (0,3) -- cycle;
    \draw (2.5,1.5) node{#2};
  \end{scope}
}
\def\arrow(#1)#2#3{%
  \begin{scope}[shift={(#1)}#3,scale=.5]
    \begin{scope}[shift={(#2)}]
      \draw[thick,color=white!80!gray,fill==white!70!gray]
      (0,1) -- (5,1) -- (5,0) -- (6,2) -- (5,4) -- (5,3) -- (0,3) -- cycle;
    \end{scope}
    \shade[top color=white!90!gray,bottom color=white!60!gray,draw=gray!40!black,thick]
    (0,1) -- (5,1) -- (5,0) -- (6,2) -- (5,4) -- (5,3) -- (0,3) -- cycle;
  \end{scope}
}
%
\begin{tikzpicture}[scale=.35]\sf
  \processor(10,30){Text Processor}

  \arrow(10.5,29){.3,.3}{,rotate=270}
  \datas(9,22){file.aux}
  \arrow(10.5,21){.3,.3}{,rotate=270}

  \processor(10,12){\ExBib}
  \arrow(19.5,13.5){.3,-.3}{}
  \data(23.5,13){file.blg}

  \arrow(18.5,18){-.3,-.3}{,rotate=90}
  \data(15,22){file.bbl}
  \arrow(18.5,26){-.3,-.3}{,rotate=90}

  \arrow(6,13.5){.3,-.3}{}
  \datas(0,13){*.bib}

  \arrow(12.5,8){-.3,-.3}{,rotate=90}
  \datas(9,4){*.bst}

  \arrow(18.5,8){-.3,-.3}{,rotate=90}
  \data(15,4){*.csf}

  \arrow(23,10){-.3,-.3}{,rotate=135}
  \datas(21,5){Config}

\end{tikzpicture}
\endgroup
\endinput
%
% Local Variables: 
% mode: latex
% TeX-master: nil
% End: 

  \caption{\ExBib\ and the Text Processor}%
  \label{fig:files}
\end{figure}
The principal interaction of a bibliography processor and a text
processor\index{text processor} has been defined by
\BibTeX\index{BibTeX@\BibTeX}. This is depicted in
figure~\ref{fig:files}. The underlying communication structure is file
based. This scheme is supported by \ExBib\ as well.

The main input from the text processor\index{text processor} is
transferred in the \texttt{aux} file. In \LaTeX\index{LaTeX@\LaTeX}
(cf. \cite{lamport:latex,goosens.mittelbach:latex.companion}) the
directive \macro{include} can be used to conditionally include parts
of a complete document. To make this work several \texttt{aux} files
are written -- one for each fragment. This \ExBib\ has to cope with
several \texttt{aux} files.

The \texttt{aux} files contain the information on the databases to be
used and the bib style. Accordingly the databases and the style are
read. As a result of the processing a formatted list of database
entries is produced in a \texttt{bbl} file. Additionally logging
information may be sent to a log file. The \texttt{bbl} file can be
read in by the text processor to include the entries into the
document. This completes the cycle.

One cycle may not be enough to resolve all citations. If the database
entries contain references (in form of \verb|\cite|
macros\index{cite@\verb/\cite/}). Then they can be resolved in a
second round. Unfortunately this may theoretically continue ad
infinitum. Practically spoken this has not been observed in real life.
Most of time one cycle or at most two of them are sufficient.


\section{Bibliography Processors -- a Short History}

\BibTeX\ is the well known bibliography processor in the \TeX\ world.
It has been written by Oren Patashnik\index{Patashnik, Oren} in 1983
to 1988. The foundations are older. \BibTeX\ refers in some aspects
back to Scribe\index{Scribe}.

The current release is \BibTeX~0.99c (\cite{btxdoc,btxhak}). The
development seems to be ended.\IM{0}

The long awaited release \BibTeX~1.0 should finalize the development
and include some additional features. Several papers have been published
(\cite{patashnik:bibtex1.0,patashnik:bibtex}) but a working version
has not been seen yet.\IM{1}

Since \BibTeX~0.99c has some deficiencies with respect to sorting and
character sets a rewrite in has been done by Niel
Kempson\index{Kempson, Niel} and Alejandro
Aguilar-Sierra\index{Aguilar-Sierra, Alejandro} around 1996. This is
\BibTeX~8. \BibTeX~8 uses internally 8-bit characters and provides
means to deal with different encodings.\IM{8}


ML\BibTeX\index{MLBibTeX@ML\BibTeX} is an attempt oif Jean-Marie
Hufflen\index{Hufflen, Jean-Marie} to rewrite \BibTeX\ and enhance ot
with features for multi-lingual processing.

\INCOMPLETE


\section{This Document}

This document is meant to be a reference document. It should contain
all information necessary to know. It is not meant to be a tutorial.
Thus do not expect tutorial type material in this document.


\section{Web Site}%
%@author Gerd Neugebauer

\begin{figure}[!ht]
  \centering
  \includegraphics[width=.5\textwidth]{img/www-extex-org}
  \caption{\texttt{www.extex.org}}
  \label{fig:www.exetex.org}
\end{figure}
There is a web site devoted to \ExTeX.\index{WWW}\index{Web Site} This
web site (see figure~\ref{fig:www.exetex.org}) can be reached via the
URL\index{www.extex.org}
\begin{quotation}
  \url{http://www.extex.org}
\end{quotation}


\section{Mailing Lists}
%@author Gerd Neugebauer

If you are ready to try \ExBib{} you might as well want to join a
mailing list to get in contact with the community.\index{mailing list}

\begin{quotation}
  \url{http://www.dante.de/listman/extex}
\end{quotation}


\section{Reporting Bugs}
%@author Gerd Neugebauer


If you find any bugs in \ExBib\ you can submit them 
%either 
via a HTML form.
% or via email. 
You can find the HTML form at
\begin{quotation}
  \url{http://www.extex.org/bugs}
\end{quotation}
%Emails containing the description can be sent to
%\begin{quotation}
%  \href{mailto:extex-bugs@dante.de}{extex-bugs@dante.de}
%\end{quotation}

Please include in your description 
\begin{itemize}
\item the source of a \emph{minimal} example showing the problem
\item the log file resulting from running this example
\item a description why you think that something went wrong and what
  the expected result would be
\item a description of the environment you are using (host
  architecture, operating system, Java version)
\end{itemize}

\endinput
%
% Local Variables: 
% mode: latex
% TeX-master: "exbib-users"
% End: 


%------------------------------------------------------------------------------
\chapter{Getting Started}
%@author Gerd Neugebauer

In this chapter we describe the steps you can take to get \ExTeX\ up
and running. We try to use as few as possible premises. Thus it should
be not too hard to get started.

\section{Prerequisites}
%@author Gerd Neugebauer

\subsection{Java}
%@author Gerd Neugebauer

You need to have Java 5\index{Java} or later installed on your
system. You can get Java for a several systems directly from
\url{java.sun.com}. Download and install it according to the
installation instructions for your environment.

To check that you have an appropriate Java on your path you can use
the command \texttt{java} with the argument \texttt{-version}. This
can be seen in the following listing:

\lstset{morecomment=[l]{\#}}%
\begin{lstlisting}{morecomment=[l][keywordstyle]{>}}
# java -version
java version "1.5.0_04"
Java(TM) 2 Runtime Environment, Standard Edition (build 1.5.0_04-b05)
Java HotSpot(TM) Client VM (build 1.5.0_04-b05, mixed mode)
#
\end{lstlisting}


\subsection{TEXMF}
%@author Gerd Neugebauer

If you want to use more than the pure \ExBib\ engine, fonts and macros
can be inherited from a texmf tree\index{texmf}. \ExBib\ itself does
not contain a full texmf tree. It comes just with some rudimentary
files necessary for testing. Thus you should have installed a texmf
tree, e.g. from a \TeX Live\index{TeXlive@\TeX Live} installation.
This can be found on the \href{http://www.ctan.org}{Comprehensive
  \TeX\ Archive Network (CTAN)}\index{CTAN}.

There is no need to install the texmf tree in a special place. You
have to tell \ExBib\ anyhow where it can be found. It is even possible
to work with several texmf trees.

One requirement for the texmf trees is that they have a file database
(\File{ls-R}). \ExBib\ can be configured to work without it, but then
\ExBib\ is deadly slow. Thus you do not really want to try this
alternative.


\section{Getting \ExBib}
%@author Gerd Neugebauer

\subsection{Getting the Installer}
%@author Gerd Neugebauer

The simplest way to get \ExBib\ up and running is to use the \ExBib\ 
installer. This installer\index{installer} is distributed as one file
\File{ExBib-setup.jar}. You can download it from

\begin{quotation}
  \url{http://www.extex.org/download/}
\end{quotation}

If you have got the installer there is no need for you to get the
sources as well. Thus you can skip the following section.


\subsection{Getting the Sources}
%@author Gerd Neugebauer

The sources of \ExTeX\ are stored in a Subversion repository. To access this
repository you need access to the internet and Subversion installed in some
way.


The coordinates of the repository are:\index{repository}\index{CVS}
\medskip
\begin{quotation}
  \texttt{https://svn.berlios.de/svnroot/repos/extex}
\end{quotation}
\bigskip

We assume here that you have access to Subversion on the command line.
This can be either a shell on a Unix-like system or something like
cygwin on Windows. We also assume that you have direct connection to
the internet or Subversion configured to access the repository on the
internet.

First we create a directory where the sources are stored:
\begin{lstlisting}{}
# mkdir ExBib
\end{lstlisting}

Next we change the current directory to this base directory:
\begin{lstlisting}{}
# cd ExBib
\end{lstlisting}

Finally we can check out the sources:
\begin{lstlisting}{}
# svn checkout https://svn.berlios.de/svnroot/repos/extex/trunk/ExBib
\end{lstlisting}

This command shows a lot of output. At the end the current directory
contains the sub-directory \texttt{trunk} which is filled with a lot
of files and directories.


\section{Installing \ExBib}
%@author Gerd Neugebauer

There are several ways to install \ExBib.
The easiest installation of \ExBib\ works with the \ExBib\ installer.
This installer is named \File{ExBib-setup.jar}. You can start the
installer with the following command line:\index{installer}

\begin{lstlisting}{}
# java -jar ExTeX-setup.jar
\end{lstlisting}

On Windows\index{Windows} with a properly installed Java\index{Java}
you can also start the installer by double-clicking
\texttt{ExBib-setup.jar} in the Explorer\index{Explorer}.

\begin{figure}[t]
  \centering
  \includegraphics[width=.8\textwidth]{img/inst1}
  \caption{The Language Selection in the Installer}

  \label{fig:inst1}
\end{figure}
The installer provides a graphical user interface with a wizard
guiding you through the installation process. The first dialog is
shown in figure~\ref{fig:inst1}. As you can see you can select one of
several languages for the installation process. Currently the
languages English and German are supported. There might be some more
at the time you are performing the
installation.\index{installer!language}\index{language!installer}

Note that the internationalization covers the installer only. \ExBib\
can be run under different language environments as well. This is
controlled by a setting at run-time. Currently only an English
language binding for \ExBib\ is provided.\index{language}

Finally you have to make sure that the executables \Prog{exbib} or
\Prog{exbib.bat} is on your path for executables.\index{path}


\subsection{Replaying an Installation}
%@author Gerd Neugebauer

Sometimes it is desirable to perform an installation on several
similar machines. This means that the answers to the questions in the
installer are the same. This process can be automated.
\begin{figure}[tp]
  \centering
  \includegraphics[width=.8\textwidth]{img/inst4}
  \caption{Generating a Auto-Configuration for the Installer}
  \label{fig:inst8}
\end{figure}

In figure~\ref{fig:inst8} you can see the last screen of the
installer. Here you have the possibility to select the button
``Generate an automatic installation script''. This produces an XML
file which can be passed to the installer to avoid the
dialogs.\index{installer}\index{installation script}

Suppose you have named the file \texttt{replay.xml} in the file
selector which pops up when the button has been pressed. Then you can
replay the installation with the following command invocation:

\begin{lstlisting}{}
# java -jar ExBib-setup.jar replay.xml
\end{lstlisting}

This supposes that the two files \File{ExTeX-setup.jar} and
\texttt{replay.xml} are in the current directory.
Finally you have to make sure that the executables \Prog{extex} or
\Prog{extex.bat} is on your path for executables.\index{path}


%------------------------------------------------------------------------------
\section{Running \ExTeX}
%@author Gerd Neugebauer

Currently \ExTeX\ can be run from the command line. In this respect it
is more or less identical to \TeX\ and can be used as a plug-in
replacement.

The following sample show a simple invocation of \ExTeX\ without any
command line arguments.

{\lstset{morecomment=[l]{*}}%
\begin{lstlisting}{}
# extex
This is ExTeX, Version 0.0 (TeX compatibility mode)
**\relax

*\end

No pages of output.
Transcript written on ./texput.log.
\end{lstlisting}}

In this case \ExTeX\ enters interaction with the user and asks for an
input file. This is indicated by the two asterisks. We have entered
\macro{relax} here to indicate that we are not willing to pass in a
file name. The \ExTeX\ system asks us to enter some command --
indicted by the single asterisk. Here we have entered \macro{end} to
indicate that we want to finish the processing. Thus \ExTeX\ 
terminates normally.

\INCOMPLETE

{\lstset{morecomment=[l]{*}}%
\begin{lstlisting}{}
# extex plain
This is ExTeX, Version 0.0 (TeX compatibility mode)
(plain Preloading the plain format: codes, registers, parameters, fonts,
more fonts, macros, math definitions, output routines, hyphenation(hyphen))
*\dump
Beginning to dump on file plain.fmt

*\end

No pages of output.
Transcript written on ./plain.log.
\end{lstlisting}}


\subsection{Command Line Parameters}
%@author Gerd Neugebauer

The invocation of the executable \Prog{extex} can be controlled by
large number of command line arguments. Those command line arguments
are described in the following list:

\begin{description}
\item[\Arg{code}]\ \\
  This parameter contains \ExTeX\ code to be executed directly. The
  execution is performed after any code specified in an input file. On
  the command line the code has to start with a backslash. This
  restriction does not hold for the property settings.

  This command line argument sets the property \Property{extex.code}
  
\item[\Arg{file}]\ \\
  This parameter contains the file to read from. A file name may not
  start with a backslash or an ambercent. It has no default.

  This command line argument sets the property \Property{extex.file}.
  
\item[\CLI{-} \Arg{file}]\ \\
  This parameter terminates the normal processing of arguments. The
  next argument -- if present -- is interpreted as input file. With
  this construction it is possible to process an input file which
  starts with one of the special characters \verb|\| or \verb|&|.

  This command line argument sets the property \Property{extex.file}
  if a file argument is present.

\item[\CLI{configuration} \Arg{resource}]\ \\
  This parameter contains the name of the configuration resource to
  use. This configuration resource is sought on the class path.
  
  This command line argument sets the property \Property{extex.config}.
  
\item[\CLI{copyright}]\ \\
  This command line option produces a copyright notice on the standard
  output stream and terminates the program afterwards.

\item[\tt\&\Arg{format}]\index{\&}
\item[\CLI{fmt} \Arg{format}]\ \\
  This parameter contains the name of the format to read. An empty
  string denotes that no format should be read. This is the default.

  This command line argument sets the property \Property{extex.format}.
  
\item[\CLI{debug} \Arg{spec}]\ \\
  This command line parameter can be used to instruct the program to
  produce debugging output of several kinds. The debug output is
  written to the log file. The specification \Arg{spec} is interpreted
  left to right. Each character is interpreted according to the
  following table:

  \begin{tabular}{lp{.4\textwidth}l}\toprule
    \textit{Spec}& \textit{Description}& \textit{See} \\\midrule
    F& 	This specifier contains the indicator whether or not to trace
    the searching for input files. & 	\Property{extex.trace.input.files}\\
    f& 	This specifier contains the indicator whether or not to trace
    the searching for font files.&      \Property{extex.trace.font.files}\\
    M& 	This specifier contains the indicator whether or not to trace
    the execution of macros.&	 	\Property{extex.trace.macros}\\
    T& 	This specifier contains the indicator whether or not to trace
    the work of the tokenizer.& 	\Property{extex.trace.tokenizer}\\
    \bottomrule
  \end{tabular}

  The following example shows a possible invocation with this
  parameter: 
\begin{lstlisting}{}
# extex -debug FfMT abc.tex
This is ExTeX, Version 0.0 (TeX compatibility mode)
...
\end{lstlisting}
  
\item[\CLI{halt-on-error}]\ \\
  This parameter contains the indicator whether the processing should
  halt after the first error which has been encountered.

  This command line argument sets the property \Property{extex.halt.on.error}.
  
\item[\CLI{help}]\ \\
  This command line option produces a short usage description on the
  standard output stream and terminates the program afterwards.
  
\item[\CLI{ini}]\ \\
  If set to true then act as ini\TeX.\index{initex@ini\TeX} In this
  case no format has to be preloaded. All parameters are set to the
  "`factory settings"'.

  This command line argument sets the property \Property{extex.ini}.

  The following example shows a possible invocation with this
  parameter: 
\begin{lstlisting}{}
# extex -ini abc.tex
This is ExTeX, Version 0.0 (TeX compatibility mode)
...
\end{lstlisting}
  
\item[\CLI{interaction} \Arg{mode}]\ \\
  This parameter contains the interaction mode. possible values are
  the numbers 0\dots3 and the symbolic names \Mode{batchmode} (0),
  \Mode{nonstopmode} (1), \Mode{scrollmode} (2), and
  \Mode{errorstopmode} (3).

  This command line argument sets the property \Property{extex.interaction}.
  
  The following example shows a possible invocation with this
  parameter:
\begin{lstlisting}{}
# extex -interaction batchmode abc.tex
This is ExTeX, Version 0.0 (TeX compatibility mode)
...
\end{lstlisting}

\item[\CLI{job-name} \Arg{name}]\ \\
  This parameter contains the name of the job. It is overwritten if a
  file is given to read from. In this case the base name of the input
  file is used instead.

  This command line argument sets the property \Property{extex.jobname}.
  
\item[\CLI{language} \Arg{language}]\ \\
  This parameter contains the name of the locale to be used for the
  messages.

  This command line argument sets the property \Property{extex.lang}.
  
\item[\CLI{output} \Arg{format}]\ \\
  This parameter contains the output format. This logical name is
  resolved via the configuration.

  This command line argument sets the property \Property{extex.output}.

  The following example shows a possible invocation with this
  parameter: 
\begin{lstlisting}{}
# extex -output pdf abc.tex
This is ExTeX, Version 0.0 (TeX compatibility mode)
\end{lstlisting}
  
\item[\CLI{progname} \Arg{name}]\ \\
  This parameter can be used to overrule the name of the program shown
  in the banner and the version information.  The following example
  shows a possible invocation and the resulting output:

\begin{lstlisting}{}
# extex -progname XeTxE -version
This is XeTxE, Version 0.0 (1.4.2_06)
#
\end{lstlisting}

  This command line argument sets the property \Property{extex.progname}.
  
\item[\CLI{texinputs} \Arg{path}]\ \\
  This parameter contains the additional directories for searching
  \ExTeX\ input files.  The directories are separated by the
  system-dependant separator.  This separator is a colon (\verb|:|) on
  Unix\index{Unix} and the semicolon (\verb|;|) on
  Windows\index{Windows}.
  
  This command line argument sets the property
  \Property{extex.texinputs}.
  
\item[\CLI{texmfoutputs} \Arg{dir}]\ \\
  This parameter contains the name of the property for the fallback if
  the output directory fails to be writable.
  
  This command line argument sets the property
  \Property{extex.outputdir.fallback}.
  
\item[\CLI{texoutputs} \Arg{dir}]\ \\
  This parameter contain the directory where output files should be
  created.

  This command line argument sets the property \Property{extex.outputdir}.
  
\item[\CLI{version}]\ \\
  This command line parameter forces that the version information is
  written to standard output and the program is
  terminated.\index{version} The version of \ExTeX\ is shown and the
  version of the Java engine\index{Java} in parentheses. The following
  example shows a possible invocation and the resulting output:

\begin{lstlisting}{}
# extex -version
This is ExTeX, Version 0.0 (1.4.2_06)
#
\end{lstlisting}
\end{description}

Command line parameters can be abbreviated up to a unique prefix --
and sometimes even more. Thus the following invocations are
equivalent:

\begin{verbatim}
  extex -v
  extex -ve
  extex -ver
  extex -vers
  extex -versi
  extex -versio
  extex -version  
\end{verbatim}

%%*****************************************************************************
%% $Id: bst-language.tex,v 0.00 2008/04/30 23:26:21 gene Exp $
%%*****************************************************************************
%% Author: Gerd Neugebauer
%%-----------------------------------------------------------------------------

\chapter{The Data Base}


\section{Syntax}

The data base in \BibTeX\index{BibTeX@\BibTeX} style consists of a
simple text file. \IM{08x1}

The following characters have a special meaning for the \BibTeX\
syntax:
\begin{verbatim}
    @ { } ( ) , # " =
\end{verbatim}
Any other character is treated equally as ordinary character.

An instruction is started with an at sign (@) followed by its name.
The name is composed of upper or lowercase letters and digits.

Whatever follows the name of the instruction depends on the
instruction. In most cases the parameters for the instruction are
following. They are enclosed in braces. For compatibility with
Scribe\index{Scribe} parentheses are sometimes allowed instead of the
braces.


\section{The \texttt{@input} Instruction}%
\index{@input|(}

Sometimes it might be desirable to split a database into several
segements. This is supported by the ability to pass inseveral
databases via the aux file. The \texttt{@input} instruction provides
another mechanism for the same which acts on the level of the database
files. \IM{x1}

The instruction takes as argument a resource name. It includes the
content as if it where present at the place of the instruction.

\begin{verbatim}
  @input(some/other/resource)
\end{verbatim}

\index{@input|)}

\section{Entries}

Any instruction which has no special meaning is considered to be an
entry in the database.
\IM{08x1}

\INCOMPLETE

\section{Names}\label{sec:names}
\IM{08x1}

Names are especially complicated and deserve a description of their
own.

\INCOMPLETE


\section{Comments}

Anything outside of entries and other declarations are considered as a
comment -- and mainly ignored. Thus you can put anything in between
the entries.

\IM{081} There is one special tag to mark comments. It is the tag
\texttt{@comment}. Since anything outside of declarations is already a
comment. Is has been considered sufficient to ignore the tag in the
input stream.
\begin{lstlisting}{}
  @comment
\end{lstlisting}

Unfortunately in the age of internet it is desirable to include email
addresses into comment -- and those may contain an @. Thus the
definition of the \texttt{@comment} declaration is slightly different
from \BibTeX\index{BibTeX@\BibTeX}:
\IM{x}
\begin{itemize}
\item If the next non-space character is an opening brace (\verb|{|)
  then a block is read and treated as comment. This means that the
  block can contain arbitrary characters -- especially the @ sign.
  On the other side the block needs to have balanced braces.
\begin{lstlisting}{}
  @comment{ This is a comment with embedded @ }
\end{lstlisting}
    
\item If the next non-space character is not an open brace character
  then just the tag is ignored.
\begin{lstlisting}{}
  @comment This is a comment
\end{lstlisting}
  
\end{itemize}


\section{The \texttt{@alias} Instruction}
\IM{x1}

\begin{verbatim}
  @alias( abc = def )
\end{verbatim}

\INCOMPLETE

\section{The \texttt{@modify} Instruction}

The \texttt{@modify} directive can be used to alter the content of
certain fields in an entry. The question is why would such a directive
be necessary when you could simply alter the entry itself. The answer
is that the entry might not be under your control. It might be
contained in another file which is included via the \texttt{@include}
directive.
\IM{x1}

\begin{verbatim}
  @modify( abc, author = {A.U. Thor} )
\end{verbatim}

\INCOMPLETE

\section{The \texttt{@string} Instruction}
\IM{08x1}

\begin{verbatim}
  @string( abc = {The value} )
\end{verbatim}

\INCOMPLETE

\section{The \texttt{@preamble} Instruction}
\IM{08x1}

\begin{verbatim}
  @preamble( "\providecommand\BibTeX{\textsc{Bib}\TeX}" )
\end{verbatim}

\INCOMPLETE


\endinput
%
% Local Variables: 
% mode: latex
% TeX-master: "../exbib-manual"
% End: 

%%*****************************************************************************
%% $Id: bst-language.tex,v 0.00 2008/04/30 23:26:21 gene Exp $
%%*****************************************************************************
%% Author: Gerd Neugebauer
%%-----------------------------------------------------------------------------

\chapter{The Styles}

\BibTeX~0.99c\index{BibTeX 0.99c@\BibTeX~0.99c} is accompanied by some
style files. They can be used out of the box. They are contained in
\ExBib\ as well. They are described here.

\section{plain}
\index{bst!plain|(}
\INCOMPLETE
\index{bst!plain|)}

\section{alpha}
\index{bst!alpha|(}
\INCOMPLETE
\index{bst!alpha|)}

\section{unsrt}
\index{bst!unsrt|(}
\INCOMPLETE
\index{bst!unsrt|)}

\section{abbrev}
\index{bst!abbrev|(}
\INCOMPLETE
\index{bst!abbrev|)}


\endinput
%
% Local Variables: 
% mode: latex
% TeX-master: "../exbib-manual"
% End: 

%%*****************************************************************************
%% $Id: bst-language.tex,v 0.00 2008/04/30 23:26:21 gene Exp $
%%*****************************************************************************
%% Author: Gerd Neugebauer
%%-----------------------------------------------------------------------------

\chapter{Tools}

\section{The \ExBib\ Util}

\subsection{Command Line Arguments}


\INCOMPLETE


\endinput
%
% Local Variables: 
% mode: latex
% TeX-master: "../exbib-manual"
% End: 

%%*****************************************************************************
%% $Id: bst-language.tex,v 0.00 2008/04/30 23:26:21 gene Exp $
%%*****************************************************************************
%% Author: Gerd Neugebauer
%%-----------------------------------------------------------------------------

\chapter{The BST Language}

\IM{08x1}%
The processing of the data base entries can be programmed with a small
special purpose programming language. Since it seems to habe no name
it is called the BST language

Usually the instructions are read from a file. The default extension
of these files is \texttt{.bst}. The content is interpreted to produce
the formatted output.

The primary goal of \BibTeX\ is the processing of bibliographic
databases. Thus the language is tailored towards the formatting of
bibliographies.

\section{The Programming Model}

The BST language is based on a simple stack based metaphor. The stack
is the central data structure in the program. The stack is able to
carry arbitrary data. Especially it is possible to push code segments
to the stack.

\begin{figure}[tb]
  \centering
  %%*****************************************************************************
%% $Id: bst-prog.tex,v 0.00 2008/05/01 13:56:41 gene Exp $
%%*****************************************************************************
%% Author: Gerd Neugebauer
%%-----------------------------------------------------------------------------
\begingroup
\begin{tikzpicture}[scale=.6]\sf\bfseries
  % --- bottom layer ---
  \shade[top color=white!97!blue,bottom color=white!90!blue,draw=blue!40!black,thick]
  (0,0) rectangle (15,10);
  \draw[fill=gray!20!white] (0,10) -- (.1,10.1) -- (15.1,10.1) -- (15,10) -- cycle;
  \draw[fill=gray!20!white] (15,10) -- (15.1,10.1) -- (15.1,.1) -- (15,0) -- cycle;

  % --- code ---
  \begin{scope}[shift={(6.25,9.5)},scale=.5]
    \begin{scope}[shift={(.5,.5)}]
      \shade[top color=white!90!yellow,bottom color=white!60!yellow,draw=red!40!black,thick]
      (0,0) -- (5,0) -- (5,2) -- (4,3) -- (0,3) -- cycle;
    \end{scope}
    \begin{scope}[shift={(.4,.4)}]
      \shade[top color=white!90!yellow,bottom color=white!60!yellow,draw=red!40!black,thick]
      (0,0) -- (5,0) -- (5,2) -- (4,3) -- (0,3) -- cycle;
    \end{scope}
    \begin{scope}[shift={(.3,.3)}]
      \shade[top color=white!90!yellow,bottom color=white!60!yellow,draw=red!40!black,thick]
      (0,0) -- (5,0) -- (5,2) -- (4,3) -- (0,3) -- cycle;
    \end{scope}
    \begin{scope}[shift={(.2,.2)}]
      \shade[top color=white!90!yellow,bottom color=white!60!yellow,draw=red!40!black,thick]
      (0,0) -- (5,0) -- (5,2) -- (4,3) -- (0,3) -- cycle;
    \end{scope}
    \begin{scope}[shift={(.1,.1)}]
      \shade[top color=white!90!yellow,bottom color=white!60!yellow,draw=red!40!black,thick]
      (0,0) -- (5,0) -- (5,2) -- (4,3) -- (0,3) -- cycle;
    \end{scope}
    \shade[top color=white!90!yellow,bottom color=white!60!yellow,draw=red!40!black,thick]
    (0,0) -- (5,0) -- (5,2) -- (4,3) -- (0,3) -- cycle;
  \end{scope}
  \draw (7.5,10.25) node {Code};

  % --- interpreter ---
  \shade[top color=white!90!blue,bottom color=white!60!blue,draw=blue!40!black,thick]
  (5,1) rectangle (10,9);
  \draw[fill=gray!20!white] (5,9) -- (5.1,9.1) -- (10.1,9.1) -- (10,9) -- cycle;
  \draw[fill=gray!20!white] (10,9) -- (10.1,9.1) -- (10.1,1.1) -- (10,1) -- cycle;
  \draw (7.5,5) node {Interpreter};
  
  \shade[top color=white!90!green,bottom color=white!60!green,draw=green!40!black,thick]
  (5,-.5) .. controls (6,0) and (9,-1) .. (10,-.5) -- (10,.5) -- (5,.5) -- cycle;
  \draw (7.5,0) node {Output Stream};


  % --- stack ---
  \shade[top color=white!10!pink,bottom color=white!90!pink,draw=pink!70!black,shift={(11.5,1)}] (0,0) -- (.5,1) -- (2.5,1) -- (2,0) -- cycle;
  \shade[top color=white!10!pink,bottom color=white!90!pink,draw=pink!70!black,shift={(11.5,1.5)}] (0,0) -- (.5,1) -- (2.5,1) -- (2,0) -- cycle;
  \shade[top color=white!10!pink,bottom color=white!90!pink,draw=pink!70!black,shift={(11.5,2)}] (0,0) -- (.5,1) -- (2.5,1) -- (2,0) -- cycle;
  \shade[top color=white!10!pink,bottom color=white!90!pink,draw=pink!70!black,shift={(11.5,2.5)}] (0,0) -- (.5,1) -- (2.5,1) -- (2,0) -- cycle;
  \shade[top color=white!10!pink,bottom color=white!90!pink,draw=pink!70!black,shift={(11.5,3)}] (0,0) -- (.5,1) -- (2.5,1) -- (2,0) -- cycle;
  \shade[top color=white!10!pink,bottom color=white!90!pink,draw=pink!70!black,shift={(11.5,3.5)}] (0,0) -- (.5,1) -- (2.5,1) -- (2,0) -- cycle;
  \shade[top color=white!10!pink,bottom color=white!90!pink,draw=pink!70!black,shift={(11.5,4)}] (0,0) -- (.5,1) -- (2.5,1) -- (2,0) -- cycle;
  \shade[top color=white!10!pink,bottom color=white!90!pink,draw=pink!70!black,shift={(11.5,4.5)}] (0,0) -- (.5,1) -- (2.5,1) -- (2,0) -- cycle;
  \shade[top color=white!10!pink,bottom color=white!90!pink,draw=pink!70!black,shift={(11.5,5)}] (0,0) -- (.5,1) -- (2.5,1) -- (2,0) -- cycle;
  \shade[top color=white!10!pink,bottom color=white!90!pink,draw=pink!70!black,shift={(11.5,5.5)}] (0,0) -- (.5,1) -- (2.5,1) -- (2,0) -- cycle;
  \shade[top color=white!10!pink,bottom color=white!90!pink,draw=pink!70!black,shift={(11.5,6)}] (0,0) -- (.5,1) -- (2.5,1) -- (2,0) -- cycle;
  \shade[top color=white!10!pink,bottom color=white!90!pink,draw=pink!70!black,shift={(11.5,6.5)}] (0,0) -- (.5,1) -- (2.5,1) -- (2,0) -- cycle;
  \shade[top color=white!10!pink,bottom color=white!90!pink,draw=pink!70!black,shift={(11.5,7)}] (0,0) -- (.5,1) -- (2.5,1) -- (2,0) -- cycle;
  \shade[top color=white!10!pink,bottom color=white!90!pink,draw=pink!70!black,shift={(11.5,7.5)}] (0,0) -- (.5,1) -- (2.5,1) -- (2,0) -- cycle;
  \shade[top color=white!10!pink,bottom color=white!90!pink,draw=pink!70!black,shift={(11.5,8)}] (0,0) -- (.5,1) -- (2.5,1) -- (2,0) -- cycle;
  \draw (12.75,8.5) node {Stack};

  % --- integers ---
  \shade[top color=white!90!pink,bottom color=white!10!pink,draw=pink!70!black]
  (1,7) rectangle (4,7.4);
  \shade[shift={(0,.4)},top color=white!90!pink,bottom color=white!10!pink,draw=pink!70!black]
  (1,7) rectangle (4,7.4);
  \shade[shift={(0,.8)},top color=white!90!pink,bottom color=white!10!pink,draw=pink!70!black]
  (1,7) rectangle (4,7.4);
  \shade[shift={(0,1.2)},top color=white!90!pink,bottom color=white!10!pink,draw=pink!70!black]
  (1,7) rectangle (4,7.4);
  \shade[shift={(0,1.6)},top color=white!90!pink,bottom color=white!10!pink,draw=pink!70!black]
  (1,7) rectangle (4,7.4);
  \draw (1,7) rectangle (4,9);
  \draw (2.5,8) node {Integers};

  % --- strings ---
  \shade[top color=white!90!pink,bottom color=white!10!pink,draw=pink!70!black]
  (1,4) rectangle (4,4.4);
  \shade[shift={(0,.4)},top color=white!90!pink,bottom color=white!10!pink,draw=pink!70!black]
  (1,4) rectangle (4,4.4);
  \shade[shift={(0,.8)},top color=white!90!pink,bottom color=white!10!pink,draw=pink!70!black]
  (1,4) rectangle (4,4.4);
  \shade[shift={(0,1.2)},top color=white!90!pink,bottom color=white!10!pink,draw=pink!70!black]
  (1,4) rectangle (4,4.4);
  \shade[shift={(0,1.6)},top color=white!90!pink,bottom color=white!10!pink,draw=pink!70!black]
  (1,4) rectangle (4,4.4);
  \draw (1,4) rectangle (4,6);
  \draw (2.5,5) node {Strings};

  % --- entries ---
  \shade[top color=white!90!pink,bottom color=white!10!pink,draw=pink!70!black]
  (1,1) rectangle (4,1.4);
  \shade[shift={(0,.4)},top color=white!90!pink,bottom color=white!10!pink,draw=pink!70!black]
  (1,1) rectangle (4,1.4);
  \shade[shift={(0,.8)},top color=white!90!pink,bottom color=white!10!pink,draw=pink!70!black]
  (1,1) rectangle (4,1.4);
  \shade[shift={(0,1.2)},top color=white!90!pink,bottom color=white!10!pink,draw=pink!70!black]
  (1,1) rectangle (4,1.4);
  \shade[shift={(0,1.6)},top color=white!90!pink,bottom color=white!10!pink,draw=pink!70!black]
  (1,1) rectangle (4,1.4);
  \draw[shift={(.3,0)},draw=pink!70!black] (1,1) -- (1,3);
  \draw[shift={(.6,0)},draw=pink!70!black] (1,1) -- (1,3);
  \draw[shift={(.9,0)},draw=pink!70!black] (1,1) -- (1,3);
  \draw[shift={(1.2,0)},draw=pink!70!black] (1,1) -- (1,3);
  \draw[shift={(1.5,0)},draw=pink!70!black] (1,1) -- (1,3);
  \draw[shift={(1.8,0)},draw=pink!70!black] (1,1) -- (1,3);
  \draw[shift={(2.1,0)},draw=pink!70!black] (1,1) -- (1,3);
  \draw[shift={(2.4,0)},draw=pink!70!black] (1,1) -- (1,3);
  \draw[shift={(2.7,0)},draw=pink!70!black] (1,1) -- (1,3);
  \draw (1,1) rectangle (4,3);
  \draw (2.5,2) node {Entries};

  \begin{scope}[shift={(-5,0)}]
    \shade[left color=white!96!blue!70!pink,right color=white!60!blue!60!pink,draw=pink!70!black]
    (0,0)   .. controls (0,-.4) and (1.5,-.5) ..
    (2,-.5)  .. controls (2.5,-.5) and (4,-.4) ..
    (4,0) -- (4,4) -- (0,4) -- cycle;
    \draw[shift={(0,4)},fill=white!96!blue!70!pink,draw=pink!60!black]
    (0,0)   .. controls (0,.4) and (1.5,.5) ..
    (2,.5)  .. controls (2.5,.5) and (4,.4) ..
    (4,0)   .. controls (4,-.4) and (2.5,-.5) ..
    (2,-.5) .. controls (1.5,-.5) and (0,-.4) ..
    (0,0);
  \end{scope}
  \draw (-3,2) node {Database};

%  \draw[shift={(0,-5)}]
%  (-.2,0) -- (0,.4) -- (0,.2) -- (1,.2) -- (1,.4) -- (1.2,0) --
%  (1,-.4) -- (1,-.2) -- (0,-.2) -- (0,-.4) -- cycle;

  \shade[shift={(-.7,2)}, top color=white, bottom color=gray,draw=gray]
  (-.2,0) -- (0,.4) -- (0,.2) -- (1.4,.2) -- (1.4,.4) -- (1.6,0) --
  (1.4,-.4) -- (1.4,-.2) -- (0,-.2) -- (0,-.4) -- cycle;

\end{tikzpicture}
\endgroup
\endinput
%
% Local Variables: 
% mode: latex
% TeX-master: nil
% End: 

  \caption{The BST Programming Model}
  \label{fig:bst-model}
\end{figure}


\subsection{The Database Context}

Whenever a bst program is executed a database is at hand. The porogram
has access to this database.

\INCOMPLETE

\subsection{Global Integers}

\INCOMPLETE

\subsection{Global Strings}

\INCOMPLETE

\subsection{Code}

\INCOMPLETE


\section{Syntax}

\INCOMPLETE

\section{Commands}


\subsection{\texttt{entry}}

\INCOMPLETE

\subsection{\texttt{integers}}

\INCOMPLETE

\begin{lstlisting}{}
  INTEGERS { output.state before.all }
\end{lstlisting}

\subsection{\texttt{strings}}

\INCOMPLETE

\begin{lstlisting}{}
  STRINGS { s t }
\end{lstlisting}

\subsection{\texttt{macro}}

\INCOMPLETE

\begin{lstlisting}{}
  MACRO {jan} {"January"}
\end{lstlisting}

\subsection{\texttt{execute}}

\INCOMPLETE

\begin{lstlisting}{}
  EXECUTE {begin.bib}
\end{lstlisting}

\subsection{\texttt{iterate}}

\INCOMPLETE

\begin{lstlisting}{}
  ITERATE{call.type$}
\end{lstlisting}

\subsection{\texttt{reverse}}

\INCOMPLETE

\begin{lstlisting}{}
  REVERSE{reverse.pass}
\end{lstlisting}

\subsection{\texttt{sort}}

\INCOMPLETE

\begin{lstlisting}{}
  SORT
\end{lstlisting}

\subsection{\texttt{read}}

\INCOMPLETE

\begin{lstlisting}{}
  READ
\end{lstlisting}

\subsection{\texttt{function}}

\INCOMPLETE

\begin{lstlisting}{}
  FUNCTION {sortify}
  { purify$
    "l" change.case$
  }
\end{lstlisting}


\subsection{\texttt{input}}
\IM{x}

\INCOMPLETE

\begin{lstlisting}{}
  INPUT{some/other/bst}
\end{lstlisting}

\section{Instructions}

\subsection{\texttt{>}}

\INCOMPLETE

\subsection{\texttt{<}}

\INCOMPLETE

\subsection{\texttt{=}}

\INCOMPLETE

\subsection{\texttt{+}}

\INCOMPLETE

\subsection{\texttt{-}}

\INCOMPLETE

\subsection{\texttt{*}}

\INCOMPLETE

\subsection{\texttt{:=}}

\INCOMPLETE

\subsection{\texttt{add.period\$}}

\INCOMPLETE

\subsection{\texttt{call.type\$}}

\INCOMPLETE

\subsection{\texttt{change.case\$}}

\INCOMPLETE

\subsection{\texttt{chr.to.int\$}}

\INCOMPLETE

\subsection{\texttt{cite\$}}

\begin{lstlisting}{}
  cite$
\end{lstlisting}

\INCOMPLETE

\subsection{\texttt{duplicate\$}}

\INCOMPLETE

\subsection{\texttt{empty\$}}

\INCOMPLETE

\subsection{\texttt{format.name\$}}

\INCOMPLETE

\subsection{\texttt{if\$}}

\INCOMPLETE

\subsection{\texttt{int.to.chr\$}}

\INCOMPLETE

\subsection{\texttt{int.to.str\$}}

\INCOMPLETE

\subsection{\texttt{missing\$}}

\INCOMPLETE

\subsection{\texttt{newline\$}}

This instruction writes a newline character to the output stream.

\begin{lstlisting}{}
  newline$
\end{lstlisting}

\subsection{\texttt{num.names\$}}

\INCOMPLETE

\subsection{\texttt{pop\$}}

\INCOMPLETE

\subsection{\texttt{preamble\$}}

\INCOMPLETE

\subsection{\texttt{purify\$}}

\INCOMPLETE

\subsection{\texttt{quote\$}}

\INCOMPLETE

\subsection{\texttt{skip\$}}

This instruction does simply nothing. In other languages this might be
called a noop. This is useful for places where instructions are
mandatory but nothing should be done.

\subsection{\texttt{stack\$}}

\INCOMPLETE

\subsection{\texttt{substring\$}}

\INCOMPLETE

\subsection{\texttt{swap\$}}

This instruction takes the two topmost elements from the stack and
exchanges their order on the stack. This means that the topmost
element becomes the second one and the second element becomes the
topmost one. The elements can be of any type.

If there are less than two elements on the stack then an error is
raised.

\subsection{\texttt{text.length\$}}

\INCOMPLETE

\subsection{\texttt{text.prefix\$}}

\INCOMPLETE

\subsection{\texttt{top\$}}

\INCOMPLETE

\subsection{\texttt{type\$}}

\INCOMPLETE

\subsection{\texttt{warning\$}}

This instruction pops a string from the stack and prints it as a
warning to the log stream.

\subsection{\texttt{while\$}}

\INCOMPLETE

\subsection{\texttt{width\$}}

This instruction pops a string from the stack and tries to compute the
width of the string when tyoeset. For this purpose the width of the
characters in the font cmr10 are used.

\subsection{\texttt{write\$}}

This instruction pops a string from the stack and prints it as a
message to the output stream.


\endinput
%
% Local Variables: 
% mode: latex
% TeX-master: "exbib-users"
% End: 



%------------------------------------------------------------------------------
\bibliographystyle{alpha}
\bibliography{references}

%------------------------------------------------------------------------------
{\scriptsize\printindex}

\end{document}%%%%%%%%%%%%%%%%%%%%%%%%%%%%%%%%%%%%%%%%%%%%%%%%%%%%%%%%%%%%%%%%%
%
% Local Variables: 
% mode: latex
% TeX-master: nil
% End: 
