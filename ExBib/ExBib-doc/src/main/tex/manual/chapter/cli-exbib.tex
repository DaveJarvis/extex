%%*****************************************************************************
%% $Id:cli-exbib.tex 7067 2008-05-18 11:06:56Z gene $
%%*****************************************************************************
%% Author: Gerd Neugebauer
%%-----------------------------------------------------------------------------


%------------------------------------------------------------------------------
\section{Running \ExBib}
%@author Gerd Neugebauer

Currently \ExBib\ can be run from the command line. In this respect it
is more or less identical to \BibTeX\index{BibTeX@\BibTeX} and can be
used as a plug-in replacement. In addition the features of
\BibTeX~8\index{BibTeX 8@\BibTeX~8} are present as well. The marks in
the margin indicate where the different features are coming from.

The executable for \ExBib\ is called \Prog{exbib}. It is a wrapper
for the real program. It comes as shell script \Prog{exbib} and the
Windows batch script \Prog{exbib.bat}. After a successful installation
those scripts should be resent on your execution path.


\subsection{Dot Files}%
\label{sec:dot.files}

Dot files are a means to provide settings in a file -- in contrast to
the command line parameters. Since the name starts in Unix\index{Unix}
with a dot they are called dot files.

\ExBib\ supports dot files. The name of the dot file is \File{.exbib}.
This file is read in the user's home directory and the current
directory -- if they exist. The settings performed in the different
places have a certain precedency. From lowest to highest they are:

\begin{itemize}
\item compiled-in defaults
\item \File{.exbib} in the home directory
\item \File{.exbib} in the current directory
\item command line arguments
\end{itemize}

Empty lines and lines starting -- after inital whitespace -- with a
hash mark (\#) are treated as comments and ignored.
\begin{lstlisting}[language=sh]
# This is a comment
\end{lstlisting}

The dot files contain key value pairs in a line-based style. The key
and the value are separated by an equality sign (=). Whitespace before
the key and around the equality sign is ignored. The value follows the
equality sign.
\begin{lstlisting}[language=sh]
exbib.file = file.aux
\end{lstlisting}

If the last character of the value on one line is a backslash
(\textbackslash) then the value is continued on the next line. The
leading whitespace on the next line is ignored.
\begin{lstlisting}{}
exbib.file = file.\
             aux
\end{lstlisting}

The following properties are recognized in a dot file:

\begin{description}
\item[\Property{exbib.file}] \ \\
  This property contains the name of the aux file to read. Note that
  only one aux file can be given. The value from the dot file in the
  current directory silently overwrites the value from the dot file in
  the home directory. If an additional aux file is given on the
  command line then an error is raised.
\item[\Property{exbib.config}] \ \\
  This property contains the name of the configuration to be used. The
  default is \texttt{exbib}.
\item[\Property{exbib.logfile}] \ \\
  This property contains the name of the log file. If it is empty then
  no log file will be written. The default is derived from the base
  name of the aux file by appending \texttt{.log}.
\item[\Property{exbib.bib.encoding}] \ \\
  This property contains the name of the encoding for reading bib
  databases. If it is not present or empty then the platform default
  is used as fallback.
\item[\Property{exbib.encoding}] \ \\
  This property contains the name of the encoding for reading aux
  files and writing. If it is not present or empty then the platform
  default is used as fallback.
\item[\Property{exbib.csf.encoding}] \ \\
  This property contains the name of the encoding for reading cs
  files. If it is not present or empty then the platform default is
  used as fallback.
\item[\Property{exbib.csf}] \ \\
  This property contains the name of a cs file to define the
  characters and sorting properties.
\item[\Property{texinputs}] \ \\
  This property contains a path for searching resources. The
  environment variable \Env{TEXINPUTS} is put into this property.
  Otherwise the default is empty.
\item[\Property{extex.texinputs}]  \ \\
  This property contains a path for searching resources. The
  environment variable \Env{EXTEX\_TEXINPUTS} is put into this property.
  Otherwise the default is empty.
\item[\Property{exbib.output}] \ \\
  This property contains the name of the output file. It replaces the
  bbl file.
\item[\Property{program.name}] \ \\
  This propery can be used to influence how the program calles itself
  in messages. User's should rarely need to play around with this
  property.
\end{description}

Note that any key can be used in a dot file. Unknown keys are silently
ignored. Thus watch out for typos in keys.


\subsubsection{Home Directory on Unix}

The home directory on a Unix\index{Unix} system is user specific
setting. It is determined from the environment variable \Env{HOME}. A
typical value is \texttt{/home/donald}.

\subsubsection{Home Directory on Windows}

The home directory on a Windows\index{Windows} system is a user
specific setting.  A typical value is
\verb|C:\windows\profiles\donald|.



\subsection{Command Line Parameters}
%@author Gerd Neugebauer

The invocation of the executable \Prog{exbib} can be controlled by
large number of command line arguments. Those command line arguments
are described in the following sections.

\subsubsection{Defining Properties}

\begin{description}
\item[\CLi{D}\Arg{property}\texttt{=}\Arg{value}]\ \\
  
  The preferred way to communicate settings to \ExBib\ is via
  properties. The most command line options are for convenience or
  backward compatibility.
  
  The setting of arbitrary properties can be accomplished with the
  command line argument \texttt{-D}. The properties are the same as
  those used in dot files (see section~\ref{sec:dot.files}). This
  command line option as shown in the following example:

\begin{lstlisting}[language=sh]
# exbib -Dexbib.file=file.aux
\end{lstlisting}

\item[\CLI{load} \Arg{file}]\ \\
  
  At startup \Prog{exbib} tries to load settings from dot files ( see
  section~\ref{sec:dot.files}). In addition the command line can name
  other file to read settings from. In contrast to the dot files the
  files named explicitly have to exist or an error is raised.
  
  The properties are the same as those used in dot files (see
  section~\ref{sec:dot.files}). This command line option as shown in
  the following example:

\begin{lstlisting}[language=sh]
# exbib -load ../.exbib
\end{lstlisting}

\end{description}


\subsubsection{The Aux File}

\begin{description}
\item[\Arg{file}]\ \\
  This parameter contains the aux file to read from. A file name may not
  start with minus sign. It has no default.
  \begin{lstlisting}[language=sh,escapechar=!]
    # exbib doc.aux
    This is exbib, Version !\Version!
  \end{lstlisting}
  
\item[\CLi{} \Arg{file}]
\item[\CLI{} \Arg{file}]\ \\
  This parameter intercepts the processing of arguments. The next
  argument -- if present -- is interpreted as input file. With this
  construction it is possible to process an input file which starts
  with the special character \verb|-|.
  \begin{lstlisting}[language=sh,escapechar=!]
    # exbib -- doc.aux
    This is exbib, Version !\Version!
  \end{lstlisting}

\end{description}

The file name given is used to determine the name of an aux file. This
means that it is either the name of an aux file or the base name which
is augmented by the extension \texttt{.aux} to find the aux file.

The main control information is taken from this aux file. This means
it contains the following items:

\begin{itemize}
\item The database files to consult.
\item The citations to extract.
\item The bst to use for formatting.
\item References to other aux files to consult as well.
\end{itemize}

\subsubsection{Version Information and Help}

\begin{description}
\item[\CLI{copying}]\ \\
  This command line option produces a copyright notice on the standard
  output stream and terminates the program afterwards.
  \begin{lstlisting}[language=sh]
# exbib --copying
                 GNU LESSER GENERAL PUBLIC LICENSE
                      Version 2.1, February 1999
Copyright (C) 1991, 1999 Free Software Foundation, Inc.
    59 Temple Place, Suite 330, Boston, MA  02111-1307  USA
Everyone is permitted to copy and distribute verbatim copies
of this license document, but changing it is not allowed.
[This is the first released version of the Lesser GPL.  It also counts
as the successor of the GNU Library Public License, version 2, hence
the version number 2.1.]
                           Preamble
 The licenses for most software are designed to take away your
freedom to share and change it.  By contrast, the GNU General Public
Licenses are intended to guarantee your freedom to share and change
free software--to make sure the software is free for all its users.
    ...
  \end{lstlisting}

\item[\CLi{h}]
\item[\CLi{?}]
\item[\CLI{help}]\ \\
  This command line option produces a short usage description on the
  standard output stream and terminates the program afterwards.
  \begin{lstlisting}[language=sh,escapechar=!]
# exbib --help
This is exbib, Version !\Version!
Usage: exbib <options> file
The following options are supported:
        -[-] <file>
                Use this argument as file name -- even when it looks like an option.
        --trad[itional] | -7
                operate in the original 7-bit mode.
        --8[bit] | -8
                force 8-bit mode, no CS file used.
    ...
\end{lstlisting}
  
\item[\CLI{release}]\ \\
  This command prints the release number to stdout and exits the
  program. This can be used to enable external programs to easily
  determine the version number of \ExBib.
\begin{lstlisting}[language=sh,escapechar=!]
# exbib --release
!\Version!
#
\end{lstlisting}

\item[\CLI{version}]\ \\
  This command line parameter forces that the version information is
  written to standard output and the program is
  terminated.\index{version} The version of \ExBib\ is shown. The
  following example shows a possible invocation and the resulting
  output:
\begin{lstlisting}[language=sh,escapechar=!]
# exbib --version
This is exbib, Version !\Version!
Copyright (C) 2002-2008 Gerd Neugebauer (mailto:gene@gerd-neugebauer.de).
There is NO warranty.  Redistribution of this software is
covered by the terms of the GNU Library General Public License.
For more information about these matters, use the command line
switch -copying.
#
\end{lstlisting}
\end{description}


\subsubsection{Internationalization}

\begin{description}
\item[\CLi{L} \Arg{lang}]
\item[\CLI{language} \Arg{lang}]\ \\
  This command line option switches the language to the given
  language. The argument is a two-letter ISO code for a language. For
  instance the value \texttt{en} represents English and \texttt{de}
  represents German.

  The language is used to select the appropriate messages for logging
  and error messages. If the given language is not supported English
  is silently used as fallback.
\begin{lstlisting}[language=sh,escapechar=!]
# exbib --language de --help
Dies ist exbib, Version !\Version!
Copyright (C) 2002-2008 Gerd Neugebauer (mailto:gene@gerd-neugebauer.de).
There is NO warranty.  Redistribution of this software is
covered by the terms of the GNU Library General Public License.
For more information about these matters, use the command line
switch -copying.
#
\end{lstlisting}
\end{description}


\subsubsection{Configurations}%
\label{sec:cli.cfg}%
\index{configuration|(}

\ExBib\ is highly configurable. The whole system is assembled from
components at startup time. The assembly is controlled from a set of
configuration files. There is one central configuration file which
acts as entry point.

\begin{description}
\item[\CLi{c} \Arg{config}]
\item[\CLI{configuration} \Arg{config}]\ \\
  Set the configuration file to be used for assembling \ExBib. The
  default is the configuration \texttt{exbib}.
  \begin{lstlisting}[language=sh,escapechar=!]
# exbib --configuration bibtex099 doc.aux
This is exbib, Version !\Version!
...
\end{lstlisting}

\item[\CLI{bibtex}]
\item[\CLI{strict}]\ \\
  Use the configuration \texttt{bibtex099} for assembling \ExBib.
\begin{lstlisting}[language=sh,escapechar=!]
# exbib --bibtex doc.aux
This is exbib, Version !\Version!
...
\end{lstlisting}
\end{description}

The following configurations are present in the distribution.

\begin{description}
\item[exbib]\ \\
  This is the default configuration which includes all features
  described in this reference manual.
\item[bibtex099]\ \\
  This is the configuration for backward compatibility. It emulates
  the features of \BibTeX~0.99c as closely as possible. The extended
  features of \ExBib\ are not present in this configuration.
\end{description}

\index{configuration|)}

\subsubsection{Encodings}%
\label{sec:encodings}%
\index{encoding|(}

The internal representation of characters uses Unicode\index{Unicode}.
In general it is necessary to translate from and to the internal
representation when reading and writing files. For this purpose the
encodings to be used can be configured.

The default is to use the default encoding for the platform \ExBib\ is
currently running. Thus it is not necessary to specify an encoding at
all.

It is guaranteed that at least the following encodings are present on
your system:

\begin{description}
\item[US-ASCII\index{encoding!US-ASCII}] 
  Seven-bit ASCII.
\item[ISO-8859-1\index{encoding!ISO-8859-1}] 
  ISO Latin Alphabet 1
\item[UTF-8\index{encoding!UTF8}] 
  Eight-bit UCS Transformation Format.
\item[UTF-16BE\index{encoding!UTF16BE}] 
  Sixteen-bit UCS Transformation Format in big-endian byte order.
\item[UTF-16LE\index{encoding!UTF16LE}] 
  Sixteen-bit UCS Transformation Format in little-endian byte order.
\item[UTF\index{encoding!UTF}] Sixteen-bit UCS Transformation Format;
  the byte order identified by an optional byte-order mark.
\end{description}

The following list has been obtained at the time of writing this
document (on a Windows system):

\noindent
\begin{multicols}5\obeylines\scriptsize\parindent=0pt
  Big5
  Big5-HKSCS
  EUC-JP
  EUC-KR
  GB18030
  GB2312
  GBK
  IBM-Thai
  IBM00858
  IBM01140
  IBM01141
  IBM01142
  IBM01143
  IBM01144
  IBM01145
  IBM01146
  IBM01147
  IBM01148
  IBM01149
  IBM037
  IBM1026
  IBM1047
  IBM273
  IBM277
  IBM278
  IBM280
  IBM284
  IBM285
  IBM297
  IBM420
  IBM424
  IBM437
  IBM500
  IBM775
  IBM850
  IBM852
  IBM855
  IBM857
  IBM860
  IBM861
  IBM862
  IBM863
  IBM864
  IBM865
  IBM866
  IBM868
  IBM869
  IBM870
  IBM871
  IBM918
  ISO-2022-CN
  ISO-2022-JP
  ISO-2022-JP-2
  ISO-2022-KR
  ISO-8859-1
  ISO-8859-13
  ISO-8859-15
  ISO-8859-2
  ISO-8859-3
  ISO-8859-4
  ISO-8859-5
  ISO-8859-6
  ISO-8859-7
  ISO-8859-8
  ISO-8859-9
  JIS\_X0201
  JIS\_X0212-1990
  KOI8-R
  KOI8-U
  Shift\_JIS
  TIS-620
  US-ASCII
  UTF-16
  UTF-16BE
  UTF-16LE
  UTF-32
  UTF-32BE
  UTF-32LE
  UTF-8
  windows-1250
  windows-1251
  windows-1252
  windows-1253
  windows-1254
  windows-1255
  windows-1256
  windows-1257
  windows-1258
  windows-31j
  x-Big5-Solaris
  x-euc-jp-linux
  x-EUC-TW
  x-eucJP-Open
  x-IBM1006
  x-IBM1025
  x-IBM1046
  x-IBM1097
  x-IBM1098
  x-IBM1112
  x-IBM1122
  x-IBM1123
  x-IBM1124
  x-IBM1381
  x-IBM1383
  x-IBM33722
  x-IBM737
  x-IBM834
  x-IBM856
  x-IBM874
  x-IBM875
  x-IBM921
  x-IBM922
  x-IBM930
  x-IBM933
  x-IBM935
  x-IBM937
  x-IBM939
  x-IBM942
  x-IBM942C
  x-IBM943
  x-IBM943C
  x-IBM948
  x-IBM949
  x-IBM949C
  x-IBM950
  x-IBM964
  x-IBM970
  x-ISCII91
  x-ISO-2022-CN-CNS
  x-ISO-2022-CN-GB
  x-iso-8859-11
  x-JIS0208
  x-JISAutoDetect
  x-Johab
  x-MacArabic
  x-MacCentralEurope
  x-MacCroatian
  x-MacCyrillic
  x-MacDingbat
  x-MacGreek
  x-MacHebrew
  x-MacIceland
  x-MacRoman
  x-MacRomania
  x-MacSymbol
  x-MacThai
  x-MacTurkish
  x-MacUkraine
  x-MS950-HKSCS
  x-mswin-936
  x-PCK
  x-UTF-16LE-BOM
  X-UTF-32BE-BOM
  X-UTF-32LE-BOM
  x-windows-50220
  x-windows-50221
  x-windows-874
  x-windows-949
  x-windows-950
  x-windows-iso2022jp
\end{multicols}

The following command line options are related to encodings:

\begin{description}
\item[\CLI{availableCharsets}]\ \\
  This instruction lists the available character sets on standard
  output and exits the program.
\begin{lstlisting}[language=sh]
# exbib --availableCharsets
Big5
Big5-HKSCS
EUC-JP
EUC-KR
GB18030
GB2312
GBK
IBM-Thai
IBM00858
IBM01140
...
\end{lstlisting}

\item[\CLi{E} \Arg{enc}]
\item[\CLI{bib-encoding} \Arg{enc}]
\item[\CLI{bib.encoding} \Arg{enc}]\ \\
  Set the encoding for reading bib database files. The encoding needs
  to be a valid character set. The fallback is the same encoding as
  the one given with \CLI{encoding}.
\begin{lstlisting}[language=sh]
# exbib --bib-encoding=UTF8 doc.aux
...
\end{lstlisting}

\item[\CLI{csf-encoding} \Arg{enc}]
\item[\CLI{csf.encoding} \Arg{enc}]\ \\
  Set the encoding for reading cs files. The encoding needs to be a
  valid character set. The fallback is the platform default encoding.
\begin{lstlisting}[language=sh]
# exbib --csf=german.csf --csf-encoding=UTF8 doc.aux
...
\end{lstlisting}

\item[\CLi{e} \Arg{enc}]
\item[\CLI{encoding} \Arg{enc}]\ \\
  Set the encoding for reading aux files and writing files. The
  encoding needs to be a valid character set. The fallback is the
  platform default encoding\index{encoding}.
\begin{lstlisting}[language=sh]
# exbib --encoding=UTF8 doc.aux
...
\end{lstlisting}

\end{description}
\index{encoding|)}

\subsubsection{CS Files}%
\index{csf|(}

Cs files contain trhe definition of the character set and sortig
order. They have been introduced in \BibTeX~8\index{BibTeX 8|\BibTeX 8}.
For details on the format refer to section~\ref{sec:csf}

\begin{description}
\item[\CLI{csfile} \Arg{csfile}]\ \\
  Read the definition of the character sets and sorting order from the
  given cs file. Refer to section~\ref{sec:csf} for details about the
  format of the file.
\begin{lstlisting}[language=sh]
# exbib --csfile=iso8859-7.csf doc.aux
...
\end{lstlisting}

\item[\CLi{7}]
\item[\CLI{traditional}]\ \\
%        	arbeite im originalen 7-Bit-Modus.
  \INCOMPLETE
\begin{lstlisting}[language=sh]
# exbib --traditional doc.aux
...
\end{lstlisting}

\item[\CLi{8}]
\item[\CLI{8bit}]\ \\
%        	Erzwinge  8-Bit-Modus, es wird keine CS-Datei eingesetzt.
  \INCOMPLETE
\begin{lstlisting}[language=sh]
# exbib --8bit doc.aux
...
\end{lstlisting}
\end{description}
\index{csf|)}

\subsubsection{Redirecting Output}

\begin{description}
\item[\CLi{l} \Arg{file}]
\item[\CLI{logfile} \Arg{file}]\ \\
  This option redirects the log output to the given file. The default
  name of the log file is derived from the base name of the aux file
  by appending \texttt{.blg}. This option overwrites this default.
  behaviour.
\begin{lstlisting}[language=sh]
# exbib --logfile=my.log doc.aux
...
\end{lstlisting}

  If the given file name is the value \texttt{-} then the output is
  sent to stdout.
\begin{lstlisting}[language=sh]
# exbib --logfile=- doc.aux
...
\end{lstlisting}

  If the given file name is empty then the log output is discarded.
\begin{lstlisting}[language=sh]
# exbib --logfile= doc.aux
...
\end{lstlisting}

\item[\CLi{o} \Arg{file}]
\item[\CLI{output} \Arg{file}]
\item[\CLI{outfile} \Arg{file}]\ \\
  This option redirects the output to the given file. The default
  name of the output file is derived from the base name of the aux file
  by appending \texttt{.bbl}. This option overwrites this default.
  behaviour.
\begin{lstlisting}[language=sh]
# exbib --outfile=my.out doc.aux
...
\end{lstlisting}

  If the given file name is the value \texttt{-} then the output is
  sent to stdout.
\begin{lstlisting}[language=sh]
# exbib --outfile=- doc.aux
...
\end{lstlisting}

  If the given file name is empty then the output is discarded.
\begin{lstlisting}[language=sh]
# exbib --outfile= doc.aux
...
\end{lstlisting}

\end{description}


\subsubsection{Minimum Cross-References}

\begin{description}
\item[\CLi{M} \Arg{n}]
\item[\CLI{min-crossrefs} \Arg{n}]
\item[\CLI{min.crossrefs} \Arg{n}] \ \\
  This option sets the minimum number of cross-references before an
  entry is not collapsed. The cross referencing in entries provides a
  means for inheriting fields between entries. Usually the entries are
  included in the output.  Nevertheless when there are only a few
  referenes to an entry this entry is collapsed into the referencing
  entries and does not appear alone. This parameter controlls this
  behaviour.
\begin{lstlisting}[language=sh]
# exbib --min-crossrefs=4 doc.aux
...
\end{lstlisting}

\end{description}

\subsubsection{Naming the Program}

\begin{description}
\item[\CLi{p} \Arg{name}]
\item[\CLI{progname} \Arg{name}]
\item[\CLI{program-name} \Arg{name}]
\item[\CLI{program.name} \Arg{name}]\ \\
  This option sets the name of the program. Thus it is possible to
  influence how the program calls itself in logging and error messages
  from outside .
\begin{lstlisting}[language=sh,escapechar=!]
# exbib --progname=BibTeX doc.aux
This is BibTeX, Version !\Version!
...
\end{lstlisting}

\end{description}


\subsubsection{Tracing and Debugging}%
\label{sec:cli.debug}

\begin{description}
\item[\CLi{d} \Arg{mode}]
\item[\CLI{debug} \Arg{mode}]\ \\
  This option turns on the debugging facilities. This means that
  additional information is written to the log file.  This can be done
  selectively. Several aspects can be debugged independently. For this
  purpose the option takes an argument. This argument can be one or
  more of the following items:
  \begin{description}
  \item[all] 
    Debug all of the items below.
  \item[none] 
    Clear all previously entered debugging items.
  \item[search]
    Debug the searching algorithm for resources.
  \item[io] 
    Debug the I/O.
  \item[mem] Debug the memory allocation. This is her for
    compatibility with \BibTeX~8. in ExBib it does nat have a meaning.
  \item[misc] 
    Debug miscellaneous items.
  \item[csf] 
    Debug the processing of CSF files.
  \item[trace] 
    Trace the processing of styles. This is the same as the command
    line switch \texttt{--trace}.
  \end{description}
\begin{lstlisting}[language=sh]
# exbib --debug=search doc.aux
...
\end{lstlisting}
The names can be given in any case combination. Thus the example above
is equivalent to the following invocation:
\begin{lstlisting}[language=sh]
# exbib --debug=SeArCH doc.aux
...
\end{lstlisting}
If several debug items are specified they can be given with several
invocations of the command line argument. This is shown in the
following example:
\begin{lstlisting}[language=sh]
# exbib --debug=search --debug=io doc.aux
...
\end{lstlisting}
The values can also be combined in one use of the parameter by
separating the values by comma (,), semicolon (;), colon (:) or space.
\begin{lstlisting}[language=sh]
# exbib --debug=search,io doc.aux
...
\end{lstlisting}

Note that several command line processors have their own idea of
special characters. Thus in most cases the embedded semicolon or space
has to be escaped properly to arrive intact at \ExBib.

\end{description}

\begin{description}
\item[\CLi{q}]
\item[\CLI{terse}]
\item[\CLI{quiet}]\ \\
  This option switches the operation to quiet mode. Nearly all
  informative messages are suppressed on standard output.
  Nevertheless they can be found in the log file -- if one is written.
\begin{lstlisting}[language=sh]
# exbib --quiet doc.aux
#
\end{lstlisting}

\item[\CLi{t}]
\item[\CLI{trace}]\ \\
  Write a detailed log of internal operations to the log file.  The
  tracing can be very useful when you try to understand the operations
  of the bst interpreter.
  
  Note that this option can drastically decrease the performance of
  operation.
\begin{lstlisting}[language=sh,escapechar=!]
# exbib --trace doc.aux
This is BibTeX, Version !\Version!
...
\end{lstlisting}

\item[\CLi{v}]
\item[\CLI{verbose}]\ \\
  This option switches the operation to verbose mode. Some more
  informative messaged might be presented during the operation.
\begin{lstlisting}[language=sh]
# exbib --verbose doc.aux
#
\end{lstlisting}
\end{description}


\subsubsection{Ignored Options}

\IM{8x} Several command line options have a special meaning in
\BibTeX~8 without a corresponding pendant in \ExBib. Most of them are
related to memory allocation. In \ExBib\ the memory allocation is
fully dynamic and no predefined sizes are necessary.

For compatibility those options are silently ignored:
\begin{description}
\item[\CLi{s}]
\item[\CLI{statistics}]
\item[\CLi{B}]
\item[\CLI{big}]
\item[\CLi{H}]
\item[\CLI{huge}]
\item[\CLi{W}]
\item[\CLI{wolfgang}]
\item[\CLI{mcites}]
\item[\CLI{mentints}]
\item[\CLI{mentstrs}]
\item[\CLI{mfields}]
\item[\CLI{mpool}]
\item[\CLI{mstrings}]
\item[\CLI{mwizfuns}]
\end{description}


\subsection{Abbreviation of Long Parameters}
%@author Gerd Neugebauer

Command line parameters can be abbreviated up to a unique prefix --
and sometimes even more. Thus the following invocations are
equivalent:\index{prefix}

\begin{verbatim}
  exbib --vers
  exbib --versi
  exbib --versio
  exbib --version  
\end{verbatim}


\subsection{Environment Variables}%
\index{environment variables|(}

In addition to passing arguments to the executable upon invocation
there is another mechanism to influence the behaviour of \ExBib.
\ExBib\ recognizes a small set of environment variables and makes use
of their value if they happen to be defined.

\begin{description}
\item [\Env{TEXINPUTS}]\ \\
  This environment variable has a long tradition in \TeX. It contains
  the path to directories where files might be found. The elements in
  this path are separated by a single letter. On Unix this is the
  colon (:) and on Windows this is the semicolon (;).

  This environment variable is transfered to the property
  \Property{texinputs}. 

\item [\Env{EXTEX\_TEXINPUTS}]\ \\
  This environment variable contains the path to directories where
  files might be found. The elements in this path are separated by a
  single letter. On Unix this is the colon (:) and on Windows this is
  the semicolon (;).

  This environment variable is transfered to the property
  \Property{extex.texinputs}. 
\end{description}


\subsubsection{Environment Variables on Unix -- sh and Friends}%
\label{sec:env-sh}

Environment variables are typically stored in the home directory of
any user. Depending on the shell a file is read at startup. \Prog{sh}
and \Prog{bash} use the file \File{.profile} for initialization. Thus
environment variables can be defined there.

The syntax is shown in the following example 

\begin{lstlisting}[language=sh]
export TEXINPUTS='/opt/extex/exbib/texmf'
\end{lstlisting}

\subsubsection{Environment Variables on Unix -- csh and Friends}

Environment variables are typically stored in the home directory of
any user. Depending on the shell a file is read at startup. \Prog{csh}
uses the file \File{.cshrc} for initialization. Thus environment
variables can be defined there.

The syntax is shown in the following example 

\begin{lstlisting}[language=csh]
setenv TEXINPUTS '/opt/extex/exbib/texmf'
\end{lstlisting}

\subsubsection{Environment Variables on WIndows}

Environment variables on Windows are defined via a dialog. In XP the
systems dialog has a tab to view and edit the environment variables.
The systems dialog can be opened by

\begin{itemize}
\item Pressing the two keys \textsf{Windows} and \textsf{Pause/Break}.
\item Selecting \textsf{Properties} in the context menu of ``My Computer''.
\item In control panel clicking \textsf{System}.
\end{itemize}

The tab \textsf{Advanced} provides a button \textsf{Environment
  Variables} which brings up a dialog to view and edit the environment
variables.

If you happen to use Cygwin\index{Cygwin} on Windows the same remarks
as given for \Prog{sh} in section~\ref{sec:env-sh} apply.
\index{environment variables|)}



\endinput
%
% Local Variables: 
% mode: latex
% TeX-master: "../exbib-manual"
% End: 
