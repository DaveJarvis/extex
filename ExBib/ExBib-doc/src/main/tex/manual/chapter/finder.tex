%%*****************************************************************************
%% Copyright (c) 2008 Gerd Neugebauer
%%
%% Permission is granted to copy, distribute and/or modify this document
%% under the terms of the GNU Free Documentation License, Version 1.2
%% or any later version published by the Free Software Foundation;
%% with no Invariant Sections, no Front-Cover Texts, and no Back-Cover Texts.
%%
%%*****************************************************************************
%% $Id:finder.tex 7067 2008-05-18 11:06:56Z gene $
%%*****************************************************************************
%% Author: Gerd Neugebauer
%%-----------------------------------------------------------------------------

\def\LogArg#1{$\langle${\itshape #1}$\rangle$}

\section{Finding Files and Resources}

All inputs needed are called resources. In the simplest case those
resources are plain files. Whenever a resource is needed a search for
it takes place. We will see how the search works and how it can be
influenced.


\subsection{The Configuration File}

\ExBib\ inherits its resource search from \ExTeX\index{ExTeX@\ExTeX}.
This resource search is rather flexible and highly configurable. With
this respect it follows the path defined by kpathsea\index{kpathsea}.

The search for a resource can be configured in a configuration file.
This configuration files utilizes resource finders which have a
configuration section each.

\INCOMPLETE


\subsection{Tracing the Search}

The tracing can be activated with the trace attributes for some
sections or globally for the whole search at runtime. As a consequence
the log file should contain messages describing the search -- where it
looked and what it found.


\subsection{File Search in the Directory Tree}

The simplest search strategy is to look into a list of directories and
maybe try a list of extensions until a suitable file is found. This
strategy is implemented in the class
\texttt{org.extex.resource.FileFinder}.

\subsubsection*{Example}

\begin{lstlisting}[language=XML]
  <Finder class="org.extex.resource.FileFinder"
          default="default">
    <default>
      <path>.</path>
      <path property="extex.texinputs"/>
      <path property="texinputs"/>
      <path env="EXTEX_TEXINPUTS"/>
      <path env="TEXINPUTS"/>
      <extension>.{type}</extension>
      <extension/>
    </default>
  </Finder>
\end{lstlisting}

\subsubsection*{Attributes}
\begin{description}
\item[class] The attribute \texttt{class} has the value
  \texttt{org.extex.resource.FileFinder} for the search in
  archives. It is mandatory.
\item[default] The attribute \texttt{default} names the type shich
  should be used if no special section for the type at hand is
  present. This attribute is mandatory.
\item[trace] The attribute \texttt{trace} can be used to turn on the
  tracing from within the configuration. It takes a boolean value as
  argument (\texttt{true} or \texttt{false}). The tracing can be
  overwritten at runtime. This attribute is optional. The default
  value is \texttt{false}.
\end{description}

\subsubsection*{Nested Elements}

The nested elements are used to select the parameters for the type of
the resource in question. The nested element with the same name as the
type is used. If no such nested element is found then the value of the
\texttt{default} attribute is used to read this nested element
instead. 

The nested element found must contain nested elements with the name
\texttt{extension}. The values of those nested \texttt{extension}
elements are appended to the resource name when it is sought.

The nested element found must contain a nested element named
\texttt{path}. It is treated as follows:
\begin{itemize}
\item If the attribute \texttt{property} is given then the body is
  ignored and the directory read from the named property.
\item If the attribute \texttt{env} is given then the body is ignored
  and the directory read from the named environment variable.
\item Otherwise the body is used as directory.
\end{itemize}

The file is constructed from the path, the resource name, and the
extension. The string \verb|{.type}| in the constructed resource name
is replaced by the currently sought resource. This value can occur in
the extension or the path. The first readable resource wins.

\subsubsection*{Tracing Output}

The following messages might be written to the log:

{\small
\begin{itemize}
\item{\tt FileFinder: Searching \LogArg{name} [\LogArg{type}]}
\item{\tt FileFinder: Trying \LogArg{full name}}
\item{\tt FileFinder: Found \LogArg{full name}}
\item{\tt FileFinder: Not found \LogArg{name}}
\item{\tt FileFinder: Failed for \LogArg{name}}
\item{\tt FileFinder: Property \LogArg{property name} is undefined.}
\item{\tt FileFinder: Environment variable \LogArg{env} is undefined.}
\item{\tt FileFinder: Configuration for `\LogArg{type}' not found;
    Using default `\LogArg{element name}'}
\end{itemize}}

Note that the messages are subject to internationalization. Thus a
translated version can appear for other user languages than English.


\subsection{Network-transparent Search}

The search strategy described here can be configured to search
resources in the network. A URL is utilized for tis purpose. As
fallback s directory search is performed. A list of directories and a
list of extensions are composed until a suitable resource is found.
This strategy is implemented in the class
\texttt{org.extex.resource.UrlFinder}.

\subsubsection*{Example}

\begin{lstlisting}[language=XML]
  <Finder class="org.extex.resource.UrlFinder"
          default="default">
    <default>
      <path>.</path>
      <path property="extex.texinputs"/>
      <path property="texinputs"/>
      <path env="EXTEX_TEXINPUTS"/>
      <path env="TEXINPUTS"/>
      <extension>.{type}</extension>
      <extension/>
    </default>
  </Finder>
\end{lstlisting}

\subsubsection*{Attributes}
\begin{description}
\item[class] The attribute \texttt{class} has the value
  \texttt{org.extex.resource.FileFinder} for the search in
  archives. It is mandatory.
\item[default] The attribute \texttt{default} names the type shich
  should be used if no special section for the type at hand is
  present. This attribute is mandatory.
\item[trace] The attribute \texttt{trace} can be used to turn on the
  tracing from within the configuration. It takes a boolean value as
  argument (\texttt{true} or \texttt{false}). The tracing can be
  overwritten at runtime. This attribute is optional. The default
  value is \texttt{false}.
\end{description}

\subsubsection*{Nested Elements}

The nested elements are used to select the parameters for the type of
the resource in question. The nested element with the same name as the
type is used. If no such nested element is found then the value of the
\texttt{default} attribute is used to read this nested element
instead. 

The nested element found must contain nested elements with the name
\texttt{extension}. The values of those nested \texttt{extension}
elements are appended to the resource name when it is sought.

The nested element found must contain a nested element named
\texttt{path}. It is treated as follows:
\begin{itemize}
\item If the attribute \texttt{property} is given then the body is
  ignored and the directory read from the named property.
\item If the attribute \texttt{env} is given then the body is ignored
  and the directory read from the named environment variable.
\item Otherwise the body is used as directory.
\end{itemize}

The file is constructed from the path, the resource name, and the
extension. The string \verb|{.type}| in the constructed resource name
is replaced by the currently sought resource. This value can occur in
the extension or the path. The first readable resource wins.

\subsubsection*{Tracing Output}

The following messages might be written to the log:

{\small
\begin{itemize}
\item{\tt UrlFinder: Searching \LogArg{name} [\LogArg{type}]}
\item{\tt UrlFinder: Trying file \LogArg{full name}}
\item{\tt UrlFinder: Trying URL \LogArg{full name}}
\item{\tt UrlFinder: Found \LogArg{full name}}
\item{\tt UrlFinder: Not found \LogArg{name}}
\item{\tt UrlFinder: Not a valid URL: \LogArg{full name}}
\item{\tt UrlFinder: Failed for \LogArg{name}}
\item{\tt UrlFinder: Property \LogArg{property name} is undefined.}
\item{\tt UrlFinder: Environment variable \LogArg{env} is undefined.}
\item{\tt UrlFinder: Configuration for `\LogArg{type}' not found;
    Using default `\LogArg{element name}'}
\end{itemize}}

Note that the messages are subject to internationalization. Thus a
translated version can appear for other user languages than English.


\subsection{Searching in texmf Trees}

The search in texmf utilizes the index file \File{ls-R}. If this file
is not found then the directory is ignored. The index file \File{ls-R}
is read in and used to find the resources.

\subsubsection*{Example}

\begin{lstlisting}[language=XML]
  <Finder class="org.extex.resource.LsrFinder"
          trace="false"
          capacity="88888"
          default="default">
    <path property="extex.texinputs"/>
    <path property="texinputs"/>
    <path env="EXTEX_TEXINPUTS"/>
    <path env="TEXINPUTS"/>

    <default>
      <extension>.{type}</extension>
      <extension/>
    </default>
  </Finder>
\end{lstlisting}

\subsubsection*{Attributes}
\begin{description}
\item[class] The attribute \texttt{class} has the value
  \texttt{org.extex.resource.LsrFinder} for the search in indexed
  directories. It is mandatory.
\item[default] The attribute \texttt{default} names the type which
  should be used if no special section for the type at hand is
  present. This attribute is mandatory.
\item[capacity] The attribute \texttt{capacity} gives an initial
  capacity for the size of the hash used to store the index entries
  from the \File{ls-R} files. Its fallback is rather small and the
  performance can be influenced positively by using this attribute
  with a proper value.
\item[trace] The attribute \texttt{trace} can be used to turn on the
  tracing from within the configuration. It takes a boolean value as
  argument (\texttt{true} or \texttt{false}). The tracing can be
  overwritten at runtime. This attribute is optional. The default
  value is \texttt{false}.
\end{description}

\subsubsection*{Nested Elements}

The nested elements are used to select the parameters for the type of
the resource in question. The nested element with the same name as the
type is used. If no such nested element is found then the value of the
\texttt{default} attribute is used to read this nested element
instead. 

The nested element found may contain nested elements with the name
\texttt{extension}. The values of those nested \texttt{extension}
elements are appended to the resource name when it is sought in the
index. The first readable resource wins.

The string \verb|{.type}| in the constructed resource name is replaced
by the currently sought resource. This value can occur in the
extension.

One nested element is treated special:

\begin{description}
\item[path] The nested element \texttt{path} specified where to look
  for the index file \File{ls-R}. There are several cases in which
  this nested element can be used:
  \begin{itemize}
  \item If the attribute \texttt{property} is given then the body is
    ignored and the directory read from the named property.
  \item If the attribute \texttt{env} is given then the body is
    ignored and the directory read from the named environment
    variable.
  \item Otherwise the body is used as directory to search the index
    file \File{ls-R}.
  \end{itemize}
\end{description}

\subsubsection*{Tracing Output}

The following messages might be written to the log:

{\small
\begin{itemize}
\item{\tt LsrFinder: Searching \LogArg{name} [\LogArg{type}]}
\item{\tt LsrFinder: Trying \LogArg{full name}}
\item{\tt LsrFinder: Found \LogArg{full name}}
\item{\tt LsrFinder: Ignoring unreadable file \LogArg{full name}}
\item{\tt LsrFinder: Database \LogArg{index} is not readable}
\item{\tt LsrFinder: Failed for \LogArg{name}}
\item{\tt LsrFinder: Loaded database file `\LogArg{index}' in \LogArg{time} ms, size=\LogArg{size}}
\item{\tt LsrFinder: Property \LogArg{property name} is undefined.}
\item{\tt FileFinder: Environment variable \LogArg{env} is undefined.}
\item{\tt LsrFinder: Configuration for `\LogArg{type}' not found;
    Using default `\LogArg{element name}'}
\item{\tt LsrFinder: Default configuration not found}
\item{\tt LsrFinder: Type `\LogArg{type}' skipped}
\end{itemize}}

Note that the messages are subject to internationalization. Thus a
translated version can appear for other user languages than English.


\subsection{Recursively Searching in Directory Trees}

Resources can be sought somewhere in a directory. This means in the
directory or recursivle in one of its sub-directories. This method
really traverses the directory tree. This is not advisable since it
can be rather slow.

\subsubsection*{Example}

\begin{lstlisting}[language=XML]
  <Finder class="org.extex.resource.FileFinderRPath"
          trace="false"
          default="default">>
    <path property="extex.texinputs"/>
    <path property="texinputs"/>

    <default>
      <extension>.{type}</extension>
      <extension/>
    </default>
  </Finder>
\end{lstlisting}


\subsubsection*{Attributes}
\begin{description}
\item[class] The attribute \texttt{class} has the value
  \texttt{org.extex.resource.FileFinderRPath} for the search in
  archives. It is mandatory.
\item[default] The attribute \texttt{default} names the type shich
  should be used if no special section for the type at hand is
  present. This attribute is mandatory.
\item[trace] The attribute \texttt{trace} can be used to turn on the
  tracing from within the configuration. It takes a boolean value as
  argument (\texttt{true} or \texttt{false}). The tracing can be
  overwritten at runtime. This attribute is optional. The default
  value is \texttt{false}.
\end{description}

\subsubsection*{Nested Elements}

The nested elements are used to select the parameters for the type of
the resource in question. The nested element with the same name as the
type is used. If no such nested element is found then the value of the
\texttt{default} attribute is used to read this nested element
instead. 

The nested element found must contain nested elements with the name
\texttt{extension}. The values of those nested \texttt{extension}
elements are appended to the resource name when it is sought.


\subsubsection*{Tracing Output}

The following messages might be written to the log:

{\small
\begin{itemize}
\item{\tt FileFinderRPath: Searching \LogArg{name} [\LogArg{type}]}
\item{\tt FileFinderRPath: Trying \LogArg{full name}}
\item{\tt FileFinderRPath: Found \LogArg{full name}}
\item{\tt FileFinderRPath: Not found \LogArg{full name}}
\item{\tt FileFinderRPath: Failed for \LogArg{name}}
\item{\tt FileFinderRPath: \LogArg{dir} is no path.}
\item{\tt FileFinderRPath: Property \LogArg{property name} is undefined.}
\item{\tt FileFinderRPath: Environment variable \LogArg{env} is undefined.}
\item{\tt FileFinderRPath: Configuration for `\LogArg{type}' not
    found; Using default `\LogArg{element name}'}
\end{itemize}}

Note that the messages are subject to internationalization. Thus a
translated version can appear for other user languages than English.


\subsection{Searching in Archives}

The preferred way to store resources in \ExTeX\ are archives. Archives
are conpressed files which contain a complete tree of directories and
files. This archive also contains administrative information,
including an index of the contents.

The archive used in \ExTeX\ have the extension \texttt{.jar}. They are
special jar files which in turn are special zip files. Thus you can
use the programs \Prog{jar}, \Prog{unzip}, or \Prog{Winzip} to get a
listing or extract the directory tree from the archive.

\subsubsection*{Example}

\begin{lstlisting}[language=XML]
  <Finder class="org.extex.resource.ClasspathArchiveFinder"
          trace="false"
          default="default">
    <default>
      <extension>.{type}</extension>
    </default>
  </Finder>
\end{lstlisting}

\subsubsection*{Attributes}
\begin{description}
\item[class] The attribute \texttt{class} has the value
  \texttt{org.extex.resource.ClasspathArchiveFinder} for the search in
  archives. It is mandatory.
\item[default] The attribute \texttt{default} names the type shich
  should be used if no special section for the type at hand is
  present. This attribute is mandatory.
\item[trace] The attribute \texttt{trace} can be used to turn on the
  tracing from within the configuration. It takes a boolean value as
  argument (\texttt{true} or \texttt{false}). The tracing can be
  overwritten at runtime. This attribute is optional. The default
  value is \texttt{false}.
\end{description}

\subsubsection*{Nested Elements}

The nested elements are used to select the parameters for the type of
the resource in question. The nested element with the same name as the
type is used. If no such nested element is found then the value of the
\texttt{default} attribute is used to read this nested element
instead. 

The nested element found must contain nested elements with the name
\texttt{extension}. The values of those nested \texttt{extension}
elements are appended to the resource name when it is sought.

The file is constructed from the resource name and the
extension. The string \verb|{.type}| in the constructed resource name
is replaced by the currently sought resource. This value can occur in
the extension or the path. The first readable resource wins.


\subsubsection*{Tracing Output}

The following messages might be written to the log:

{\small
\begin{itemize}
\item{\tt ClasspathArchiveFinder: Searching \LogArg{name} [\LogArg{type}]}
\item{\tt ClasspathArchiveFinder: Trying \LogArg{full name}}
\item{\tt ClasspathArchiveFinder: Found \LogArg{full name}}
\item{\tt ClasspathArchiveFinder: Not found \LogArg{full name}}
\item{\tt ClasspathArchiveFinder: Failed for \LogArg{full name}}
\item{\tt ClasspathArchiveFinder: Property \LogArg{property name} is undefined.}
\item{\tt ClasspathArchiveFinder: Configuration for `\LogArg{type}'
    not found;  Using default `\LogArg{element name}'}
\item{\tt ClasspathArchiveFinder: Default configuration not found}
\item{\tt ClasspathArchiveFinder: Type `\LogArg{type}' skipped}
\item{\tt ClasspathArchiveFinder: Index loaded in \LogArg{time} ms, size=\LogArg{size}}
\item{\tt ClasspathArchiveFinder: Using index `\LogArg{index name}'}
\end{itemize}}

Note that the messages are subject to internationalization. Thus a
translated version can appear for other user languages than English.


\subsection{The Default Search}

The default search specification of \ExBib{} is shown below:

\lstinputlisting[language=XML]{../../../../../ExBib-core/src/main/resources/config/path/exbibFinder.xml}


\endinput
%
% Local Variables: 
% mode: latex
% TeX-master: "../exbib-manual"
% End: 
