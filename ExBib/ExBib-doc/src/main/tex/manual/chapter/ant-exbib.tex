%%*****************************************************************************
%% $Id: ant-exbib.tex,v 0.00 2008/05/25 08:43:35 gene Exp $
%%*****************************************************************************
%% Author: Gerd Neugebauer
%%-----------------------------------------------------------------------------

\section{The \ExBib{} Ant Task}%
\index{Ant|(}

\subsection{Invocation}

\ExBib\ provides an integration into Apache Ant. This allows the
invocation of \ExBib\ from within Ant. For this purpose an Ant task
can be defined as follows.

\begin{lstlisting}[language=XML,morekeywords={taskdef}]
 <taskdef name="ExBib"
          classname="org.extex.exbib.ant.ExBibTask" />
\end{lstlisting}\index{taskdef}

This assumes that the jars from the \ExBib\ distribution are on the
class path. You can extend the class path for this definition with the
classpath attribute as shown below.

\begin{lstlisting}[language=XML,morekeywords={taskdef}]
 <taskdef name="ExBib"
          classname="org.extex.exbib.ant.ExBibTask"
          classpath="classes" />
\end{lstlisting}\index{taskdef}

As a result you have defined the task named \texttt{ExBib}. This task
can be used in arbitrary targets.

\begin{lstlisting}[language=XML,morekeywords={target}]
  <target name="simple"
          description="This is a simple invocation of ExBib." >
    <ExBib file="file.aux"/>
\end{lstlisting}\index{target}

Note that a version of Ant is required. Ant is not not included in the
distribution. Thus it is possible to use the \ExBib\ ant task together
with the existing Ant installation.


\subsection{Options}

The body of the task invocation can contain options. They are treated
as if they where read from a properties file. The names and values are
those which could also be taken from a dot file.

\begin{lstlisting}[language=XML,morekeywords={target}]
  <target name="simple"
          description="This is a simple invocation of ExBib." >
    <ExBib file="file.aux">
      exbib.encoding=UTF-8
    </ExBib>
\end{lstlisting}\index{target}

Note that typos in option names are silently ignored.

\subsection{Parameters}

The invocation of the task can be controlled by several parameters.
Those parameters are given as attributes to the task. 

\begin{description}
\item[file="\Arg{file}"] \ \index{file!Ant task parameter}\\
  The parameter is the name of the \texttt{aux} file to read further
  parameters from. This attribute is mandatory.

\begin{lstlisting}[language=XML,morekeywords={target}]
  <target name="simple"
          description="This is a simple invocation of ExBib." >
    <ExBib file="file.aux"/>
\end{lstlisting}\index{target}

\item[encoding="\Arg{enc}"] \ \index{encoding!Ant task parameter}\\
  This option can be used to specify the encoding for reading files.
  The default is to use the platform default encoding. For possible
  values refer to section~\ref{sec:encodings}.

\begin{lstlisting}[language=XML,morekeywords={target}]
  <target name="simple"
          description="This is a simple invocation of ExBib." >
    <ExBib file="file.aux"
           encoding="UTF-8" />
\end{lstlisting}\index{target}

\item[bibEncoding="\Arg{enc}"] \ \index{bibEncoding!Ant task parameter}\\
  This option can be used to specify the encoding for reading database
  files. The default is to use the same value as the parameter
  \texttt{encoding}.  For possible values refer to
  section~\ref{sec:encodings}.

\begin{lstlisting}[language=XML,morekeywords={target}]
  <target name="simple"
          description="This is a simple invocation of ExBib." >
    <ExBib file="file.aux"
           bibEncoding="UTF-8" />
\end{lstlisting}\index{target}

\item[csf="\Arg{csf}"] \ \\
  The option can be used to specify the CSF which cintains character
  definitions and sorting order specification.

\begin{lstlisting}[language=XML,morekeywords={target}]
  <target name="simple"
          description="This is a simple invocation of ExBib." >
    <ExBib file="file.aux"
           csf="german.csf" />
\end{lstlisting}\index{target}

\item[csfEncoding="\Arg{enc}"] \ \index{csfEncoding!Ant task parameter}\\
  This option contains the encoding for reading cs files. The encoding
  needs to be a valid character set. The fallback is the platform
  default encoding.

\begin{lstlisting}[language=XML,morekeywords={target}]
  <target name="simple"
          description="This is a simple invocation of ExBib." >
    <ExBib file="file.aux"
           csf="german.csf" 
           csfEncoding="UTF-8" />
\end{lstlisting}\index{target}

\item[minCrossrefs="\Arg{number}"] \ \index{minCrossrefs!Ant task parameter}\\
  The parameter is a number. It gives the number of cross references
  before the entries are left alone and not collapsed. The default
  value is 2.

\begin{lstlisting}[language=XML,morekeywords={target}]
  <target name="simple"
          description="This is a simple invocation of ExBib." >
    <ExBib file="file.aux"
           minCrossrefs="3" />
\end{lstlisting}\index{target}

\item[output="\Arg{file}"] \ \index{output!Ant task parameter}\\
  This parameter redirects the output for the default type of
  references to the given file. The default is derived from the name
  of the aux file by using the extension \texttt{.bbl}.

\begin{lstlisting}[language=XML,morekeywords={target}]
  <target name="simple"
          description="This is a simple invocation of ExBib." >
    <ExBib file="file.aux"
           output="file.bbl" />
\end{lstlisting}\index{target}

\item[logfile="\Arg{file}"] \ \index{logfile!Ant task parameter}\\  
  This option can be used to redirect the output to a file. The
  default is to print the informative messages to the console only.

\begin{lstlisting}[language=XML,morekeywords={target}]
  <target name="simple"
          description="This is a simple invocation of ExBib." >
    <ExBib file="file.aux"
           logfile="file.blg" />
\end{lstlisting}\index{target}

\item[config="\Arg{config}"] \ \index{config!Ant task parameter}\\
  This parameter can be used to specify the configuration for
  assembling \ExBib{} (see also section~\ref{sec:cli.cfg}). The
  default value is \texttt{exbib}. The value \texttt{bibtex099} can be
  used to switch to the compatibility mode for
  \BibTeX~0.99c\index{BibTeX 0.99c@\BibTeX~0.99c}.

\begin{lstlisting}[language=XML,morekeywords={target}]
  <target name="simple"
          description="This is a simple invocation of ExBib." >
    <ExBib file="file.aux"
           config="bibtex099" />
\end{lstlisting}\index{target}
  
\item[load="\Arg{file}"] \ \index{load!Ant task parameter}\\
  In contrast to the command line interface no dot files (see
  section~\ref{sec:dot.files}) are read by the Ant task. This
  attribute can be used to load dot files.
  
  The value is the name of the parameter file to load. It can be
  relative to the current directory or absolute.  If the first letter
  is a \verb|~| the this is replaced with the user's home directory.

  This attribute can be given several times to load different dot
  files.

\begin{lstlisting}[language=XML,morekeywords={target}]
  <target name="simple"
          description="This is a simple invocation of ExBib." >
    <ExBib file="file.aux"
           load="~/.exbib"
           load="./.exbib" />
\end{lstlisting}\index{target}

\item[debug="\Arg{flags}"] \ \index{debug!Ant task parameter}\\
  This option can be used to specify debug options (see also
  section~\ref{sec:cli.debug}).

\begin{lstlisting}[language=XML,morekeywords={target}]
  <target name="simple"
          description="This is a simple invocation of ExBib." >
    <ExBib file="file.aux"
           logfile="file.blg"
           debug="trace,search" />
\end{lstlisting}\index{target}

\end{description}

\index{Ant|)}

\endinput
%
% Local Variables: 
% mode: latex
% TeX-master: "../exbib-manual"
% End: 
