%%*****************************************************************************
%% $Id$
%%*****************************************************************************
%% Author: Gerd Neugebauer
%%-----------------------------------------------------------------------------

\chapter{Behind the Scenes}

\section{The \texttt{aux} File}


\IM{08x1} The selection of databases and entries in \ExBib\ is mainly
controlled by the \texttt{aux} file. The aux file is usually produced
by \LaTeX\index{LaTeX@\LaTeX} and read by \ExBib. Since the aux file
is also used to communicate between successive runs of
\LaTeX\index{LaTeX@\LaTeX} it can contain information not relevant for
\ExBib. This irrelevant information is ignored.

Whenever the \texttt{aux} file is read the relevant information is
found by a strict pattern matching. The macros described below will be
recognized only if they are on a line by their own. Any characters
preceding or following them on the same line will lead to the
skipping when the \texttt{aux} file is read.

\subsection{Support for Multiple Bibliographies}

\ExBib\IM{x} contains support for multiple sets of citations. Each
such set is characterized by a own set of databases, a own set of
citations and a own bib style. Thus it is necessary to allow one as
well as several sets in one aux file. This is accomplished by adding
an optional parameter to the specific declaration macros in the aux
file.

The argument in brachets should mainly consist of letters and digit.
In practice nearly all characters are allowed. Nevertheless some of
them might lead to touble somewhere else.

The argument in brackets is taken as the extension of the file to be
produced. It is added to the base name of the aux file separated by a
period. Thus you should not include the period in brackets unless you
want to have it doubled.


\subsection{The Declaration of Databases}

One information required concerns the databases to consult. This is
given with the macro \macro{bibdata}. The argument is a list of file
or resource names separated by comma. Those databases are read in.
\IM{08x1}

\begin{lstlisting}[language={[LaTeX]TeX}]
  \bibdata{the/bibliography}
\end{lstlisting}

Multiple sets of databases are supported by an optional argument to
the macro \macro{bibdata}. This optional argument is enclosed in
brackets. It contains the extension for the output file. The default
is \texttt{bbl}.\IM{x}

The following example is equivalent to the example above:

\begin{lstlisting}[language={[LaTeX]TeX}]
  \bibdata[bbl]{the/bibliography}
\end{lstlisting}


\subsection{The Declaration of Citations}

The aux file can select a subset of entries from the database. This is
accomplished with the macro \macro{citation}. The argument is a comma
separated list of citation keys. If the special value \verb|*| is
encountered then all entries are selected. This value overrules any
other citation keys given.
\IM{08x1}

\begin{lstlisting}[language={[LaTeX]TeX}]
  \citation{a.key,another.key}
\end{lstlisting}

Multiple sets of citations are supported by an optional argument to
the macro \macro{citation}. This optional argument is enclosed in
brackets. It contains the extension for the output file. The default
is \texttt{bbl}.\IM{x}

The following example is equivalent to the example above:

\begin{lstlisting}[language={[LaTeX]TeX}]
  \citation[bbl]{a.key,another.key}
\end{lstlisting}


\subsection{The Declaration of the Style}

The aux file must declare the bst to be used. This can be done with
the macro \macro{bibstyle}. The argument is a file or resource to be
used as bst file.\IM{08x1}

\begin{lstlisting}[language={[LaTeX]TeX}]
  \bibstyle{the/style}
\end{lstlisting}

Multiple sets of citations are supported by an optional argument to
the macro \macro{bibstyle}. This optional argument is enclosed in
brackets. It contains the extension for the output file. The default
is \texttt{bbl}.\IM{x}

The following example is equivalent to the example above:

\begin{lstlisting}[language={[LaTeX]TeX}]
  \bibstyle[bbl]{the/style}
\end{lstlisting}

\subsection{The Declaration of Includes}

The aux file can reference to other aux files containing supplementary
information. Those references are given  with the macro
\macro{@include}. The argument is another aux file to be consulted.
This aux file can in turn contain agaion references to other aux
files. All those references are resolved recursively.\IM{08x1}

\begin{lstlisting}[language={[LaTeX]TeX},%
  alsoletter={@},morekeywords={@include}]
  \@include{the/included/file}
\end{lstlisting}


\endinput
%
% Local Variables: 
% mode: latex
% TeX-master: "../exbib-manual"
% End: 
