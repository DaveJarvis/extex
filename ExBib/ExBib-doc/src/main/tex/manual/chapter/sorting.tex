%%*****************************************************************************
%% Copyright (c) 2008 Gerd Neugebauer
%%
%% Permission is granted to copy, distribute and/or modify this document
%% under the terms of the GNU Free Documentation License, Version 1.2
%% or any later version published by the Free Software Foundation;
%% with no Invariant Sections, no Front-Cover Texts, and no Back-Cover Texts.
%%
%%*****************************************************************************
%% $Id: sorting.tex,v 0.00 2008/05/28 22:47:23 gene Exp $
%%*****************************************************************************
%% Author: Gerd Neugebauer
%%-----------------------------------------------------------------------------

\chapter{Sort Order}%
\label{sec:sort.order}

Sorting is an important feature in the bibliography. Thus several
mechanisms are provided to determine the sorting order od entries.
the preferred way to specify the sort order is in the aux file with
the help of the macro \verb|\biboptions|. Alternatively the sort order
can be specified via the command line.

In any case the specification is a string naming the sorting method
and optional parameters.


\section{Reverse Ordering}

Whatever sorting order is given the need arises to describe an
ascending and descending variant of it. The default is ascending
order. The descending order can be achieved by reversing the given
order.

The reversing can be achieved wir the keyword \texttt{reverse}
followed by the a sort order specification after a colon. This is
shown in the following example.

\begin{verbatim}
  sort=reverse:unicode
\end{verbatim}

\section{Unicode Ordering}

The Unicode orderings assume a certain semantics for the code points.
This semantics is given by the Unicode specification.

\subsection{Unicode}

The sort order \texttt{unicode} sorts the entries in ascending order
according to the lexicographical ordering of their sort keys.

\begin{verbatim}
  sort=unicode
\end{verbatim}

\subsection{ignoreCase}

The sort order \texttt{ignoreCase} sorts the entries in ascending order
according to the lexicographical ordering of their sort keys
translated to their lowercase counterpart. This translation is
possible since the meaning of the characters are known from the
Unicode tables.

\begin{verbatim}
  sort=ignoreCase
\end{verbatim}

\section{Locale Based Ordering}

The sort order \texttt{locale} sorts the entries in ascending order
according to the lexicographical ordering of their sort keys
normalized by the locale given as argument.

\begin{verbatim}
  sort=locale:de
\end{verbatim}

\section{Rule Based Collators}

\begin{verbatim}
  sort=rbc:resource
\end{verbatim}

\INCOMPLETE

%%*****************************************************************************
%% Copyright (c) 2008-2010 Gerd Neugebauer
%%
%% Permission is granted to copy, distribute and/or modify this document
%% under the terms of the GNU Free Documentation License, Version 1.2
%% or any later version published by the Free Software Foundation;
%% with no Invariant Sections, no Front-Cover Texts, and no Back-Cover Texts.
%%
%%*****************************************************************************
%% $Id:csf.tex 7067 2008-05-18 11:06:56Z gene $
%%*****************************************************************************
%% Author: Gerd Neugebauer
%%-----------------------------------------------------------------------------

\section{CSF}%
\label{sec:csf}%
\index{csf|(}

One problem of the original \BibTeX~0.99c is the restriction to 7 bit
characters. To solve this problem the cs files have been introduced in
\BibTeX~8\index{BibTeX 8|\BibTeX 8}:
\IM{8x}

\begin{verbatim}
  sort=csf
\end{verbatim}

\begin{verbatim}
  sort=csf:resource
\end{verbatim}

\INCOMPLETE

\index{csf|)}

\endinput
%
% Local Variables: 
% mode: latex
% TeX-master: "../exbib-manual"
% End: 



\endinput
%
% Local Variables: 
% mode: latex
% TeX-master: nil
% TeX-master: "exbib-manual"
% End: 
