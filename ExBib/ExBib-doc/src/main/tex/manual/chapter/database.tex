%%*****************************************************************************
%% $Id:database.tex 7067 2008-05-18 11:06:56Z gene $
%%*****************************************************************************
%% Author: Gerd Neugebauer
%%-----------------------------------------------------------------------------

\chapter{The Data Base}


\section{Syntax}

The data base in \BibTeX\index{BibTeX@\BibTeX} style consists of a
simple text file. \IM{08x1}

The following characters have a special meaning for the \BibTeX\
syntax:
\begin{verbatim}
    @ { } ( ) , # " =
\end{verbatim}
Any other character is treated equally as ordinary character.

An instruction is started with an at sign (@) followed by its name.
The name is composed of upper or lowercase letters and digits.

Whatever follows the name of the instruction depends on the
instruction. In most cases the parameters for the instruction are
following. They are enclosed in braces. For compatibility with
Scribe\index{Scribe} parentheses are sometimes allowed instead of the
braces.


\section{The \texttt{@input} Instruction}%
\index{@input|(}

Sometimes it might be desirable to split a database into several
segements. This is supported by the ability to pass inseveral
databases via the aux file. The \texttt{@input} instruction provides
another mechanism for the same which acts on the level of the database
files. \IM{x1}

The instruction takes as argument a resource name. It includes the
content as if it where present at the place of the instruction.

\begin{lstlisting}[language=bibtex,alsoletter={@},morekeywords={@input}]
  @input(some/other/resource)
\end{lstlisting}

The formal syntax of this instruction is as follows:
\begin{syntax}
  <instruction> \+ |@input| | \char`\{\ | <resource name> | \char`\} |
\end{syntax}%
The instruction name \texttt{input} is taken case insensitive.
\index{@input|)}


\section{Entries}

Any instruction which has no special meaning is considered to be an
entry in the database.
\IM{08x1}

\begin{lstlisting}[language=bibtex]
  @Book{          knuth:texbook,
    author      = {Donald E. Knuth},
    title       = {The {\TeX book}},
    publisher   = {Addison-Wesley Publishing Company},
    address     = {Reading, Mass.},
    year        = 1989,
    edition     = {15},
    volume      = {A},
    series      = {Computers and Typesetting}
  }
\end{lstlisting}

\INCOMPLETE

The formal syntax of this instruction is as follows:
\begin{syntax}
  <instruction> \+ |@| <type> | \char`\{\ | <key> | , | <attributes> | \char`\} | \\
                \| |@| <type> | ( | <key> | , | <attributes> | ) |\\
  <attributes>  \= <attribute>\\
                \| <attribute> | , | <attributes>\\
  <attribute>   \= <field name> | = | <value>\\
  <value>       \= <atomic value>\\
                \| <atomic value> | \# | <value>\\
  <atomic value>\= |"| <string> |"|\\
                \| |\char`\{| <block> |\char`\}| \\
                \| <number>\\
                \| <macro name>
\end{syntax}%
Several syntactic entities deserve a definition:

\begin{description}
\item[\Tag{type}] \ \\
  A \Tag{type} is a non-empty sequence of characters which is not one
  of the special characters or white-space. The type is considered
  case insensitive; i.e. upper and lowercase letters are considered
  the same.
\item[\Tag{key}] \ \\
  A \Tag{key} is a non-empty sequence of characters which is not one
  of the special characters or white-space. The type is considered
  case insensitive.
\item[\Tag{field name}] \ \\
  A \Tag{field name} is a non-empty sequence of characters which is
  not one of the special characters or white-space. The type is
  considered case insensitive.
\item[\Tag{string}] \ \\
  A \Tag{string} is a sequence of characters with balenced braces
  which does not contain a double quote \verb|"| at brace level 0.
\item[\Tag{block}] \ \\
  A \Tag{block} is a sequence of characters with balanced braces; The
  braces preceded by a backslash do not count for balancing.
\item[\Tag{number}] \ \\
  A \Tag{number} is a not empty sequence of digits.
\item[\Tag{macro name}] \ \\
  A \Tag{macro name} is a non-empty sequence of characters which is
  not one of the special characters or white-space. The type is
  considered case insensitive.
\end{description}


\section{Names}\label{sec:names}%
\index{name|(}

Names are especially complicated and deserve a description of their
own. 
\IM{08x1}

\subsection{Name Components}

\BibTeX\ uses four components for names. Any name is analyzed and
decomposed into the four parts. The following parts of names are
considered:

\begin{description}
\item[Last part] \ \\
  The last name or christian name of a person is usually the last
  major component of a name. This is the only part which is not
  optional.
\item[First parts] \ \\
  The first name or given name of a person is usually the first
  component of a name. This part is optional.
\item[Von part] \ \\
  The von part of a name usually comes between first name and last
  name and starts with lowercase letters. It is optional.
\item[Junior part] \ \\
  The junior part of a name is an addition appended to the name. This
  part is optional.
\end{description}

The parts are separated by white space. This whitespace is only
consoidered if it occurs at brace level 0. Thus the grouping with
braces is honored. It can be used to tie together parts which would be
torn apart otherwise.

Any part can consist of several words. Commas can be used to structure
a name. Thus we will use the number of commas for the analysis as
well.
\def\First#1{#1}%
\def\Last#1{\textbf{#1}}%
\def\Von#1{\textit{#1}}%
\def\Jr#1{\underline{#1}}%

\subsubsection*{No Commas}

If a name does not contain a comma then the following pattern is used
to determine the parts of the name:
\begin{syntax}
  <name> \+ <first>* <von>* <last> <jr>* 
\end{syntax}%
The name parts \emph{first} and \emph{last} consit of words for which
the first letter is an uppercase letter. The name parts \emph{von} and
\emph{jr} consit of words for which the first letter is a lowercase
letter.

The following examples show how the names are classified. The last
name is printed in bold, the first name is printed in roman, the von
part is printed in italics, and the junior part is underlined.

\begin{quote}\obeylines
  \Last{Aristotle}
  \First{Leslie} \Last{Lamport}
  \First{Donald Ervin} \Last{Knuth}
  \First{Johannes Chrysostomus Wolfgangus Theophilus} \Last{Mozart}
  \First{Ludwig} \Von{van} \Last{Beethoven}
  \First{Otto Eduard Leopold} \Von{von} \Last{Bismarck-Sch�nhausen}
  \First{Miguel} \Von{de} \Last{Cervantes Saavedra}
  \First{Sammy} \Last{Davis} \Jr{jr.}
  \First{Don Quixote} \Von{de la} \Last{Mancha}
\end{quote}


\subsubsection*{One Comma}

If a name does not contain a comma then the following pattern is used
to determine the parts of the name:
\begin{syntax}
  <name> \+ <von>* <last> | , | <first>* \\
         \|  <first>* <von>* <last> | , | <jr>* 
\end{syntax}%
The name parts \emph{first} and \emph{last} consit of words for which
the first letter is an uppercase letter. The name parts \emph{von} and
\emph{jr} consit of words for which the first letter is a lowercase
letter.

The following examples show how the names are classified. The last
name is printed in bold, the first name is printed in roman, the von
part is printed in italics, and the junior part is underlined.

\begin{quote}\obeylines
  \Last{Lamport}, \First{Leslie}
  \Last{Knuth}, \First{Donald Ervin} 
  \Last{Mozart}, \First{Johannes Chrysostomus Wolfgangus Theophilus} 
  \Last{Beethoven}, \First{Ludwig} \Von{van} 
  \Von{von} \Last{Bismarck-Sch�nhausen}, \First{Otto Eduard Leopold} 
  \Von{de} \Last{Cervantes Saavedra}, \First{Miguel} 
  \First{Sammy} \Last{Davis}, \Jr{Jr.}
\end{quote}


\subsubsection*{Two Commas}

If a name does not contain a comma then the following pattern is used
to determine the parts of the name:

\begin{syntax}
  <name> \+ <last> <von>* | , | <jr> | , | <first>* 
\end{syntax}%
The name parts \emph{first} and \emph{last} consit of words for which
the first letter is an uppercase letter. The name parts \emph{von} and
\emph{jr} consit of words for which the first letter is a lowercase
letter.

The following examples show how the names are classified. The last
name is printed in bold, the first name is printed in roman, the von
part is printed in italics, and the junior part is underlined.

\begin{quote}\obeylines
  \Last{Davis}, \First{Sammy}, \Jr{Jr.}
\end{quote}


\subsubsection*{More Commas}

More than two commas are not understood by the name parsing in \ExBib.


\subsection{Name Lists}
\index{name list|(}

Names usually come in bibliographies as single names or as lists of
names. Thus we have to take care of lists of names. Those lists are
made up of single names separated by the word \texttt{and} surrounded
by whitespace. This separator is only considered at brace level 0.
Thus it is possible to protect embedded words ``and''. Those might be
parts of company names -- e.g. acting an author.

The following example is recognizes as two names:
\begin{quote}\obeylines
  \Last{Barnes} \texttt{and} \Last{Noble}
\end{quote}
The following example is recognized as a single name name:
\begin{quote}\obeylines
  \Last{\char`\{ Barnes and Noble\char`\}}
\end{quote}

The formal syntax is as follows:
\begin{syntax}
  <namelist> \= <name>\\
             \| <name> \verb*/ and / <namelist> 
\end{syntax}%

\index{name list|)}%
\index{name|)}


\section{Comments}%
\index{@comment|(}

Anything outside of entries and other declarations are considered as a
comment -- and mainly ignored. Thus you can put anything in between
the entries.
\IM{081}

There is one special tag to mark comments. It is the tag
\texttt{@comment}. Since anything outside of declarations is already a
comment. Is has been considered sufficient to ignore the tag in the
input stream.
\begin{lstlisting}[language=bibtex]
  @comment
\end{lstlisting}

Unfortunately in the age of internet it is desirable to include email
addresses into comment -- and those may contain an @. Thus the
definition of the \texttt{@comment} declaration is slightly different
from \BibTeX\index{BibTeX@\BibTeX}:
\IM{x}
\begin{itemize}
\item If the next non-space character is an opening brace (\verb|{|)
  then a block is read and treated as comment. This means that the
  block can contain arbitrary characters -- especially the @ sign.
  On the other side the block needs to have balanced braces.
\begin{lstlisting}[language=bibtex]
  @comment{ This is a comment with embedded @ }
\end{lstlisting}
    
\item If the next non-space character is not an open brace character
  then just the tag is ignored.
\begin{lstlisting}[language=bibtex]
  @comment This is a comment
\end{lstlisting}
 
\end{itemize}

Formally spoken those two cases correspond to the following syntax
specification:
\begin{syntax}
  <instruction> \+ |@comment| |\char`\{| <block> |\char`\}| \\
                \| |@comment| 
\end{syntax}
The instruction name is compared case insensitive.
\index{@comment|)}


\section{The \texttt{@alias} Instruction}%
\index{@alias|(}

The \texttt{@alias} instruction can be used to define alternative keys
for an entry. Whenever an entry is references with one of its keys all
aliases are considered to be the same. For instance when an entry has
two keys defined via the \texttt{@alias} and both of them are uned in
a \macro{cite} then it is included in the result list only once.
\IM{x1}

\begin{lstlisting}[language=bibtex]
  @alias( abc = def )
\end{lstlisting}

To define an alias give the new key followed by an equality sign and
the old key as argument to the \texttt{@alias} instruction. The enty
for the old key has to exist at the time the \texttt{@alias}
instruction is encountered.

The formal syntax of this instruction is as follows:
\begin{syntax}
  <instruction> \+ |@alias| | \char`\{\ | \tag{key} | = | \tag{key} | \char`\} | \\
                \| |@alias| | ( | \tag{key} | = | \tag{key} | ) | 
\end{syntax}%
The instruction name is compared case insensitive.
\index{@alias|)}

\section{The \texttt{@modify} Instruction}%
\index{@modify|(}

The \texttt{@modify} instruction can be used to alter the content of
certain fields in an entry. The question is why would such a directive
be necessary when you could simply alter the entry itself. The answer
is that the entry might not be under your control. It might be
contained in another file which is included via the \texttt{@include}
directive.
\IM{x1}

\begin{lstlisting}[language=bibtex]
  @modify( abc,
           author = {A.U. Thor} )
\end{lstlisting}

The argument of the \texttt{@modify} instruction looks formally like
the argument for an entry. Instead of specifying a new entry an
existing entry with the given key is sought and the given field values
are overwritten. If no entry corresponding to the given key can be
found then an error is raised.


The formal syntax of this instruction is as follows:
\begin{syntax}
  <instruction> \+ |@modify| | \char`\{\ | <key> | , | <attributes> | \char`\} | \\
                \| |@modify| | ( | <key> | , | <attributes> | ) | 
\end{syntax}%
The instruction name is compared case insensitive.
\index{@modify|)}


\section{The \texttt{@string} Instruction}%
\index{@string|(}

The \texttt{@string} instruction can be used to define macros. Each
macro has a name and a value. When a field of an entry is accessed the
field value is computed. The computation concatenats its parts. In
this course the macros contained are resolved.
\IM{08x1}

The macros can be defined in several places. One of them is the
datatebase file. Another is the style file. The place where the
definition is located does not make a difference.

\begin{lstlisting}[language=bibtex]
  @string( abc = {The value} )
\end{lstlisting}

The formal syntax of this instruction is as follows:
\begin{syntax}
  <instruction> \+ |@string| | \char`\{\ | <macro name> | = | <value> | \char`\} | \\
                \| |@string| | ( | <macro name> | = | <value> | ) | 
\end{syntax}%
The instruction name is compared case in-sensitove.
\index{@string|)}


\section{The \texttt{@preamble} Instruction}
\index{@preamble|(}

Sometines it is necessary ensure that a certain definition is made
which is used in the database. Those definitions are collected and can
be included at the beginning of the generated output.
\IM{08x1}

\begin{lstlisting}[language=bibtex]
  @preamble( "\providecommand\BibTeX{\textsc{Bib}\TeX}" )
\end{lstlisting}

The formal syntax of this instruction is as follows:
\begin{syntax}
  <instruction> \+ |@preamble| | \char`\{\ | <value> | \char`\} | \\
                \| |@preamble| | ( | <value> | ) | 
\end{syntax}%
The instruction name is compared case in-sensitove.
\index{@preamble|(}

\endinput
%
% Local Variables: 
% mode: latex
% TeX-master: "../exbib-manual"
% End: 
