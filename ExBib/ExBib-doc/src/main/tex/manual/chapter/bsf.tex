%%*****************************************************************************
%% Copyright (c) 2008-2010 Gerd Neugebauer
%%
%% Permission is granted to copy, distribute and/or modify this document
%% under the terms of the GNU Free Documentation License, Version 1.2
%% or any later version published by the Free Software Foundation;
%% with no Invariant Sections, no Front-Cover Texts, and no Back-Cover Texts.
%%
%%*****************************************************************************
%% $Id$
%%*****************************************************************************
%% Author: Gerd Neugebauer
%%-----------------------------------------------------------------------------

\newenvironment{methods}{\begin{description}\def\method##1{\item[##1]\ \\}%
}{\end{description}}

\section{Groovy as Style Language}

An experimental feature of \ExBib{} is the replanceemnt of the BST
language by some other language for writing styles. Here the Bean
Scripting Framework (BSF) is used as adapter for other languages. Any
language made available this way can be plugged in with minor effort
as style language.

As an example the language Groovy has been made available this way.


\subsection{Invoking the Groovy-enhanced \ExBib}

The property \Property{exbib.processor} determines which processor is
used. The value of this property must be set to th value \verb|groovy|
to activate the Groovy entension language. In the simplest case this
is done on the command line as shown in the following example:

\begin{lstlisting}[language=sh]
# exbib -Dexbib.processor=groovy myDocument.aux
\end{lstlisting}

If you want to activate this property for any invocation of \ExBib\ in
this directory then you can alternatively create a file named
\File{.exbib} in the current directory. This file should contain the
following line:

\begin{lstlisting}
exbib.processor=groovy
\end{lstlisting}

Then the usual invocation uses the Groovy extension instead of bst
style files:

\begin{lstlisting}[language=sh]
# exbib myDocument.aux
\end{lstlisting}

See also section \ref{sec:dot.files} for details.


\subsection{Specifying a Groovy Style}

A Groovy style file can be specified in the same way a bst style file
is given. In \LaTeX\ the command \Macro{bibliographystlye} writes the
name of the style file is written to the aux file. The name of the
style file is given without any extension -- like \verb|.gy| or
\verb|.groovy| for Groovy files.


\subsection{Accessing the Database in a Groovy Style}

The databases specified in the aux file are automaticaly read at
startup. The code contained in the Groovy style is executed after the
database has been read in. The database can be accessed via the global
variable \Var{bibDB}. This variable contains an object which
implements the interface \texttt{org.extex.exbib.core.db.DB}. The full
power of this Java interface is described in the Javadoc.


\subsection{The Output}

The global variable \Var{bibWriter} contains an instance of a class
implementing the interface \texttt{org.extex.exbib.core.io.Writer}.
This writer can be used to write to the output file.

It has a method \Method{print} to print some strings:

\begin{lstlisting}[language=Java]
  bibWriter.print("{\em ", series, "\/}" )
\end{lstlisting}

It has a method \Method{println} to print some strings followed by a
newline character:

\begin{lstlisting}[language=Java]
  bibWriter.println("\newcommand{\etalchar}[1]{$^{#1}$}" )
\end{lstlisting}


\subsection{The Processor Context}

The processor context is defined in the interface \texttt{Processor}.
The processor can be accesed through the global variable
\Var{bibProcessor}. This gives access to additional features like the
log file.

\begin{methods}

  \method{String getCite(String key)}\MethodIndex{getCite}%
    Get the original cite key for a given key. I.e. the casing might be
    different.

  \method{DB getDB()}\MethodIndex{getDB}%
    Getter for the database.

  \method{List<String> getEntries()}\MethodIndex{getEntries}%
    Getter for the entries.

  \method{Logger getLogger()}\MethodIndex{getLogger}%
    Getter for the logger.

  \method{String getMacro(String name)}\MethodIndex{getMacro}%
    Getter for a certain macro. If the requested macro is not defined
    \texttt{null} is returned.

  \method{List<String> getMacroNames()}\MethodIndex{getMacroNames}%
    Getter for the list of all defined macro names.

  \method{long getNumberOfWarnings()}\MethodIndex{getNumberOfWarnings}%
    Getter for the number of warnings issued.

  \method{Writer getOutWriter()}\MethodIndex{getOutWriter}%
    Getter for the output writer. This is the same value as the one provided in
    the variable \Var{bibWriter}.

  \method{boolean isKnown(String type)}\MethodIndex{isKnown}%
    Check if the entry type is known.

  \method{void warning(String message)}\MethodIndex{warning}%
    This method can be used to issue a warning message.

\end{methods}


\subsection{Methods of the DB}

The important methods of the DB provide reading access to the stored
information. The DB is defined in the interface \texttt{org.extex.exbib.core.db.DB}.
The important methods are described below.

\begin{methods}
  \method{Entry getEntry(String key}\MethodIndex{getEntry}%
    Get a single entry stored under the given reference key. The key
    is normalized before the entry is sought, i.e. the search is case
    insensitive.
    
    If no record is stored under the given key then <code>null</code>
    is returned.

  \method{String getExpandedMacro(String key)}\MethodIndex{getExpandedMacro} %
    Compute the expanded representation of a macro. The expansion is
    performed by recursively replacing macro names by the definitions
    and concatenating the resulting list of values.
    
    For any macro which is not defined during the expansion the empty
    string is used instead.
    
    If the named macro the expansion is started with does not exist
    then \texttt{null} is returned.

  \method{Value getMacro(String name)}\MethodIndex{getMacro}%
    Retrieve the definition of a macro.

  \method{List<String> getMacroNames()}\MethodIndex{getMacroNames}%
    Retrieve the list of macro names.

  \method{int getMinCrossrefs()}\MethodIndex{getMinCrossrefs}%
    Getter for minCrossrefs.
    
    The parameter minCrossrefs determines when an entry which is
    referenced by several entries should be collated into the
    referencing entries. For example this can be the case for articles
    in a collection. Here the collection is shown as separate entry
    only if at least minCrossref entries in the result reference it.

  \method{Value getPreamble()}\MethodIndex{getPreamble}%
    Getter for the preamble.

  \method{String getPreambleExpanded()}\MethodIndex{getPreambleExpanded}%
    Getter for the preamble with all constituents expanded and combined into
    one string.

  \method{void setMinCrossrefs(int minCrossref)}\MethodIndex{setMinCrossrefs}%
    Setter for minCrossrefs.

  \method{void sort() throws ConfigurationException}\MethodIndex{sort}%
    Sort the entries according to the configured sorter.
    Initially the entries are in the order in which they are read in.

\end{methods}

\subsection{Methods of the Entry}

The entry is define din the class \texttt{org.extex.exbib.core.db.Entry}. 
The important methods are describe below.

\begin{methods}
  \method{Value get(String key, DB db) throws ExBibException}\MethodIndex{get}%
    Getter for the Value stored under a given key in the Entry. If the
    value is not found locally and the field crossref is present then
    the value is requested from the entry stored under the crossref key
    in the database.

  \method{String getExpanded(String key, DB db) throws ExBibException}\MethodIndex{getExpanded}%
    This method searches for a normal value and concatenates all
    expanded constituents. Macros are looked up in the database given and
    their values are inserted.

  \method{String getKey()}\MethodIndex{getKey}%
    Getter for the reference key for this Entry.

  \method{List<String> getKeys()}\MethodIndex{getKeys}%
    Getter for the list of keys for "normal" values. Note that those keys may
    lead to \texttt{null} values..

  \method{ValueItem getLocal(String key)}\MethodIndex{getLocal}%
    Getter for a local value. The local values are stored independently from
    the normal values. This means that they have a name-space of their own.

  \method{int getLocalInt(String key)}\MethodIndex{getLocalInt}%
    Getter for a local value. The local values are stored independently from
    the normal values. This means that they have a name-space of their own.

  \method{String getLocalString(String key)}\MethodIndex{getLocalString}%
    Getter for a local value. The local values are stored independently from
    the normal values. This means that they have a name-space of their own.

  \method{Locator getLocator()}\MethodIndex{geLocatort}%
    Getter for the locator for this Entry. The locator allows you to
    determine where this Entry is coming from.

  \method{String getSortKey()}\MethodIndex{getSortKey}%
    Getter for the sort key of this Entry. The sort key is stored as local
    value under the key \texttt{sort.key\$}.

  \method{String getType()}\MethodIndex{getType}%
    Getter for the type of this Entry.

  \method{Iterator<String> iterator()}\MethodIndex{iterator}%
    Getter for an iterator over all keys of normal values.

  \method{void set(String key, String value)}\MethodIndex{set}%
    Setter for the Value stored under a given key in the Entry. The String
    valued parameter is wrapped into a  Value before it is stored.

  \method{void set(String key, Value value)}\MethodIndex{set}%
    Setter for the Value stored under a given key in the Entry. An already
    existing value is overwritten silently.

  \method{setKey(String key)}\MethodIndex{setKey}%
    Setter for the reference key for this Entry.

  \method{void setLocal(String key, int value)}\MethodIndex{setLocal}%
     Setter for a local value. The local values are stored independently from
     the normal values. This means that they have a namespace of their own.

    This method wraps its int argument into a VNumber before it is stored.

  \method{void setLocal(String key, String value)}\MethodIndex{setLocal}%
     Setter for a local value. The local values are stored independently from
     the normal values. This means that they have a namespace of their own.

     This method wraps its int argument into a VString before it is stored.

  \method{void setLocal(String key, ValueItem value)}\MethodIndex{setLocal}%
    Setter for a local value. The local values are stored independently
    from the normal values. This means that they have a name-space of
    their own.

  \method{void setSortKey(String sortKey)}\MethodIndex{setSortKey}%
    Setter for the sort key of this Entry. The sort key is stored as
    local value under the key \texttt{sort.key\$}.

  \method{void setType(String t)}\MethodIndex{setType}%
    Setter for the type of this Entry.

  \method{String toString()}%
\end{methods}

\subsection{Methods of the Value}

The values are used as the right hand side of a entry item. It is a list of
\texttt{ValueItem}s.

\begin{methods}
  \method{void add(Value item)}\MethodIndex{add}%
    Add all value items contained in a value to the current value.

  \method{void add(ValueItem item)}\MethodIndex{add}%
    Add a new value item to the end of the value.

  \method{String expand(DB db)}\MethodIndex{expand}%
    Expand the whole value into a single String. All macros are replaced by
    their values and delimiting braces or quotes are omitted.

  \method{boolean isEmpty()}\MethodIndex{isEmpty}%
    Tests whether the value is empty.

  \method{Iterator<ValueItem> iterator()}\MethodIndex{iterator}%
    Getter for an Iterator for all elements.

\end{methods}


\subsection{Methods of the ValueItem}

A ValueItem is an object stored as a part of a \texttt{Value}. It is an
interface for which the implementations \texttt{VBlock}, \texttt{VMacro},
\texttt{VNumber}, and \texttt{VString} are present. 

\begin{methods}

  \method{void expand(StringBuilder sb, DB db)}\MethodIndex{expand}%
    This method expands the ValueItem and appends the expansion to the given
    StringBuilder. Macros are looked up in the database given and their
    values are inserted.

  \method{String getContent()}\MethodIndex{getContent}%
    The getter for the content.

  \method{void setContent(String value)}\MethodIndex{setContent}%
    Setter for the content.

\end{methods}


\subsection{Methods of the Locator}

The locator is a means to determine where something is coming from.

\begin{methods}

  \method{int getLineNumber()}\MethodIndex{getLineNumber}%
    Getter for the line number.

  \method{int getLinePointer()}\MethodIndex{getLinePointer}%
    Getter for the position in the line.

  \method{String getResourceName()}\MethodIndex{getResourceName}%
    Getter for the resource name. The resource can be a file or the content of
    a URL or a member of an archive or something else. It is a general concept
    whch can be used as needed.

\end{methods}


\subsection{Accessing \BibTeX\ Functions}

Sometimes it is desirable to have access to the original or enhanced
function of the \BibTeX\ style language. There are two ways to access
them. The complex functions -- which are likely to be needed this way
-- are provided as static methods of implementing classes. All
functions can be accesed via implementing function objects.

The implementation classes described below can be found in a Java
package This packege is \texttt{org.extex.exbib.core.bst.code.impl}.
It must be used for the imports of the classes or for direct refernce.

\begin{methods}

  \method{String ChangeCase.changeCase(String type, String string)}\MethodIndex{changeCase}%
  Apply the function \texttt{change.case\$}\fctIndex{change.case\$}.

  \method{String FormatName.formatName(String names, int index, String format)}\MethodIndex{formatName}%
  Apply the function \texttt{format.name\$}\fctIndex{format.name\$}.

  \method{String NumNames.numNames(String names)}\MethodIndex{numNames}%
  Apply the function \texttt{num.names\$}\fctIndex{num.names\$}.

  \method{String Purify.purify(String string)}\MethodIndex{purify}%
  Apply the function \texttt{purify\$}\fctIndex{purify\$}.

  \method{String Substring.substring(String string, int startIndex, int endIndex)}\MethodIndex{substring}%
  Apply the function \texttt{substring\$}\fctIndex{substring\$}.

  \method{int TextLength.textLength(String string)}\MethodIndex{textLength}%
  Apply the function \texttt{text.length\$}\fctIndex{text.length\$}.

  \method{int Width.width(String string)}\MethodIndex{width}%
  Apply the function \texttt{width\$}\fctIndex{width\$}.

\end{methods}
\endinput%%%%%%%%%%%%%%%%%%%%%%%%%%%%%%%%%%%%%%%%%%%%%%%%%%%%%%%%%%%%%%%%%%%%%%
%
% Local Variables: 
% mode: latex 
% TeX-master: "../exbib-manual"
% End:

