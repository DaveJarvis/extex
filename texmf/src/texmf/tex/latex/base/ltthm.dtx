% \iffalse meta-comment
%
% Copyright 1993 1994 1995 1996 1997 1998 1999 2000 2001 2002 2003 2004 2005 2006
% The LaTeX3 Project and any individual authors listed elsewhere
% in this file. 
% 
% This file is part of the LaTeX base system.
% -------------------------------------------
% 
% It may be distributed and/or modified under the
% conditions of the LaTeX Project Public License, either version 1.3c
% of this license or (at your option) any later version.
% The latest version of this license is in
%    http://www.latex-project.org/lppl.txt
% and version 1.3c or later is part of all distributions of LaTeX 
% version 2005/12/01 or later.
% 
% This file has the LPPL maintenance status "maintained".
% 
% The list of all files belonging to the LaTeX base distribution is
% given in the file `manifest.txt'. See also `legal.txt' for additional
% information.
% 
% The list of derived (unpacked) files belonging to the distribution 
% and covered by LPPL is defined by the unpacking scripts (with 
% extension .ins) which are part of the distribution.
% 
% \fi
%
% \iffalse
%%% From File: ltthm.dtx
%<*driver>
% \fi
\ProvidesFile{ltthm.dtx}
             [1995/10/10 v1.0f LaTeX Kernel (Theorems)]
% \iffalse
\documentclass{ltxdoc}
\GetFileInfo{ltthm.dtx}
\title{\filename}
\date{\filedate}
 \author{%
  Johannes Braams\and
  David Carlisle\and
  Alan Jeffrey\and
  Leslie Lamport\and
  Frank Mittelbach\and
  Chris Rowley\and
  Rainer Sch\"opf}

\begin{document}
\maketitle
 \DocInput{\filename}
\end{document}
%</driver>
% \fi
%
% \CheckSum{115}
%
% \section{Theorem Environments}
%
%
%  The user creates his own theorem-like environments with the command\\
%      |\newtheorem|\marg{name}\marg{text}\oarg{counter}  or\\
%      |\newtheorem|\marg{name}\oarg{oldname}\marg{text}\\
%  This defines the environment \meta{name} to be just as one would
%  expect a theorem environment to be, except that it prints \meta{text}
%  instead of ``Theorem''.
%
%  If \meta{oldname} is given, then environments \meta{name} and
%  \meta{oldname} use the same counter, so using a \meta{name}
%  environment advances the number of the next \meta{name} environment,
%  and vice-versa.
%
%  If \meta{counter} is given, then environment \meta{name} is numbered
%  within \meta{counter}.
% 
%  E.g., if \meta{counter} = |subsection|, then the first \meta{name} in
%  subsection 7.2 is numbered \meta{text} 7.2.1.
%
%  The way \meta{name} environments are numbered can be changed by
%  redefining |\the|\meta{name}.
%
% \StopEventually{}
%
%
% \changes{v1.0a}{1994/03/28}{Initial version, split from latex.dtx}
% \changes{v1.0c}{1994/05/25}{Modify documentation}
% \changes{v1.0f}{1995/10/10}{Make \cs{newtheorem} `only preamble'}
% \changes{v1.0g}{1995/10/16}
%      {Revert to previous \cs{newtheorem} behaviour}
%
% \begin{oldcomments}
%
%  DOCUMENT STYLE PARAMETERS
%
%  \@thmcounter{COUNTER} : A command such that
%               \edef\theCOUNTER{\@thmcounter{COUNTER}}
%     defines \theCOUNTER to produce a number for a theorem environment.
%     The default is:
%            BEGIN \noexpand\arabic{COUNTER} END
%
%  \@thmcountersep : A separator placed between a theorem number and
%         the number of the counter within which it is numbered.
%         E.g., to make the third theorem of section 7.2 be numbered
%         7.2-3, \@thmcountersep should be \def'ed to '-'.  Its
%         default is '.'.
%
%  \@begintheorem{NAME}{NUMBER} : A command that begins a theorem
%         environment for a 'theorem' named 'NAME NUMBER' --
%         e.g., \@begintheorem{Lemma}{3.7} starts Lemma 3.7.
%
%  \@opargbegintheorem{NAME}{NUMBER}{OPARG} : 
%         A command that begins a theorem
%         environment for a 'theorem' named 'NAME NUMBER' with optional
%         argument OPARG -- e.g., \@begintheorem{Lemma}{3.7}{Jones}
%         starts `Lemma 3.7 (Jones):'.
%
%  \@endtheorem : A command that ends a theorem environment.
%
% \newtheorem{NAME}{TEXT}[COUNTER] ==
%   BEGIN
%     if \NAME is definable
%       then \@definecounter{NAME}
%            if COUNTER present
%              then \@newctr{NAME}[COUNTER] fi
%                   \theNAME ==  BEGIN \theCOUNTER \@thmcountersep
%                                       eval\@thmcounter{NAME}      END
%              else \theNAME ==  BEGIN eval\@thmcounter{NAME} END
%            \NAME == \@thm{NAME}{TEXT}
%            \endNAME == \@endtheorem
%       else  error
%     fi
%   END
%
% \newtheorem{NAME}[OLDNAME]{TEXT}==
%   BEGIN
%     if counter OLDNAME nonexistant
%       then ERROR
%       else
%            if \NAME is definable
%              then BEGIN
%                   \theNAME == \theOLDNAME
%                   \NAME == \@thm{OLDNAME}{TEXT}
%                   \endNAME == \@endtheorem
%                   END
%              else  error
%            fi
%     fi
%   END
%
% \@thm{NAME}{TEXT} ==
%   BEGIN
%    \refstepcounter{NAME}
%    if next char = [
%       then \@ythm{NAME}{TEXT}
%       else \@xthm{NAME}{TEXT}
%    fi
%   END
%
% \@xthm{NAME}{TEXT} ==
%   BEGIN
%    \@begintheorem{TEXT}{\theNAME}
%    \ignorespaces
%   END
%
% \@ythm{NAME}{TEXT}[OPARG] ==
%   BEGIN
%    \@opargbegintheorem{TEXT}{\theNAME}{OPARG}
%    \ignorespaces
%   END
% \end{oldcomments}
%
% \begin{macro}{\newtheorem}
% |\newtheorem| ought really be allowed only in the preamble
% Which would be good document style, and allow some main memory to be
% saved by declaring these commands to be
% |\@onlypreamble|. Unfortunately the \LaTeX\ book indicates that
% |\newtheorem| may be used anywhere in  the document\ldots
%    \begin{macrocode}
%<*2ekernel>
\def\newtheorem#1{%
  \@ifnextchar[{\@othm{#1}}{\@nthm{#1}}}
%    \end{macrocode}
% \end{macro}
%
% \begin{macro}{\@nthm}
%    \begin{macrocode}
\def\@nthm#1#2{%
  \@ifnextchar[{\@xnthm{#1}{#2}}{\@ynthm{#1}{#2}}}
%    \end{macrocode}
% \end{macro}
%
% \begin{macro}{\@xnthm}
%
% 92/09/18 RmS: Changed |\@addtoreset| to |\@newctr| to produce error
%               message if counter |#3| does not exist (to be
%               consistent with behaviour of |\newcounter|)
%    \begin{macrocode}
\def\@xnthm#1#2[#3]{%
  \expandafter\@ifdefinable\csname #1\endcsname
    {\@definecounter{#1}\@newctr{#1}[#3]%
     \expandafter\xdef\csname the#1\endcsname{%
       \expandafter\noexpand\csname the#3\endcsname \@thmcountersep 
          \@thmcounter{#1}}%
     \global\@namedef{#1}{\@thm{#1}{#2}}%
     \global\@namedef{end#1}{\@endtheorem}}}
%    \end{macrocode}
% \end{macro}
%
% \begin{macro}{\@ynthm}
%    \begin{macrocode}
\def\@ynthm#1#2{%
  \expandafter\@ifdefinable\csname #1\endcsname
    {\@definecounter{#1}%
     \expandafter\xdef\csname the#1\endcsname{\@thmcounter{#1}}%
     \global\@namedef{#1}{\@thm{#1}{#2}}%
     \global\@namedef{end#1}{\@endtheorem}}}
%    \end{macrocode}
% \end{macro}
%
% \begin{macro}{\@othm}
% \changes{LaTeX2.09}{1992/01/10}
%         {(RmS) Check for existence of theorem environment}
% \changes{LaTeX2.09}{1992/08/19}
%         {(RmS) Changed error message to complain about undefined
%         counter}
% \changes{v1.0b}{1994/04/09}{Use standard counter error message (FMi)}
% \changes{v1.0c}{1994/04/17}{Use new std counter error message (FMi)}
%    \begin{macrocode}
\def\@othm#1[#2]#3{%
  \@ifundefined{c@#2}{\@nocounterr{#2}}%
    {\expandafter\@ifdefinable\csname #1\endcsname
    {\global\@namedef{the#1}{\@nameuse{the#2}}%
  \global\@namedef{#1}{\@thm{#2}{#3}}%
  \global\@namedef{end#1}{\@endtheorem}}}}
%    \end{macrocode}
% \end{macro}
%
% \begin{macro}{\@thm}
%    \begin{macrocode}
\def\@thm#1#2{%
  \refstepcounter{#1}%
  \@ifnextchar[{\@ythm{#1}{#2}}{\@xthm{#1}{#2}}}
%    \end{macrocode}
% \end{macro}
%
% \begin{macro}{\@xthm}
% \begin{macro}{\@ythm}
%    \begin{macrocode}
\def\@xthm#1#2{%
  \@begintheorem{#2}{\csname the#1\endcsname}\ignorespaces}
\def\@ythm#1#2[#3]{%
  \@opargbegintheorem{#2}{\csname the#1\endcsname}{#3}\ignorespaces}
%    \end{macrocode}
% \end{macro}
% \end{macro}
%
% Default values
% \begin{macro}{\@thmcounter}
% \begin{macro}{\@thmcountersep}
%    \begin{macrocode}
\def\@thmcounter#1{\noexpand\arabic{#1}}
\def\@thmcountersep{.}
%    \end{macrocode}
% \end{macro}
% \end{macro}
%
%  \begin{macro}{\@begintheorem}
%  \begin{macro}{\@opargbegintheorem}
%  \begin{macro}{\@endtheorem}
%  \changes{LaTeX2.09}{1991/08/14}
%         {Moved \cs{itshape} after \cs{item} to make it work with
%         NFSS}
%    Providing theorem defaults.
%  \task{???}{add `reset@font?}
%    \begin{macrocode}
\def\@begintheorem#1#2{\trivlist 
   \item[\hskip \labelsep{\bfseries #1\ #2}]\itshape}
\def\@opargbegintheorem#1#2#3{\trivlist
      \item[\hskip \labelsep{\bfseries #1\ #2\ (#3)}]\itshape}
\def\@endtheorem{\endtrivlist}
%</2ekernel>
%    \end{macrocode}
%  \end{macro}
%  \end{macro}
%  \end{macro}
%
% \Finale
%
