% \iffalse meta-comment
%
% Copyright 1993 1994 1995 1996 1997 1998 1999 2000 2001 2002 2003 2004 2005 2006
% The LaTeX3 Project and any individual authors listed elsewhere
% in this file. 
% 
% This file is part of the LaTeX base system.
% -------------------------------------------
% 
% It may be distributed and/or modified under the
% conditions of the LaTeX Project Public License, either version 1.3c
% of this license or (at your option) any later version.
% The latest version of this license is in
%    http://www.latex-project.org/lppl.txt
% and version 1.3c or later is part of all distributions of LaTeX 
% version 2005/12/01 or later.
% 
% This file has the LPPL maintenance status "maintained".
% 
% The list of all files belonging to the LaTeX base distribution is
% given in the file `manifest.txt'. See also `legal.txt' for additional
% information.
% 
% The list of derived (unpacked) files belonging to the distribution 
% and covered by LPPL is defined by the unpacking scripts (with 
% extension .ins) which are part of the distribution.
% 
% \fi
% Filename: ltnews11.tex 01/06/1999
% This is issue 11 of LaTeX News.

\documentclass
%    [lw35fonts]     % uncomment this line to get Times
   {ltnews}

% \usepackage[T1]{fontenc}

\publicationmonth{June}
\publicationyear{1999}
\publicationissue{11}

\begin{document}

\maketitle

\section{Back in sync}

The last release of \LaTeX{} was delayed even longer than you have
come to expect.  We hope that it proved worth waiting for.  It
required a major integration of the code from several people and,
independently, the introduction of the LPPL (see \LaTeX{} News~10) plus
several related changes to our internal systems.  It therefore seemed
sensible to wait until everything was complete rather than do things
in too much hurry.

This seem to have been a successful strategy as
the recent patch release was related to an isolated change that was
done many months previously.  If this release does not appear a lot
closer to its nominal date then \ldots~well, you will not be reading
this sentence!

\section{Yearly release cycles}

With the year 2000 rapidly approaching, we intend to switch to a
release frequency of just one per year (with patches if necessary) for
the core of \LaTeXe{}. These days the system is sufficiently stable
that the original update policy is costing everybody more time than is
now warranted.

\section{LPPL update}

Thanks to extensive and valuable input from Matt Swift
(\email{swift@alum.mit.edu}) we now have a clearer and more detailed
form of the \LaTeX{} Project Public Licence.  This release contains
both the original version (in \file{lppl-1-0.txt}) and the updated
version, LPPL~1.1.

\section{The future of Sli\TeX{}}

We still get a very small trickle of reports about this part of the
system (if you are no longer able to recall \LaTeX~2.09 then you will
know it as the \class{slides} class).  We have not classified them (in
our minds at least) as bugs since we have always known that there are
many problems with this class.  It is clear to us that the only
sensible action would be to redesign the system completely; in
particular, to remove much of its complexity whose purpose is to
support 10-year-old overlay technology.  However, this would take a
lot too much time and would be completely out of proportion to its
current usage.

We are therefore planning to make the \class{slides} class
unsupported in the sense that any problem related to the use of
invisible fonts is considered to be a feature (The \LaTeXe{} manual by
Leslie Lamport doesn't even describe this part of the class any more).
Of course, if it still has its enthusiasts then we are happy to cede
it to their loving care (somewhat like a preserved steam locomotive,
in some parts of the world).

\section{Fontenc package peculiarities}

The \verb=\usepackage= interface normally ensures that a package is
loaded only once.  The \pkg{fontenc} package has become an
exception to this rule: it can be loaded several times using different
options, e.g., allowing the user to add a font encoding in the
preamble. This comes at a price for package writers: the low-level
commands (see \file{ltclass.dtx}) used to check if a package was
loaded, and with which options, do not work for the \pkg{fontenc}
package.

\section{New math font encodings}
    
As we announced in \LaTeX{} News~9, a joint working group of the
\TeX{} Users Group and the \LaTeX3 Project has developed a new
\mbox{8-bit} math font encoding for \TeX{}.
The reason why this work is not yet released is because of other
exciting developments in the world of math fonts and math characters.
It is obviously wise to ensure that the encoding work is fully
integrated with the available fonts.

Those interested are reminded that further information about the Math
Font Group may be found on the World Wide Web at:\\
\url{http://www.tug.org/twg/mfg/}.

\section{Tools distribution}

The \pkg{multicol} package has now got a small but useful extension
which allows you to force a column break where this is really
necessary.  This is done with the command \verb=\columnbreak=, which
can be used like \verb=\pagebreak= (e.g.,~within paragraphs) except
that it cannot have an optional argument and thus it always forces a
new column.

\section{Coming soon}

Major work on a new class file structure to support flexible
designs is well under way; some of this work will be presented at the
TUG'99 conference in Vancouver, Canada.  With a bit of luck much of
this work could be ready for integration into the next release---so
watch this space!

\end{document}
