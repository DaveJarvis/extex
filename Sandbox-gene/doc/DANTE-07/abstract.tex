%%*****************************************************************************
%% $Id: abstract.tex,v 0.00 2007/01/14 19:45:08 gene Exp $
%%*****************************************************************************
%% Author: Gerd Neugebauer
%%-----------------------------------------------------------------------------
\documentclass{scrartcl}

\usepackage[latin1]{inputenc}
\usepackage{german}

\providecommand*{\ExTeX}{\ifx\texorpdfstring\undefined
  \textrm{% the logo comes always with serifs
    \ensuremath{\textstyle\varepsilon_{\kern-0.15em\mathcal{X}}}%
    \kern-.15em\TeX}%
  \else\texorpdfstring{%
  \textrm{% the logo comes always with serifs
    \ensuremath{\textstyle\varepsilon_{\kern-0.15em\mathcal{X}}}%
    \kern-.15em\TeX
  }}{ExTeX}%
  \fi
}

\begin{document}%%%%%%%%%%%%%%%%%%%%%%%%%%%%%%%%%%%%%%%%%%%%%%%%%%%%%%%%%%%%%%%

\title{Wie geht es eigentlich\ldots\ {\boldmath\ExTeX}?}
\author{Gerd Neugebauer}
%\date{}
\maketitle

\section*{Question \& Answer Session}

\ExTeX\ wurde als Open-Spource-Projekt gestartet, um \TeX\ in einer
Reinkarnation f�r das 21.~Jahrhundert neu erstehen zu lassen. Seit dem
Start des Projektes wurde eine weiter Weg zur�ckgelegt. Trotzdem ist
\ExTeX\ noch nicht f�r einen produktiven Einsatz bereit.

Mit \ExTeX\ werden einige zentralen Ziele verfolgt:

\begin{itemize}
\item \ExTeX\ soll in einem Basis-Modus weitestgehend kompatibel zu
  \TeX\ sein. 
\item \ExTeX\ soll konfigurierbar und erweiterbar sein.
\item \ExTeX\ soll neben dem Kommandozeilen-Programm auch eine
  Schnittstelle zur Nutzung als Bibliothek enthalten.
\end{itemize}

Dabei kann aus den Erfahrungen aus anderen Systemen zur�ckgrgriffen
werden:

\begin{itemize}
\item $\varepsilon$-\TeX
\item pdf\TeX
\item $\Omega$
\end{itemize}

Diese haben bereits partiell Eingang in die Entwicklung von \ExTeX\
gefunden. Einige wesentliche interne Strukturen wurden darauf
ausgerichtet, die \TeX-Erweiterungen der genannten Systeme zu
unterst�tzen. 

In dieser Fragestunde kann ein Einblick in den Stand der Entwicklung
gewonnen werden. An Hand einer praktischen Demonstration k�nnen einige
fertiggestellte Eigenschaften gezeigt werden.


\end{document}%%%%%%%%%%%%%%%%%%%%%%%%%%%%%%%%%%%%%%%%%%%%%%%%%%%%%%%%%%%%%%%%%
%
% Local Variables: 
% mode: latex
% TeX-master: nil
% End: 
